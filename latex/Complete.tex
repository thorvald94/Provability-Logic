%! TeX program = lualatex

\documentclass[../main.tex]{subfiles}

\begin{document}
In this section we will prove Solovay's completeness theorems. The proof of
these theorems follows a technique invented by Robert Solovay, which today is
known as \textit{Solovay construction}. This technique is a way of embedding
Kripke models into arithmetic

\section{The First Theorem}

\begin{lem}
	For all modal sentences $\varphi$ we have that :
	$$GL\vdash\varphi\Rightarrow\forall ^*(\PRA\vdash\varphi^*)$$
\end{lem}
\begin{proof}
	The proof will be done as an induction proof.
\end{proof}
\begin{thm}[Solovay's first completeness theorem]
	For all modal sentences $\varphi$ we have that:
	$$\forall ^* (\PRA\vdash \varphi^*)\Rightarrow \GL\vdash \varphi$$
\end{thm}
This theorem will be shown by contraposition: If $\GL\not\vdash \varphi$ then we
have one $^*$ such that $\PRA\vdash\varphi^*$. This will be a rather complex
construction, and will take the rest of this subsection to prove. So for the
rest of this subsection fix a sentence $\varphi$ such that $GL\vdash \varphi$
and let $\ol{K}=(W,R,\phi)$ be a Kripke model such that $w_0\vdash\varphi$.

We will assume that $K=\{0,\ldots, n\}$ for some finite $n$ and that $w_0=1$.
$R$ can be extended by setting $0Ri$ for each $i$ in $W$. Is shall be noted
that $0$ is not part of our Kripke model.

Further we will first intuitively define a function $F\omega\rightarrow\{0,\ldots, n\}$ in the
following way: Set $F(0)=0$. Further we define $F(x+1)$ in the following way:
If $x+1$ is the code of a proof that $\lim_{k\rightarrow\infty}(k)\not =z$ for
some $z$ accessible to $F(x)$ we set $F(x+1)=z$ otherwise we have that
$F(x+1)=F(x)$.

The way this function works can be explained intuitively by the following
quote:
\begin{displayquote}
	Imagine a refugee who is admitted from one country to another only if
	he/she provides a proof not to stay there forever. If the refugee is
	also never allowed to go to one of the previously visited countries,
	he/she must eventually stop somewhere. So, an honest refugee will never
	be able to leave his/her country of origin. [Beklemeishev og artemov,
	find biktex reference]
\end{displayquote}

Before we can give a formal definition of the function $F$, we will have to do
some work, and introduce some notation. First of we will let $G$ be a partial
recursive function with the following $\Sigma_1$ graph: $\psi v_0v_1$. From
this graph we can obtain the following $\Sigma_2$ formula:
$$\exists v_0\forall v_1>v_0:\psi v_1v$$

Which says that $G$ has limit $v$. We will abbreviate this as $L=v$. We will
use this notation to define $F$ in the following way:
$$F(v_0)=v_1\leftrightarrow\begin{cases}
	(v_0=\0\wedge v_1=\0)\ \vee \\
	(v_0>\0\wedge Prov(v_0,\Godelnum{L\not =\0})\wedge F(v_0-\ol 1)\ol
	Rv_2)\ \vee\\
	(v_0>\0\wedge \forall v_2\leq v_0\neg(Prov(v_0,\Godelnum{\not
	=v_2})\wedge\ol{F}(v_0-\ol{1})\ol{R}v_2\ \wedge\\
	v_1=\ol{F}(v_0-\ol{1}))
	\\
	\end{cases}
$$
\textbf{Mere formelt defi:}

Thus we have that $F$ have the following properties:
\begin{enumerate}
	\item $F(0)=0$
	\item If $x+1$ proves that $L\not =\ol z$ and we have that $F(x)Rz$,
		then $F(x+1)=z$
	\item Else we have that $F(x+1)=F(x)$
\end{enumerate}
By the Recursion theorem this function exists. Further it is a total function
since it is defined by recursion.

Further we will expand the langauge with a new function constant (This can be
done, since $\psi$ is the graph of a total function) $\ol F$ with the following
defining axiom:
$$\ol F(v_0)=v_1\leftrightarrow\psi (v_0)=v_1$$
We will now prove and state a few lemmas before proving the main theorem.


\begin{lem}
	\label{lem:2}
	The following three statements holds:
	\begin{enumerate}
		\item $\PRA\vdash\forall v_0(\ol Fv_0\leq \ol n)$
		\item For all $x\in\omega$ we have that:\ $$\PRA\vdash\forall
			v_o(\ol
			Fv_0=\ol x\rightarrow\forall v_1>v_0(\ol Fv_1=\ol x\vee
			\ol x\ol R\ol Fv_1))$$
		\item $\PRA\vdash\exists v_0v_1\forall v_2>v_0(\ol Fv_2=v_1)$. 
	\end{enumerate}
	Where (3) just means that $\PRA\vdash\exists v_1(L=v_1)$.
\end{lem}
\begin{proof}
	\textbf{Dette bevis er ikke helt "færdigt". Færdiggør det selv senere}

	We will prove each part separately 

	\begin{enumerate}

		\item We will prove this part by induction.

			\textbf{Base case:} $F(0)=0\leq n$ is is clearly true.

			\textbf{Induction step:} Assume that $F(x)\leq n$ is
			true.Then we have that $F(x+1)$ is in the range of $F$
			and this $\leq n$ or we have that $F(x+1)\leq n$ so the
			induction is complete.
		\item We will start of by write the formula we have to prove as
			the following equivalent formula:
			$$\forall v_1\forall v_0(\ol{F}v_0=\ol x\rightarrow
			F(v_0 + v_1+\ol 1)=\ol x\vee \ol x\ol R\ol
			F(v_0+v_1+1))$$
			To prove this we will use induction on $v_1$. This is
			an induction on a $\Pi_1$ formula; which is a possible
			induction for \PRA\ by [Ref] 

			\textbf{Base case} if $v_1=0$ then we have that 

			\textbf{Induction step} bla

			\textbf{Kom tilabe til dette sted}

		\item Here we will first prove the following:
			$$\forall v_0(\exists v_1(\ol
			Fv_1=v_0)\rightarrow\exists v_1(L=v_1))$$
			This is clearly true for $v_0\leq n$. \textbf{Kig på
			dette senere}.
\end{enumerate}
\end{proof}
\textbf{Kommenter på Inductionen brugt ovenover.}
\begin{cor}
	$\PRA\vdash L\leq \ol n$, i.e $\PRA\vdash \bigvee_{x\leq n} L=\ol x$
\end{cor}
\begin{proof}
	By lemma \ref{lem:2} 1) and 3) and the following implication:
	$$\PRA\vdash v \leq \ol n\rightarrow \bigvee_{x\leq n} v=\ol x$$
	The corollary follows.
\end{proof}
\begin{lem}
	\label{lem:4}
	For all $x,y\leq n$ we have that:
	\begin{enumerate}
		\item $L=\ol x\wedge \ol x \ol R\ol y\rightarrow
			\Con_{\PRA+L=\ol y}$
		\item $\PRA\vdash L=\ol x\wedge \ol x\not = \ol y\wedge \neq
			\ol x \ol R\ol y\rightarrow\neg \Con_{\PRA+L=\ol y}$
		\item $\PRA\vdash L=\ol x\wedge \ol x >\0\rightarrow
			\Pr(\Godelnum{L=\neg x})$
	\end{enumerate}

\end{lem}
\begin{proof}
		We will prove each statement separately:

		\textbf{1)} Let $xRy$ and assume for contradiction that $L
			=\ol x\wedge \text{Pr}(\Godelnum{L\not =\ol y})$. Since we
			have that $L=\ol x$ we can chose a $v_0$ such that
			$\forall v_2(v_2>v_0\rightarrow \ol Fv_2=\ol x)$ and we
			can also chose $v_1+\ol 1>v_0$ such that
			$\text{Prov}(v_1+\ol 1,\Godelnum{L\not =\ol y})$
			But we also have the following:
			$$\PRA\vdash\text{Prov}(v_1+\ol 1,\Godelnum{L\not
			=y})\wedge\ol Fv_1=\ol \x\wedge \ol x\ol R\ol
			y\rightarrow\ol F(v_1+\ol 1)=\ol y$$
			This contradicts with $\forall v_2>v_0(\ol Fv_0=\ol x)$
			which came from the assumption that $L=\ol x$ so we con
			conclude:
			$$\PRA\vdash L=\ol x\wedge\ol x\ol R\ol
			y\rightarrow\neg\text{Pr}(\Godelnum{L\not =\ol y})$$
			Where we have that $\neg\text{Pr}(L\not =\ol y)$ is
			equivalent to $\text{Con}_{\PRA+L=\ol y}$ and 1)
			follows.

			\textbf{2)}
		  By [ref???] We have the following deduction:
			\begin{align}
				\PRA\vdash L=\ol x&\rightarrow \exists v_0(\ol
				Fv_0=\ol x)\\
						  &\rightarrow\text{Pr}(\Godelnum{
						  \exists v_0(\ol
					  Fv_0=\ol x)})\label{eq:21}
			\end{align} 
			Further we have from lemma \ref{lem:2}.2 and [D2, lob?]  that:
			\begin{align}
				\PRA&\vdash\forall v_0(\ol Fv_0=\ol
				x\rightarrow(L=\ol x\vee \ol x\ol RL))\\
				\PRA&\vdash(\Godelnum{\forall v_0(\ol Fv_0=\ol
					x\rightarrow (L=\ol x\vee \ol x\ol
				RL))})\label{eq:22}
			\end{align}
			From \ref{eq:21} and \ref{eq:22} we have the following:
			\begin{equation}
				\label{eq:23}
				\PRA\vdash L=\ol x
				\rightarrow\text{Pr}(\Godelnum{L=\ol x\vee \ol
				x\ol RL})
			\end{equation}
			asd?

			Which gives us [Why?]
			$$\PRA\vdash \ol x\not =\ol y\wedge \neg (\ol x\ol R\ol
			y)\rightarrow \text{Pr}(\Godelnum{\ol x\not =y\wedge
			\neg (\ol x\ol R\ol y)})$$
			This with \ref{eq:23} gives us:
			
			$$
				\PRA\vdash L=\ol x\wedge \ol x\not =\ol y\wedge
			\neg\ol x\ol R\ol
			y\rightarrow\text{Pr}(\Godelnum{L=\ol x\vee \ol x\ol
			RL})\wedge \text{Pr}(\Godelnum{\ol x\not =\ol y\wedge
			\neg\ol x\ol  R\ol y})$$
			From which we can deduce:
			$$\PRA\vdash L\ol x\wedge \ol x\not =\ol y\wedge \neg
			\ol x\ol R\ol y\rightarrow \text{Pr}(\Godelnum{L\not
			=\ol y})$$
			Which gives us 2)

			\textbf{3)} From the least number princible we have
			that:
			$$\PRA\vdash L=\ol x\wedge\ol x>\ol 0\rightarrow
			\exists v(\ol F(v+\ol 1)=\ol x\wedge \ol Fv\not =\ol
			x)$$
			By the definition of $F$ we have for such a $v$ the
			following:
			$$\PRA\vdash \ol F(v+\ol 1)=\ol x\ol Fv\not =\ol
			x\rightarrow \text{Prov}(v+\ol 1,\Godelnum{L\not =\ol
			x})$$
			And thus we have the following:
			$$\PRA\vdash L=\ol x\wedge \ol x> \ol 0\rightarrow
			\text{Pr}(\Godelnum{L\not =\ol x})$$
\end{proof}
Lemma \ref{lem:2} and \ref{lem:4} gives us the most of the basic fact we need
about $L$ and $F$; at least the facts about them that we can prove in $\PRA$.
We will also need the following result, which cannot be proven i $\PRA$.
\begin{lem}
	\label{lem:5}
	The following two statements are true, but they cannot be proven in
	\PRA.
	\begin{enumerate}
		\item $L=\0$
		\item For $0\leq x\leq n$ we have that $\PRA+L=\ol x$ is
			consistent.
	\end{enumerate}
\end{lem}
\begin{proof}
	\textbf{1)} By lemma \ref{lem:2}.3 the limit $L$ exists. If $x>0$ we have by lemma
	\ref{lem:5} that:
	\begin{align*}
		L=\ol x &\Rightarrow \PRA\vdash L\not =\ol x\\
		     &\Rightarrow L\not = \ol x
	\end{align*}
	Since \PRA\ is sound. But this is a contradiction and we must conclude
	that $L=0$.

	\textbf{2)} Since we have that $L=\ol 0$ is true and we have that \PRA\
	is sound, we have  that $\PRA+L=\ol 0$ is consistent. For $x>0$ we will
	apply lemma \ref{lem:5}.1 and get:
	$$\PRA\vdash L=\ol 0\wedge \ol 0\ol R\ol x\rightarrow
	\text{Con}_{\PRA+L=\ol x}$$
	We have that the antecedent is true and hence that
	$\text{Con}_{PRA+L=\ol x}$ is true, which proves this part of the
	lemma.
\end{proof}

We have now shown all the important basic properties that $F$ and $L$ holds. In
the next part of the proof we will simulate the Kripke model $\mathcal{K}=\la
W,R,\phi\ra$, where $\mathcal{K}_1\not\vdash \varphi$. For this end we will let $L=\ol x$ assume the nodes of
$W=\{1,\ldots n\}$. We will start of by defining the interpretation $^*$.We
will start of by defining the interpretation $^*$.  So for any $p$ let:
$$p^*=\bigvee\{L=\ol x:1\leq x\leq n\ \text{and}\ x\in \phi(p)\}$$
If this disjunction is empty, we will set it to be $\ol 0=\ol 1$. Further we
are only interested the sentence $\varphi$; i.e the set:
$$S(\varphi)=\{\psi:\ \psi\ \text{is a subformula of }\ \varphi\}$$
We will not look at $p\not\in S(\varphi)$

The following lemma is the crucial result about this interpretation, and the
theorem will follow easily from this result:
\begin{lem}
	\label{lem:10}
	Let $1\leq x\leq n$. For any $\psi$ and $^*$ as defined just above we
	have that:
	\begin{enumerate}
		\item $x\vdash \psi\Rightarrow\PRA\vdash L=\ol x\rightarrow
			\psi^*$
		\item $x\not\vdash\psi\Rightarrow\PRA\vdash L=\ol
			x\rightarrow \neg \psi^*$
	\end{enumerate}
\end{lem}
\begin{proof}
We will use induction on complexity of $\psi$. If $\psi=p$, then since $p$ is a
disjunct of $p^*$ we get:
$$\vdash_x\varphi\Rightarrow\PRA\vdash L=\ol x\rightarrow p^*$$
Which proves 1, for $\psi$ atomic. For 2 observe that if $\not\vdash_x p$, then
$L=\ol x$ contradicts over disjunct of $p^*$ and thus:
$$\not\vdash_x p\rightarrow\PRA\vdash L=\ol x\rightarrow\neg p^*$$
The cases where  $\psi$ is $\neg\theta,\theta\wedge\sigma,\theta\vee\sigma$ and
$\theta\rightarrow\sigma$ are trivial. So we will just look at the case where
$\psi=\Box\theta$. Here we make the following deductions:
\begin{align*}
	\vdash_x\Box\theta&\Rightarrow \forall
	y(xRy\Rightarrow\vdash_y\theta)\\
			  &\Rightarrow \forall(xR\Rightarrow\PRA\vdash L=\ol
			  y\rightarrow\theta^*)\\
			  &\Rightarrow\bigwedge_{xRy}(\PRA\vdash L=\ol
			  y\rightarrow\theta^*)\\
			  &\Rightarrow\PRA\vdash\bigvee_{xRy} L=\ol
			  y\rightarrow \theta^*\\
			  &\Rightarrow\PRA\vdash\text{Pr}(\Godelnum{\bigvee_{xRy} L=\ol
			  y})\rightarrow\text{Pr}(\Godelnum{\theta^*})
\end{align*}
And by the last line we can get the following by using the axioms of \GL
\begin{equation}
	\label{eq:1}
			 \vdash_x\Box\theta\Rightarrow \PRA\vdash\text{Pr}(\Godelnum{\bigvee_{xRy} L=\ol
			  y})\rightarrow\text{Pr}(\Godelnum{\theta^*})
\end{equation}
We can now invoke lemma \ref{lem:5} 2 and 3 and get:
\begin{equation}
	\label{eq:2}
	\PRA\vdash L=\ol
	x\rightarrow\bigwedge_{\neg(xRz)}\text{Pr}(\Godelnum{L\not =z})
\end{equation}
and therfore [Hvorfor?]
\begin{equation}
	\label{eq:3}
	\PRA\vdash L=\ol x\rightarrow \text{Pr}(\Godelnum{\bigvee_{xRy}L=\ol y})
\end{equation}
Now by \ref{eq:1} and \ref{eq:3} we get that:
\begin{align*}
	\vdash_x\Box\theta&\Rightarrow\PRA\vdash L=\ol x\rightarrow
	\text{Pr(\Godelnum{\theta^*})}\\
			  &\Rightarrow\PRA\vdash L=\ol x\rightarrow
			  (\Box\theta)^*
\end{align*}
I.e we have proven part 1 of the lemma. Similarly we can do the following
deduction:
\begin{align*}
	\not\vdash_x\Box\theta&\Rightarrow\exists y(xRy\wedge
	\not\vdash_y\theta)\\
			      &\Rightarrow \exists y(xRy\wedge\PRA\vdash L=\ol
			      y\rightarrow\neg\theta^*)\\
			      &\Rightarrow\exists y(xRy\wedge\PRA\vdash
			      \theta^*\rightarrow L \not =\ol y)\\
			      &\Rightarrow\exists
			      y(xRy\wedge\PRA\vdash\text{Pr}(\Godelnum{
			      \theta^*})\rightarrow\text{Pr}(\Godelnum{L\not
		      =\ol y})
\end{align*}
But by lemma \ref{lem:5} we get that if $xRy$:
$$\PRA\vdash L=\ol x\rightarrow\neg \text{Pr}(\Godelnum{L\not =\ol y})$$
all in all this gives us:
$$\PRA\vdash L=\ol x\rightarrow\neg\text{Pr}(\Godelnum{\theta^*})$$
Which just is the following we where trying to show:
$$\PRA\vdash L=\ol x\rightarrow\neg(\Box\theta)^*$$
\end{proof}
We can now finally prove Solovay's first completeness theorem:
\begin{proof}[Proof of Solovay's first completeness theorem]
	By lemma \ref{lem:10} we have that
	$$\vdash_1\varphi\Rightarrow\PRA\vdash L=\ol 1\rightarrow\varphi^*$$
	But by lemma [Hvad?] we have that $\PRA\ +L=1$ is consistent, from
	which it follows that $\PRA+\neg\varphi^*$ is consistent; so $\varphi^*$
	is not a theorem af $\PRA$ and thus we have:
	$\PRA\not\vdash\varphi^*$ and the theorem follows by contraposition.
\end{proof}

\section{The Second Theorem}
Solovay's Second Completeness Theorem is a strengthening of the first one,
[How?]
\begin{thm}
	For all modal sentences $\varphi$, the following is equivalent:
	\begin{enumerate}
		\item $\textbf{GLS}\vdash\varphi$
		\item $\GL\vdash\displaystyle\bigwedge_{\Box \psi\in
			S(\varphi)}(\Box\psi\rightarrow\psi)\varphi$
		\item $\varphi$ is true in all $\varphi$-sound Kripke models
		\item $\forall^*(\varphi^*\ \text{is true})$
	\end{enumerate}
\end{thm}
Some parts of this theorem has already been proven. We have
$(1)\Leftrightarrow(2)$ by theorem [asd?] and we have $(2)\Leftrightarrow(3)$
by theorem [asd?]. \textbf{Udbyd dette senere}

For proving $(4)\Rightarrow(3)$ we will again make use of contraposition. Let
$\mathcal{\ol K}=((1,\ldots,n),R,\phi)$ be given and let $1$ be the root.
Further let $\varphi$ be such that $\not\vdash_1\varphi$.
We
will assume that $\mathcal{\ol K}$ is $\varphi$-sound. This means that we have:
$\vdash_1\Box\psi\rightarrow\psi$ for all $\psi\in S(\varphi)$

We will set $0Rw$ for all $w\in W$. We will now create a new model
$\mathcal{K'}$ where we have added $0$. We will create this new model in the
following way:
\begin{align*}
	&W'=\{0,1\ldots,n\}\\
	&R'\ \text{extends}\ R\ \text{by assuming that}\ 0R'x\ \text{for all}\
	x\in W\\
	&\alpha_0=0\\
	&\phi'\ \text{extends}\ \phi\ \text{by putting}\ 0\in\phi(p)\
	\text{iff}\ 1\in\phi(p)\ \text{for all}\ p\in S(\varphi)
\end{align*}
We will abuse notation and let $R$ denote $R'$ and $\phi$ denote $\phi'$.
\begin{lem}
	For all $\psi\in S(\varphi)$ we have that:
	$$0\vdash\psi\ \text{iff}\ 1\vdash \psi$$
\end{lem}
\begin{proof}
\end{proof}
\begin{lem}
	Let $0\leq x\leq n$. For any $\psi\in S(\varphi)$ and any $^*$ as
	defined we have that:
	\begin{enumerate}
		\item $x\vdash\psi\Rightarrow\PRA\vdash L=\ol
			x\rightarrow\psi^*$
		\item $x\not\vdash\varphi\Rightarrow \PRA\vdash \ol x\rightarrow
			\neg\psi^*$
	\end{enumerate}
\end{lem}
\begin{proof}
\end{proof}

We will now define a function $F$ in the same way as before.
\begin{align*}
	F(0)&=0\\
	F(x+1)&=\begin{cases}
		y &\ \text{Prov}(\ol x+\ol 1,\Godelnum{L\not =\ol y})\wedge
			xRy\\
			F(x) &\ \text{else}
		\end{cases}
\end{align*}
All the Lemmas about $F$ and $L$ still holds. But the behavior $\phi$ has
changed, since we must added the node  $0$ and we can only use sub formulas of
$\varphi$. So we define:
$$p^*=\bigvee\{L=\ol x: 0\leq x\leq n\wedge x\in\phi(p)\}$$
For $p\in S(\varphi)$ and let $p^*$ be random for all $p$'s that is not a
subformula of $\varphi$. We now prove and state the following lemma (that
reminds of what?):
\begin{lem}
	Let $0\leq x\leq n$. For any $\psi\in S(\varphi)$ and $^*$ as defined
	above we have:
	\begin{enumerate}
		\item $\vdash_x\psi\Rightarrow\PRA\vdash L=\ol
			x\rightarrow\psi^*$
		\item $\not\vdash_x\psi\Rightarrow\PRA\not\vdash L=\ol
			x\rightarrow\psi^*$
	\end{enumerate}
\end{lem}
\begin{proof}
	For $0<x$ the proof is identical to the proof of [ref?]. So we will
	only prove the case where $x=0$. The is again an induction on the
	complexity of $\psi$. We will only prove the cases where
	$\psi=\Box\theta$.

	Let $\psi=\Box\theta$. We then have:
	\begin{align*}
		\vdash_0\Box\theta&\Rightarrow\forall x(1\leq x\leq
		n\Rightarrow \vdash_x\theta)\\
				  &\Rightarrow \forall x(1\leq x\leq
				  n\Rightarrow \PRA\vdash L=\ol
				  x\rightarrow\theta^*)
	\end{align*}
	Since $x>1$ and this case of the lemma has been proven. We can also
	make the following deduction by the induction hypothesis:
	\begin{align*}
		\vdash_0\Box\theta&\Rightarrow\vdash_1\theta\\
				  &\Rightarrow\vdash_0\theta\\
				  &\Rightarrow\PRA\vdash L=\ol
				  0\rightarrow\theta^*
	\end{align*}
	By combing these two we get:
	\begin{align*}
		\vdash_0\Box\theta&\Rightarrow\bigwedge_{x\leq n}(\PRA\vdash
		L=\ol x\rightarrow\theta^*)\\
				  &\Rightarrow\PRA\vdash(\bigvee_{x\leq n}L=\ol
				  x)\rightarrow\theta^*\\
				  &=\PRA\vdash\text{Pr}(\Godelnum{\bigvee L=\ol
				  x})\rightarrow\text{Pr}(\Godelnum{\theta^*})
	\end{align*}
	But by corollary [<ref?] we have that $\PRA\vdash\bigvee L=\ol x$ and
	thus $\PRA\vdash\text{Pr}(\Godelnum{\bigvee L=\ol x})$ so all in all we
	have:
	\begin{align*}
		\vdash_0\Box\theta&\Rightarrow \PRA\vdash
		\text{Pr}(\Godelnum{\theta^*})
				  &\Rightarrow\PRA\vdash L=\ol
				  0\rightarrow\text{Pr}(\Godelnum{\theta^*})
	\end{align*}
	This proves (1). The proof of (2) is a bit easier. We have:
	\begin{align*}
		\not\vdash_0\Box\theta&\Rightarrow\exists x(1\leq x\leq n\wedge
		\not\vdash_x\theta)\\
				      &\Rightarrow\exists x(1\leq x\leq
				      n\wedge\PRA\vdash L=\ol x\rightarrow
				      \neg\theta^*)\\
				      &\Rightarrow\exists x(1\leq x\leq n\wedge
				      \PRA\vdash\theta^*\rightarrow L\not =\ol
				      x)\\
				      &\Rightarrow\PRA\vdash L=\ol 0\rightarrow
				      \neg \text{Pr}(\Godelnum{\theta^*})
	\end{align*}
	And thus by Lemma [Ref?] we have that $\PRA\vdash L=\ol
	0\rightarrow\neg\text{Pr}(\Godelnum{L\not =\ol x})$ for $x>0$
\end{proof}
We thus have that $L=\ol 0$ is true, and we  can now finally prove the second completeness theorem:
\begin{proof}[Proof of the Second Completeness theorem]
	Assume that $\varphi$ is false in $\mathcal{\ol K}$ i.e
	$\not\vdash_1\varphi$. Then by lemma [asd] we get $\not\vdash_0\varphi$
	and the just proven lemma gives:
	$$\PRA\vdash L=\ol 0\rightarrow\neg\varphi^*$$
	Since we have that $L=\ol 0$ is true we then get $\neg\varphi^*$ which
	just means that $\varphi^*$ is false.
\end{proof}
\subsection{Using the Second Completeness Theorem}
We can use the second completeness theorem to prove the following theorem named
after Rosser:
\begin{thm}
	There is a $\Sigma_1$-sentences $\varphi$ such that:
	\begin{enumerate}
		\item $\PRA\not\vdash\varphi$
		\item $\PRA\not\vdash\not\varphi$
		\item $\PRA\vdash \Con\rightarrow\neg \Pr(\Godelnum{\varphi})$
		\item $\PRA\vdash \Con\rightarrow\neg
			PPr(\Godelnum{\neg\varphi})$
	\end{enumerate}
\end{thm}
\section{Further Completeness Results}
\end{document}
