% TeX program = lualatex

\documentclass[../main.tex]{subfiles}

\begin{document}
In this section we will prove Solovay's completeness theorems. The proof of
these theorems follows a technique invented by Robert Solovay, which today is
known as a \textit{Solovay construction}. This technique is a way of embedding
Kripke models into arithmetic. The first theorem show that the $\Box$-operator of
the logic \GL\ behaves like the proof predicate $\text{Pr}$ from \PRA. The
second theorem shows that the modal logic \GLS. 

Further these theorems shows that the proof predicate of arithmetics can be
axiomatized, by the axioms of \GL\ and \GLS.

\section{Soundness}
For each  formula in $\mathcal{L}_\Box$ we want to assign a sentence of
$\mathcal{L}_\PRA$.
This can be done in the following way.

\begin{defi}
	An interpretation of $\mathcal{L}_\Box$ in \PRA\ is a function that to
	each formula $\alpha$ of $\mathcal{L}_\Box$ assigns a sentence
	$\alpha^*$ of \PRA\ which satisfies the following requirements:
	\begin{enumerate}
		\item For atomic $p$, $p^*$ is a formula of the language of
			arithmetic 
		\item $(\bot)^*="0=1"$
		\item $(\alpha\rightarrow\beta)^*="\alpha^*\rightarrow\beta^*"$
\item $(\Box\alpha)^*="\Pr(\Godelnum{\alpha^*})"$
	\end{enumerate}
	We will further say that a  modal formula $\alpha$ is \PRA-valid if, in every interpretation
	$^*$, $\alpha^*$ is a theorem of \PRA.
\end{defi}

The goal of the current section is to prove the that the set of \PRA-valid
formulas are the theorems of \GL. I.e we want to prove the following bi-implication:

\[\GL\vdash\alpha\Leftrightarrow \forall^*(\PRA\vdash\alpha^*)\]
This formula says that \GL\ completely captures what \PRA\ can say about its own
provability. This result can be expanded to other fragments of arithmetics. So
the modal logic \GL\ is the modal logic that captures what a lot of  different fragments of
arithmetics can say about its own provability. Later on in the project it will be shown for which fragments this
is the case. 

In \ref{chap:second} we will show that \GLS\ is the modal logic of provability
in truth. This is called Solovay's second completeness theorem.

We will start with proving the "$\Rightarrow$" implication; i.e that every
theorem of \GL\ is \PRA-valid. The other way will be a bit harder to prove, and
that is the part that is known as Solovay's first completeness theorem.
\begin{thm}[Soundness]
	For all modal sentences $\varphi$ we have that :
	$$\GL\vdash\varphi\Rightarrow\forall ^*(\PRA\vdash\varphi^*)$$
\end{thm}
\begin{proof}
	The proof will be done as an induction proof on the number of axioms
	and rules of inference use in a \GL-proof of a formula $\alpha$. The
	proof will be done by looking at the last rule or axiom schema used in
	the proof of $\alpha$.

	The case of A1 and R1 are clear. We further have that the theorems of
	\PRA\ are closed under modus ponens so A2 is also clear. 

	Assume that  the last step of the proof of $\alpha$ is an instance of
	R2. We have that $\Pr(x)$ is $\Sigma_1$ formula, so it is equivalent to
	some formula of the form $\exists y R(x,y)$, where $R$ is a primitive
	recursive predicate. We know that if a $\Sigma_1$ sentences is true it
	is provably, so this shows the case of R2.

	The axiom A3 can be derived from the others, so this case is redudant.

	We will now just have to show the case of A4. This case follows by
	Löb's theorem.
\end{proof}
\section{The First Theorem}
Having shown soundness, we will now show the completeness theorem:
\begin{thm}[Solovay's first completeness theorem]
	For all modal sentences $\varphi$ we have that:
	$$\forall ^* (\PRA\vdash \alpha^*)\Rightarrow \GL\vdash \alpha$$
\end{thm}

This theorem will be shown by contraposition. So we want to show the following:
"If $\GL\not\vdash \alpha$ then we
have one $^*$ such that $\PRA\not\vdash\alpha^*$". 
The start of the proof is the following: If $\GL\not\vdash\alpha$ then
$\GL\not\vDash\alpha$ and thus there is a finite pointed Kripke model $\mathcal{K}=\la
W,R,\phi,w_0\ra$ in which we have $w_0\not\in\phi(\alpha)$. The goal is then to
find an interpretation $^*$ such that $\PRA\not\vdash\alpha^*$.

The construction of this
$^*$ is rather complex, and will take the rest of this subsection to prove. So for the
rest of this subsection fix a sentence $\varphi$ such that $\GL\not\vdash \varphi$
and let $\mathcal{K}=\la W,R,\phi,w_0\ra$ be a pointed Kripke model such that $w_0\in\phi(\alpha)$.

We will assume without loss of generality  that $W=\{0,\ldots, n\}$ for some finite $n$ and that $w_0=1$.
$R$ can be extended by setting $0Ri$ for each $i$ in $W$. Is shall be noted
that $0$ is not part of our Kripke model, but we will in section
ref{chap:second} create a model where it is a part of the model.

Further we will first intuitively define a function $\varphi:\omega\rightarrow\{0,\ldots, n\}$ in the
following way: Set $\varphi(0)=0$. Further we define $\varphi(x+1)$ in the following way:
If $x+1$ is the code of a proof that $\lim_{k\rightarrow\infty}(k)\not =z$ for
some $z$ accessible to $\varphi(x)$ we set $\varphi(x+1)=z$ otherwise we have that
$\varphi(x+1)=\varphi(x)$.

The way this function works can be explained intuitively by the following
quote:
\begin{displayquote}
	Imagine a refugee who is admitted from one country to another only if
	he/she provides a proof not to stay there forever. If the refugee is
	also never allowed to go to one of the previously visited countries,
	he/she must eventually stop somewhere. So, an honest refugee will never
	be able to leave his/her country of origin. \parencite{ArteBe}
\end{displayquote}

Before we can give a formal definition of the function $\varphi$, we will have to
introduce some notation. First of we will let $\psi$ be an arbitrary  partial
recursive function with the following $\Sigma_1$ graph: $\tau v_0v_1$. From
this graph we can obtain the following $\Sigma_2$ formula:
$$\exists v_0\forall v_1>v_0:\tau v_1v$$

Which says that $\psi$ has limit $v$. We will abbreviate this as $L_\psi=v$. We will
use this notation to define $\varphi$ in the following way:
$$\varphi(v_0)=v_1\leftrightarrow\begin{cases}
	(v_0=\0\wedge v_1=\0)\ \vee \\
	(v_0>\0\wedge Prov(v_0,\Godelnum{L\not =\dot v_1})\wedge \varphi(v_0\dot -\ol 1)\ol
	Rv_2)\ \vee\\
	(v_0>\0\wedge \forall v_2\leq v_0\neg(Prov(v_0,\Godelnum{L\not
	=v_2})\wedge\ol{ \varphi}(v_0\dot-\ol{1})\ol{R}v_2\ \wedge\\
	v_1=\ol{\varphi}(v_0-\ol{1}))
	\\
	\end{cases}
$$
We will define the graph $\tau v_0v_1=\exists v_3\chi v_3v_0v_1$ of the partial
recursive function $\varphi$. Since we have that the graph of $\varphi$ is
$\Sigma_1$ we have that the graph $\tau$ is also $\Sigma_1$ and thus the graph
$\chi$ is $\Delta_0$. We will define a formula $\Phi(\chi)$ as the
disjunction of the following three formulas:
\begin{enumerate}
	\item $v_0=\0\wedge v_1=\0$
	\item $v_0>\0\wedge \text{Prov}(v_0,\Godelnum{L_\tau\not =v_1})\wedge\exists
		v_4(\tau(v_0\dot-\ol 1,v_4)\wedge v_4\ol Rv_1$)
\item  $v_0>\0 \wedge \exists v_3v_4\forall v_2\leq
	v_0\neq(\text{Prov}(v_0,\Godelnum{L_\tau\not =v_2})\wedge
	\chi(v_3,v_0-\ol 1,v_4)\wedge v_4\ol Rv_1)\\
	\wedge \tau(v_0-\ol 1,v_1)$
\end{enumerate}
Thus we have that the set $\Phi(\chi)$ can be seen as a primitive recursive
function of $\Godelnum{\chi}$; i.e we have:
\[\Godelnum{\Phi(\chi)}=\vartheta(\Godelnum{\chi})\]

Since primitive functions are recursive we have by the Recursion Theorem that
we can pick an $n$ such that $\varphi_{\vartheta(n)}=\varphi_n$. Now set
$\varphi=\varphi_n$. This definition of $\varphi$ is rather involved. But we
have defined $\varphi$ by its own graph, and avoided the imperant circularity
by using the recursion theorem to not rely on the graph.

We further define the relation $\ol R$ by listing all pairs $(x,y)\in R$ (We
can do this since the set $R$ is finite) and
then define:
\[v_0 \ol Rv_1: \bigvee_{(x,y)\in R)}(v_0=\ol x\wedge v_1=\ol y) \]
Thus we have that $\varphi$ have the following properties:
\begin{enumerate}
	\item $\varphi(0)=0$
	\item If $x+1$ proves that $L\not =\ol z$ and we have that $\varphi(x)Rz$,
		then $\varphi(x+1)=z$
	\item Else we have that $\varphi(x+1)=\varphi(x)$
\end{enumerate}
Further it is a total function
since it is defined by recursion, and this can be proven in \PRA.
\begin{prop}
	Let $\psi$ be the $\Sigma_1$-formula that defines the graph of
	$\varphi$. Then:
	\[\PRA\vdash\forall v_0\exists !v_1\psi v_0v_1\]
\end{prop}
\begin{proof}
	The uniqueness of $v_1$ is given by the definition of $\varphi$. The
	existence of a value is given by induction on $v_0$ in the $\Sigma_1$ formula $\exists
	v_1\psi v_0v_1$ in \PRA, and thus this induction is possible in \PRA. 

	\textbf{Base step:} When $v_0=0$ we have that $\psi(0)=0$ by
	definition of $\varphi$; i.e $\exists v_1\psi v_0v_1$, where $v_1=0$.

	\textbf{Induction step:} Assume that $\exists v_1\psi v_0v_1$ holds for
	$v_0=n$, i.e we have a $v_1=m$ such that $\varphi(n)=m$. Look at
	$\varphi(n+1)$, then if $n+1$ proves that $L_\psi\not =\ol z$ and we
	have that $\varphi(n)Rz$, then we will have that $\varphi(n+1)=z$, i.e
	$v_1=z$ otherwise we will have that $\varphi(n+1)=\varphi(n)=m$, and
	the induction is done.
\end{proof}
Further we will expand the langauge with a new function constant (This can be
done, since $\psi$ is the graph of a total function) $\ol \varphi$ with the following
defining axiom:
$$\ol \varphi(v_0)=v_1\leftrightarrow\psi (v_0)=v_1$$

We will now prove and state a few lemmas about the function $\varphi$ and the
limit of this function $L\tau=L$. These will build up the proof of Solovay's
First Completeness Theorem. Thus we will use the function $\varphi$ and the
limit $L$ to deduce the theorem

\begin{lem}
	\label{lem:2}
	The following three statements holds:
	\begin{enumerate}
		\item $\PRA\vdash\forall v_0(\ol \varphi v_0\leq \ol n)$
		\item For all $x\in\omega$ we have that:\ $$\PRA\vdash\forall
			v_o(\ol
			\varphi v_0=\ol x\rightarrow\forall v_1>v_0(\ol \varphi v_1=\ol x\vee
			\ol x\ol R\ol \varphi v_1))$$
		\item $\PRA\vdash\exists v_0v_1\forall v_2>v_0(\ol \varphi v_2=v_1)$. 
	\end{enumerate}
	Where (3) just means that $\PRA\vdash\exists v_1(L=v_1)$.
\end{lem}
\begin{proof}

	We will prove each part separately 

	\begin{enumerate}

		\item We will prove this part by induction.

			\textbf{Base case:} $\varphi(0)=0\leq n$ is is clearly true.

			\textbf{Induction step:} Assume that $\varphi(x)\leq n$ is
			true.Then we have that $\varphi(x+1)$ is in the range
			of $\varphi$
			and this $\leq n$ or we have that $\varphi(x+1)\leq n$ so the
			induction is complete.
		\item We will start of by write the formula we have to prove as
			the following equivalent formula:
			$$\forall v_1\forall v_0(\ol{\varphi}v_0=\ol x\rightarrow
			\varphi(v_0 + v_1+\ol 1)=\ol x\vee \ol x\ol R\ol
			\varphi(v_0+v_1+1))$$
			To prove this we will use induction on $v_1$. This is
			an induction on a $\Pi_1$ formula; which is a possible
			induction for \PRA\ by [Ref] 

			\textbf{Base case:} if $v_1=0$ then clearly we have that
			$\varphi(v_0+1)=\ol x$ or $\ol x\ol R\ol
			\varphi(v_0+1)$ by the definition of $\varphi$.

			\textbf{Induction step:} Assume that it holds for
			$v_1=n-1$, i.e that we for all $v_0$ have that:
			\[\varphi(v_0+\ol n)=\ol x\ \text{or}\ \ol x\ol
			R\ol\varphi(v_0+\ol n)\]
			We will split the rest of this proof up in two parts;
			one where the first disjunct is true, and one where the
			second is true.

			\textbf{The first disjunct is true:}. Assume that both
			$v_0+n+1$ codes a proof that $L\not =z$ and
			$\varphi(v_0+\ol n) Rz$ is true. This means that
			$\varphi(v_0+\ol n+1)=z$. But this is just that $x\ol
			R\varphi(v_0+\ol n+1)$, and the lemma is true in this case.
			If these assumptions are not true, we will then have
			that $\varphi(v_0+\ol n+1)=\varphi(v_0+\ol n)$. But
			then we have that $\varphi(v_0+\ol n+1)=\ol x$, and the
			lemma holds.
			
			\textbf{The second disjunct is true:} We will again
			start of by assuming that $L\not =z$ and
			$\varphi(v_0+n)Rz$; i.e $\varphi(v_0+n+1)=z$. We then
			get that $\varphi(v_0+n)R\varphi(v_0+n+1)$ and by the
			transitivity of $R$ we get: $x\ol R\varphi(v_0+n+1)$.
			If we otherwise have that
			$\varphi(v_0+n)=\varphi(v_0+n+1)$ we clearly have that
			$\ol x\ol R\varphi(v_0+n+1)$.

		\item Here we will first prove the following:
			$$\forall v_0(\exists v_1(\ol
			Fv_1=v_0)\rightarrow\exists v_1(L=v_1))$$
			This is clearly true for $v_0> n$. For $v_0\leq n$ we
			will use induction on the converse of $R$. By 2) from
			this lemma it holds for maximal nodes $y$ in $W$. If
			$y$ is not a maximal note and $\exists v_1(\ol
			\varphi(v_1)=\ol x)$ then again by 2) we either get
			that $L=\ol x$ or that for some $y$ such that $xRy$:
			$\exists v_1(\ol \varphi(v_1)=\ol y)$, and the induction
			hypothesis gives us that $\exists v_1(L=v_1)$. But
			since the set $W$ is finite we will end up with
			getting:
			\[\PRA\vdash\exists v_1(\ol\varphi(v_1)=\ol
			0)\rightarrow\exists v_1(L=v_1)\]
			But we have that $\PRA\vdash \ol \varphi(\ol 0)=\ol 0$
			so clearly: $\PRA\vdash\exists v_1(L=v_1)$
\end{enumerate}
\end{proof}

The induction in the above proof was not done inside of \PRA, both was done as
metamathematical induction outside of \PRA. This done because the formula
$\exists v_1(L=v_1)$ is $\Sigma_2$ and thus we can not use induction on this
formula inside of \PRA.

\begin{cor}
	$\PRA\vdash L\leq \ol n$, i.e $\PRA\vdash \bigvee_{x\leq n} L=\ol x$
\end{cor}
\begin{proof}
	By lemma \ref{lem:2} 1) and 3) and the following implication:
	$$\PRA\vdash v \leq \ol n\rightarrow \bigvee_{x\leq n} v=\ol x$$
	The corollary follows.
\end{proof}
\begin{lem}
	\label{lem:4}
	For all $x,y\leq n$ we have that:
	\begin{enumerate}
		\item $L=\ol x\wedge \ol x \ol R\ol y\rightarrow
			\Con_{\PRA+L=\ol y}$
		\item $\PRA\vdash L=\ol x\wedge \ol x\not = \ol y\wedge \neg
			(\ol x \ol R\ol y)\rightarrow\neg \Con_{\PRA+L=\ol y}$
		\item $\PRA\vdash L=\ol x\wedge \ol x >\0\rightarrow
			\Pr(\Godelnum{L\not=\ol x})$
	\end{enumerate}

\end{lem}
\begin{proof}
		We will prove each statement separately:

		\textbf{1)} Let $xRy$ and assume for contradiction that $L
			=\ol x\wedge \text{Pr}(\Godelnum{L\not =\ol y})$. Since we
			have that $L=\ol x$ we can chose a $v_0$ such that
			$\forall v_2(v_2>v_0\rightarrow \ol Fv_2=\ol x)$ and we
			can also chose $v_1+\ol 1>v_0$ such that
			$\text{Prov}(v_1+\ol 1,\Godelnum{L\not =\ol y})$
			But we also have the following:
			$$\PRA\vdash\text{Prov}(v_1+\ol 1,\Godelnum{L\not
			=y})\wedge\ol Fv_1=\ol \x\wedge \ol x\ol R\ol
			y\rightarrow\ol F(v_1+\ol 1)=\ol y$$
			This contradicts with $\forall v_2>v_0(\ol Fv_0=\ol x)$
			which came from the assumption that $L=\ol x$ so we con
			conclude:
			$$\PRA\vdash L=\ol x\wedge\ol x\ol R\ol
			y\rightarrow\neg\text{Pr}(\Godelnum{L\not =\ol y})$$
			Where we have that $\neg\text{Pr}(L\not =\ol y)$ is
			equivalent to $\text{Con}_{\PRA+L=\ol y}$ and 1)
			follows.

			\textbf{2)}
		  By [ref???] We have the following deduction:
			\begin{align}
				\PRA\vdash L=\ol x&\rightarrow \exists v_0(\ol
				Fv_0=\ol x)\\
						  &\rightarrow\text{Pr}(\Godelnum{
						  \exists v_0(\ol
					  Fv_0=\ol x)})\label{eq:21}
			\end{align} 
			Further we have from lemma \ref{lem:2}.2 and [D2, lob?]  that:
			\begin{align}
				\PRA&\vdash\forall v_0(\ol Fv_0=\ol
				x\rightarrow(L=\ol x\vee \ol x\ol RL))\\
				\PRA&\vdash(\Godelnum{\forall v_0(\ol Fv_0=\ol
					x\rightarrow (L=\ol x\vee \ol x\ol
				RL))})\label{eq:22}
			\end{align}
			From \ref{eq:21} and \ref{eq:22} we have the following:
			\begin{equation}
				\label{eq:23}
				\PRA\vdash L=\ol x
				\rightarrow\text{Pr}(\Godelnum{L=\ol x\vee \ol
				x\ol RL})
			\end{equation}
			asd?

			Which gives us [Why?]
			$$\PRA\vdash \ol x\not =\ol y\wedge \neg (\ol x\ol R\ol
			y)\rightarrow \text{Pr}(\Godelnum{\ol x\not =y\wedge
			\neg (\ol x\ol R\ol y)})$$
			This with \ref{eq:23} gives us:
			
			$$
				\PRA\vdash L=\ol x\wedge \ol x\not =\ol y\wedge
			\neg\ol x\ol R\ol
			y\rightarrow\text{Pr}(\Godelnum{L=\ol x\vee \ol x\ol
			RL})\wedge \text{Pr}(\Godelnum{\ol x\not =\ol y\wedge
			\neg\ol x\ol  R\ol y})$$
			From which we can deduce:
			$$\PRA\vdash L\ol x\wedge \ol x\not =\ol y\wedge \neg
			\ol x\ol R\ol y\rightarrow \text{Pr}(\Godelnum{L\not
			=\ol y})$$
			Which gives us 2)

			\textbf{3)} From the least number princible we have
			that:
			$$\PRA\vdash L=\ol x\wedge\ol x>\ol 0\rightarrow
			\exists v(\ol F(v+\ol 1)=\ol x\wedge \ol Fv\not =\ol
			x)$$
			By the definition of $F$ we have for such a $v$ the
			following:
			$$\PRA\vdash \ol F(v+\ol 1)=\ol x\ol Fv\not =\ol
			x\rightarrow \text{Prov}(v+\ol 1,\Godelnum{L\not =\ol
			x})$$
			And thus we have the following:
			$$\PRA\vdash L=\ol x\wedge \ol x> \ol 0\rightarrow
			\text{Pr}(\Godelnum{L\not =\ol x})$$
\end{proof}
Lemma \ref{lem:2} and \ref{lem:4} gives us the moss of the  facts that we need
about $L$ and $\varphi$; at least the facts about them that we can prove in $\PRA$.
We will also need the following result, which cannot be proven in $\PRA$.
\begin{lem}
	\label{lem:5}
	The following two statements are true, but they cannot be proven in
	\PRA.
	\begin{enumerate}
		\item $L=\0$
		\item For $0\leq x\leq n$ we have that $\PRA+L=\ol x$ is
			consistent.
	\end{enumerate}
\end{lem}
\begin{proof}
	\textbf{1)} By lemma \ref{lem:2}.3 the limit $L$ exists. If $x>0$ we have by lemma
	\ref{lem:4} that:
	\begin{align*}
		L=\ol x &\Rightarrow \PRA\vdash L\not =\ol x\\
		     &\Rightarrow L\not = \ol x
	\end{align*}
	Since \PRA\ is sound. But this is a contradiction and we must conclude
	that $L=0$.

	\textbf{2)} Since we have that $L=\ol 0$ is true and we have that \PRA\
	is sound, we have  that $\PRA+L=\ol 0$ is consistent. For $x>0$ we will
	apply lemma \ref{lem:4}.1 and get:
	$$\PRA\vdash L=\ol 0\wedge \ol 0\ol R\ol x\rightarrow
	\text{Con}_{\PRA+L=\ol x}$$
	We have that the antecedent is true and hence that
	$\text{Con}_{PRA+L=\ol x}$ is true, which proves this part of the
	lemma.
\end{proof}

We have now shown all the important basic properties that $\varphi$ and $L$ holds. In
the next part of the proof we will simulate the Kripke model $\mathcal{K}=\la
W,R,\phi\ra$, where $1\not\in\phi(\alpha)$. For this end we will let $L=\ol x$,
for $x>0$, assume the rule nodes of
$W=\{1,\ldots n\}$. We will start of by defining the interpretation $^*$. So for any $p$ let:
$$p^*=\bigvee\{L=\ol x:1\leq x\leq n\ \text{and}\ x\in \phi(p)\}$$
If this disjunction is empty, we will set it to be $\ol 0=\ol 1$. Further we
are only interested the sentence $\alpha$; i.e the set:
$$S(\alpha)=\{\beta:\ \beta\ \text{is a subformula of }\ \alpha\}$$
We will not look at $p\not\in S(\alpha)$

The following lemma is the crucial result about this interpretation, and the
theorem will follow easily from this result:
\begin{lem}
	\label{lem:10}
	Let $1\leq x\leq n$. For any $\beta$ and $^*$ as defined just above we
	have that:
	\begin{enumerate}
		\item $x\in\phi(\beta)\Rightarrow\PRA\vdash L=\ol x\rightarrow
			\beta^*$
		\item $x\not\in\phi(\beta)\Rightarrow\PRA\vdash L=\ol
			x\rightarrow \neg \beta^*$
	\end{enumerate}
\end{lem}
\begin{proof}
We will use induction on complexity of $\psi$. If $\beta=p$, then since $L=\ol
x$ is a
disjunct of $p^*$ we get:

$$x\in\phi(\varphi)\Rightarrow\PRA\vdash L=\ol x\rightarrow p^*$$

Which proves 1, for $\psi$ atomic. For 2 observe that if $x\not\in\phi( p)$, then
$L=\ol x$ contradicts all the  disjuncts of $p^*$ and thus:

$$x\not\in\phi(p)\rightarrow\PRA\vdash L=\ol x\rightarrow\neg p^*$$

The cases where  $\psi$ is $\neg\gamma,\gamma\wedge\sigma,\gamma\vee\sigma$ and
$\gamma\rightarrow\sigma$ are trivial. So we will just look at the case where
$\psi=\Box\gamma$. Here we make the following deductions:
\begin{align*}
	x\in\phi(\Box\theta)&\Rightarrow \forall
	y(xRy\Rightarrow y\in\phi(\gamma))\\
			  &\Rightarrow \forall(xR\Rightarrow\PRA\vdash L=\ol
			  y\rightarrow\gamma^*)\\
			  &\Rightarrow\bigwedge_{xRy}(\PRA\vdash L=\ol
			  y\rightarrow\gamma^*)\\
			  &\Rightarrow\PRA\vdash\bigvee_{xRy} L=\ol
			  y\rightarrow \gamma^*\\
			  &\Rightarrow\PRA\vdash\text{Pr}(\Godelnum{\bigvee_{xRy} L=\ol
			  y})\rightarrow\text{Pr}(\Godelnum{\gamma^*})
\end{align*}
And by the last line we can get the following by using the axioms of \GL
\begin{equation}
	\label{eq:1}
	x\in\phi(\Box\gamma)\Rightarrow \PRA\vdash\text{Pr}(\Godelnum{\bigvee_{xRy} L=\ol
			  y})\rightarrow\text{Pr}(\Godelnum{\gamma^*})
\end{equation}
We can now invoke lemma \ref{lem:5} 2 and 3 and get:
\begin{equation}
	\label{eq:2}
	\PRA\vdash L=\ol
	x\rightarrow\bigwedge_{\neg(xRz)}\text{Pr}(\Godelnum{L\not =z})
\end{equation}
and therfore we have that:
\begin{equation}
	\label{eq:3}
	\PRA\vdash L=\ol x\rightarrow \text{Pr}(\Godelnum{\bigvee_{xRy}L=\ol y})
\end{equation}
Now by \ref{eq:1} and \ref{eq:3} we get that:
\begin{align*}
	\vdash_x\Box\theta&\Rightarrow\PRA\vdash L=\ol x\rightarrow
	\text{Pr(\Godelnum{\gamma^*})}\\
			  &\Rightarrow\PRA\vdash L=\ol x\rightarrow
			  (\Box\gamma)^*
\end{align*}
I.e we have proven part 1 of the lemma. Similarly we can do the following
deduction:
\begin{align*}
	x\not\in\phi(\Box\gamma)&\Rightarrow\exists y(xRy\wedge
	y\not \in\phi(\gamma))\\
			      &\Rightarrow \exists y(xRy\wedge\PRA\vdash L=\ol
			      y\rightarrow\neg\gamma^*)\\
			      &\Rightarrow\exists y(xRy\wedge\PRA\vdash
			      \gamma^*\rightarrow L \not =\ol y)\\
			      &\Rightarrow\exists
			      y(xRy\wedge\PRA\vdash\text{Pr}(\Godelnum{
			      \gamma^*})\rightarrow\text{Pr}(\Godelnum{L\not
		      =\ol y})
\end{align*}
But by lemma \ref{lem:5} we get that if $xRy$:
$$\PRA\vdash L=\ol x\rightarrow\neg \text{Pr}(\Godelnum{L\not =\ol y})$$
all in all this gives us:
$$\PRA\vdash L=\ol x\rightarrow\neg\text{Pr}(\Godelnum{\gamma^*})$$
Which is just just  the following we where trying to show:
$$\PRA\vdash L=\ol x\rightarrow\neg(\Box\gamma)^*$$
\end{proof}
We can now finally prove Solovay's first completeness theorem:
\begin{proof}[Proof of Solovay's first completeness theorem]
	By lemma \ref{lem:10} we have that
	$$1\not\in\phi(\alpha)\Rightarrow\PRA\vdash L=\ol 1\rightarrow\neg\alpha^*$$
	But by lemma \ref{lem:5} we have that $\PRA\ +L=1$ is consistent, from
	which it follows that $\PRA+\neg\varphi^*$ is consistent; so $\varphi^*$
	is not a theorem af $\PRA$ and thus we have:
	$\PRA\not\vdash\varphi^*$ and the theorem follows by contraposition.
\end{proof}

With this proof we can give a interpretation of the set $W$ and relation $R$ in
a model of \GL. The set $W$ consists of recursively axiomitized extensions of
\PRA. And we have that $w_1Rw_2$ if and only if $w_1\vdash\text{Con}(w_2)$,
i.e the theory $w_1$ has the consistency of theory $w_2$ as one of its
theorems. It is clear that this relation is transitive and conversely
well-founded.

Lastly we can give an uniform version of Solovay's first completeness theorem:
\begin{thm}
	There is an arithmetical interpretations $^*$ such that for each modal
	formula $\alpha$ we have that:
	\[\PRA\vdash\alpha^*\ \text{iff}\ \vdash_\GL\alpha\]
\end{thm}
\section{The Second Theorem}
\label{chap:second}
Solovay's Second Completeness Theorem is a strengthening of the first one,
since it tells about trueness of a formuae $\alpha^*$ in the standard model
$\mathcal{N}=\la
\omega,+,\cdot\ra$ of aritmetics. The proof will follow that of the first
theorem, but we will look at the modal logic \GLS\ instead of the modal logic
\GL.
\begin{thm}
	For all modal sentences $\varphi$, the following is equivalent:
	\begin{enumerate}
		\item $\textbf{GLS}\vdash\alpha$
		\item $\GL\vdash\displaystyle\bigwedge_{\Box \beta\in
			S(\alpha)}(\Box\beta\rightarrow\beta)\rightarrow\alpha$
		\item $\alpha$ is true in all $\alpha$-reflexive FT Kripke models
		\item $\forall^*(\alpha^*\ \text{is true})$
	\end{enumerate}
\end{thm}
Some parts of this theorem has already been proven. We have
$(1)\Leftrightarrow(2)\Leftrightarrow(3)$ by theorem \ref{thm:MainGLS}. The
implication $(1)\Rightarrow(4)$ is clear, so we just have to prove
$(4)\Rightarrow(3)$

For proving $(4)\Rightarrow(3)$ we will again make use of contraposition. Let
$\mathcal{K}=\la(1,\ldots,n),R,\phi)\ra$ be given and let $1$ be the root.
Further let $\alpha$ be such that $1\not\in\phi(\alpha))$.
We
will assume that $\mathcal{K}$ is $\alpha$-reflexive. This means that we have:
$1\in\phi(\Box\beta\rightarrow\beta)$ for all $\beta\in S(\alpha)$

We will set $0Rw$ for all $w\in W$. We will now create a new model
$\mathcal{K'}$ where we have added $0$, so in this proof it is part of the
Kripke model we will look at. We will create this new model in the
following way:
\begin{align*}
	&W'=\{0,1\ldots,n\}\\
	&R'\ \text{extends}\ R\ \text{by assuming that}\ 0R'x\ \text{for all}\
	x\in W\\
	&\alpha_0=0\\
	&\phi'\ \text{extends}\ \phi\ \text{by putting}\ 0\in\phi'(p)\
	\text{iff}\ 1\in\phi(p)\ \text{for all}\ p\in S(\varphi)
\end{align*}
We will abuse notation and let $R$ denote $R'$ and $\phi$ denote $\phi'$.
\begin{lem}
	For all $\beta\in S(\alpha)$ we have that:
	$$0\in\phi(\beta)\ \text{iff}\ 1\in\phi(\beta) $$
\end{lem}
\begin{proof}
	Kig på intro af GLS.
\end{proof}

We will now define a function $varphi$ in the same way as before.
\begin{align*}
	\varphi(0)&=0\\
	\varphi(x+1)&=\begin{cases}
		y &\ \text{Prov}(\ol x+\ol 1,\Godelnum{L\not =\ol y})\wedge
			xRy\\
			\varphi(x) &\ \text{else}
		\end{cases}
\end{align*}
All the Lemmas about $\varphi$ and $L$ still holds, since the function $\varphi$ is
only determined by the frame $\la W,R\ra$ and not the Kripke model that we are
looking at. But the behavior of $\phi$ has
changed, since we must added the node  $0$ and we can only use sub formulas of
$\alpha$. So we define:
$$p^*=\bigvee\{L=\ol x: 0\leq x\leq n\wedge x\in\phi(p)\}$$
For $p\in S(\alpha)$ and let $p^*$ be random for all $p$'s that is not a
subformula of $\alpha$. We now prove and state the following lemma is analogies
to lemma \ref{lem:5}:
\begin{lem}
	Let $0\leq x\leq n$. For any $\beta\in S(\alpha)$ and $^*$ as defined
	above we have:
	\begin{enumerate}
		\item $x\in\phi(\beta)\Rightarrow\PRA\vdash L=\ol
			x\rightarrow\beta^*$
		\item $x\not\in\phi(\beta)\Rightarrow\PRA\vdash L=\ol
			x\rightarrow\neg\beta^*$
	\end{enumerate}
\end{lem}
\begin{proof}
	For $0<x$ the proof is identical to the proof of [ref?]. So we will
	only prove the case where $x=0$. The is again an induction on the
	complexity of $\beta$. We will only prove the cases where
	$\beta=\Box\gamma$.

	Let $\beta=\Box\gamma$. We then have:
	\begin{align*}
		0\in\phi(\Box\gamma)&\Rightarrow\forall x(1\leq x\leq
		n\Rightarrow x\in\phi(\gamma))\\
				  &\Rightarrow \forall x(1\leq x\leq
				  n\Rightarrow \PRA\vdash L=\ol
				  x\rightarrow\gamma^*)
	\end{align*}
	Since $x>1$ and this case of the lemma has been proven, when we proved
	the lemma for the first completeness theorem. We can also
	make the following deduction by the induction hypothesis:
	\begin{align*}
		0\in\phi(\Box\gamma)&\Rightarrow 1\in(\gamma)\\
				    &\Rightarrow 0\in\gamma)\\
				  &\Rightarrow\PRA\vdash L=\ol
				  0\rightarrow\gamma^*
	\end{align*}
	By combing these two we get:
	\begin{align*}
		0\in\phi(\Box\gamma)&\Rightarrow\bigwedge_{x\leq n}(\PRA\vdash
		L=\ol x\rightarrow\gamma^*)\\
				  &\Rightarrow\PRA\vdash(\bigvee_{x\leq n}L=\ol
				  x)\rightarrow\gamma^*\\
				  &=\PRA\vdash\text{Pr}(\Godelnum{\bigvee L=\ol
				  x})\rightarrow\text{Pr}(\Godelnum{\gamma^*})
	\end{align*}
	But by corollary [<ref?] we have that $\PRA\vdash\bigvee L=\ol x$ and
	thus $\PRA\vdash\text{Pr}(\Godelnum{\bigvee L=\ol x})$ so all in all we
	have:
	\begin{align*}
		0\in(\Box\gamma)&\Rightarrow \PRA\vdash
		\text{Pr}(\Godelnum{\gamma^*})\\
				  &\Rightarrow\PRA\vdash L=\ol
				  0\rightarrow\text{Pr}(\Godelnum{\gamma^*})
	\end{align*}
	This proves (1). The proof of (2) is a bit easier. We have:
	\begin{align*}
		0\not\in\phi(\Box\gamma)&\Rightarrow\exists x(1\leq x\leq n\wedge
		x\not\in\phi(\gamma))\\
				      &\Rightarrow\exists x(1\leq x\leq
				      n\wedge\PRA\vdash L=\ol x\rightarrow
				      \neg\gamma^*)\\
				      &\Rightarrow\exists x(1\leq x\leq n\wedge
				      \PRA\vdash\gamma^*\rightarrow L\not =\ol
				      x)\\
				      &\Rightarrow\PRA\vdash L=\ol 0\rightarrow
				      \neg \text{Pr}(\Godelnum{\gamma^*})
	\end{align*}
	And thus by Lemma [Ref?] we have that $\PRA\vdash L=\ol
	0\rightarrow\neg\text{Pr}(\Godelnum{L\not =\ol x})$ for $x>0$
\end{proof}
We thus have that $L=\ol 0$ is true, and we  can now finally prove the second completeness theorem:
\begin{proof}[Proof of the Second Completeness theorem]
	Assume that $\alpha$ is false in $\mathcal{K}$ i.e
	$1\not\in\phi(\alpha)$. Then by lemma [asd] we get
	$0\not\in\phi(\alpha)$
	and the just proven lemma gives:
	$$\PRA\vdash L=\ol 0\rightarrow\neg\alpha^*$$
	Since we have that $L=\ol 0$ is true we then get $\neg\alpha^*$ which
	just means that $\alpha^*$ is false.
\end{proof}
\subsection{Using the Second Completeness Theorem}
We can use the second completeness theorem to prove the following theorem named
after Rosser:
\begin{thm}
	There is a arithmetical sentences $F$ such that:
	\begin{enumerate}
		\item $\PRA\not\vdash F$
		\item $\PRA\not\vdash\neg F$
		\item $\PRA\vdash \Con\rightarrow\neg \Pr(\Godelnum{F})$
		\item $\PRA\vdash \Con\rightarrow\neg
			\text{Pr}(\Godelnum{\neg F
			})$
	\end{enumerate}
\end{thm}

\begin{proof}
	We define the following pointed Kripke model:
	\[\mathcal{K}=\la \{1,2,3\},\{(1,2),(1,3)\},\{(1,p),(2,p)\},1\ra\]
	it can be visualized by the following graph:
\begin{figure}[h]
	\begin{center}
\begin{tikzpicture}[baseline={(0,0)}]
	\node[shape=circle,draw=black] (1) at (0,0) {$1$};
	\node[shape=circle,draw=black] (2) at (2,2) {$2$};
	\node[shape=circle,draw=black] (3) at (-2,2) {$3$};
	\path [-] (1) edge node[left] {} (2);
	\path [-] (1) edge node[left] {} (3);
\end{tikzpicture}
\end{center}
\end{figure}
Where $p$ is true in world $2$. We will further let $\alpha$ be the
following formula:
\[\neg\Box p\wedge\neg\Box\neg p\wedge\Box(\neg\Box\bot\rightarrow\neg\Box
p\wedge\neg\Box\neg p)\]
We have that $\alpha$ is true in $\mathcal{K}$. Since we have that:
\begin{enumerate}
	\item $1\in\phi(\neg\Box p)$ since we have that $1R3$ and that
		$3\not\in\phi(p)$
	\item $1\in\phi(\neg\Box\neg p)$ since we have that $1R2$ and that
		$2\in\phi(p)$
	\item $1\in\phi(\Box(\neg\Box\bot\rightarrow\neg\Box
		p\wedge\neg\Box\neg p)$ since we have that
		$2,3\in\phi(\Box\bot)$
\end{enumerate}
It can be shown that this model is $\alpha$-reflexive; the proof of this can be
found in \parencite{Smor1985}
\end{proof}
This result can be strengthened and it can be shown that the found formula is
actually $\Sigma_1$
\section{Generalisations  of the Completeness Theorems}

In this section we have shown the Solovay's Completeness Theorems for \PRA.
But these theorems holds for a much wider class of fragments of Peano
Arithmetics. It can be shown that the theorem holds for
theories $T$ that fulfills the following to conditions:
\begin{enumerate}
	\item $T$ extends $I\Delta_0+\text{EXP}$
	\item Let $\alpha(x)$ be $ \Delta_0$ formula, then $T$ does not prove any
		false sentences of the form $\exists x\alpha(x)$
\end{enumerate}

This proof does not use the Recursion theorem for creating the function
$\varphi$, but instead uses the diagonalization lemma. The
proof of this will not this will not be given here, but it can be found in
\parencite{Dick1991}. This generalization shows that \GL\ axiomatize a very big
range of different fragments of arithetmics; i.e the proof predicate in all of
these fragments behaves in the same way.

It is not known if Solovays theorems holds for weaker conditions than
$I\Delta_0+\text{EXP}$; i.e we do not know if it holds for example for the theory
$I\Delta_0+\Omega_1$.



Solovay's theorems can also be proven for some fragments without the use of
fixed points. The two different proof strategies seen so fair each uses fixed
points. Either by using the recursion theorem or by using the digitalization
theorem for a given arithmetic theory. Fedor Pakhomov has proven the theorems
without the use of the fixed point lemma or the recursion theorem in
\parencite{Fedo2017}. This
newer proof might be able to giver a specific lower bound on the strength of
the fragments Solovays theorems holds for.



\section{Solovays Completeness Theorems and Fixed Points}

From table \ref{Fig} and the arithmetical soundness theorem, we can conclude the following things:
\begin{enumerate}
	\item From line 3, we have that
		$\vdash_\GL\sBox(p\leftrightarrow\Box\neg
		p)\leftrightarrow\sBox(p\leftrightarrow\Box\bot)$; i.e a
		sentence $F$ of \PRA\ is equivalent to its own unprovability if and only
		if $F$ is equivalent to the assertion that \PRA is
		inconsistent.
		For let $^*$ be such that $F=p^*$, then we if $F$ is such that
		$\PRA\vdash F\leftrightarrow\text{Pr}(\Godelnum{\neg F})$ which
		is the same as $\PRA\vdash(p\leftrightarrow\Box\neg p)^*$ and
		thus we have: $\PRA\vdash\Box(p\leftrightarrow\Box\neg p)^*$
		and hence $\PRA\vdash\sBox(p\leftrightarrow\Box\neg p)^*$. But
		by the soundness theorem we get that:
		$\PRA\vdash(\sBox(p\leftrightarrow\Box\neg
		p)\leftrightarrow\sBox(p\leftrightarrow\Box\bot))^*$, and thus
		$\PRA\vdash\sBox(p\leftrightarrow\Box\bot)^*$ whence
		$\PRA\vdash(p\leftrightarrow\Box\bot)^*$ and thus $\PRA\vdash
		F\leftrightarrow\text{Pr}(\Godelnum{\bot})$. Which shows that
		$F$ is equivalent to the inconsistency of \PRA.
	\item We can from line 1 and 6 infer that a sentence of \PRA\ is
		equivalent to its own unprovability if and only if it is
		equivalent to the assertion that \PRA\ is consistent and that a
		sentence of \PRA\ that is equivalent to the assertion that it
		is unprovable if it is provable if and only if it is equivalent
		to the assertion that if the inconsistency of \PRA\ is provable
		then \PRA\ is provable.
	\item From line 10 we can conclude that for arbitrary sentence $F$ and
		$G$ of \PRA\ that: $F$ is equivalent to the conduction of $F$ is
		provable and $G$ is true if and only if $F$ is equivalent to
		the conduction of $G$ is provable and true. Here we take $^*$
		such that $p^*=F$ and $q^*=G$.
	\item From line 6 again we can conclude that a fixed point may have a
		bigger modal degree, than formula of which it is a fixed
		point: the formula $\Box p\rightarrow\Box\neg p$ has modal degree $1$
		and the formula $\Box\Box\bot\rightarrow\Box\bot$ has degree
		$2$. We can conclude the same for line 7, where the fixed point
		has modal degree $3$ and the formula it is a fixed point for
		has degree $2$.
	\item The modal formula $\alpha$ has at most the same modal degree as
		the modal formula $\sigma$.
	\item We can further see that line 1 gives that sentences of arithmetic
		that express their own unprovability are equivalent to the
		assertion arithmetic is consistent. Line 2 is the same as Löb's
		answer to Henkin's question, and line 4 gives that refutable
		sentences are those equivalent to their own consistency.
\end{enumerate}

\section{Concluding remarks: Implications of Solovay's Theorems}

The clearest implication of the generalized versions of Solovay's Completeness
Theorems is that the proof predicate of arithmetics can be axiomatized;
everything there is to say about the predicate, can be deduced from the
derivability conditions and Löb's theorem. This can be seen both as a positive
and negative result. Positive since the proof predicate follows some simple
rules and these rules  does not alter even though you add more induction to the
fragment of arithmetics you are working with. The last point can also be seen
as a negative one; there is no way to differentiate the different fragments of
arithmetics by looking at their proof predicate, since the proof predicates are
all axiomatized by the same modal logic.

Further the proportional provability logic is not effected by Quines critique of
modal logic as being unintelligible, since it has a very clear and unambiguous
arithmetic interpretation. This sets it apart from the more philosophical
interpretations of modal logic, since these interpretations do not have a clear
meaning.

Lastly the theorems also show that modal logic can be interesting for
mathematicians that are not interested in the philosophical part of logic. The
theorems shows that modal logic can be used to get further knowledge about
mathematical systems and it can probably be used in other areas of mathematics.
Modal logic also have some uses in both topology and set theory
\parencite{Artemov2007}.
\end{document}
