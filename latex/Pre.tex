\documentclass[../main.tex]{subfiles}

\begin{document}

In this section I will state a few results and definitions from my two project
I have written; one about modal logic and one about Gödel's incompleteness
theorems.
These results will be given without proof. The proofs can be found in
\textbf{Indsæt ref}.

The main points of this section will be the definition of the predicate
$\text{Pr}(\cdot)$ which is constructed in the proof of  Gödel's Incompleteness
Theorems, and the definition of the modal logic \GL\
(The name comes from Gödel and Löb) and a few of the properties of this modal
logic. The
main result of this project will be Solovay's arithmetic completeness theorem,
that simply put states that these two predicates (from different mathematical
systems) "behaves" in the
same way.

\section{Modal Logic}
Modal logic will play an important rule in this project. The main point here,
is that provability can be seen as a modality. This will be swhon later on in
the project.

\subsection{The language of modal logic}
In this subsection we will define our language $\mathcal{L}_\Box$ of modal logic. 
We will have start of with a  set called $\Phi$, which consists of propositional letter
$p,q,\ldots$. We can further define the primitive symbols of our modal language
$\mathcal{L}_\Box$:

\begin{defi}
	The following symbols are the primitive in our language
	$\mathcal{L}_\Box$
	\begin{enumerate}
		\item Every letter from the set $\Phi$.
		\item The logical constants $\bot$ (zero-ary) and $\rightarrow$
			(binary)
		\item The modal operator $\Box$ (unary).
	\end{enumerate}
\end{defi}
We can now define the well formed formulas of $\mathcal{L}_\Box$, or as we will
call them for the: \textit{modal formulas}.

\begin{defi}
	We define a \textit{modal formula} recursively in the following way:
	\begin{enumerate}[label=\roman*]
		\item Each $p\in\Phi$ is a modal formula
		\item $\bot$ is a modal formula
		\item If $\alpha$ and $\beta$ are modal formulas then so is
			$\alpha\rightarrow\beta$.
		\item If $\alpha$ is a modal formula then so is $\Box\alpha$
		\item Nothing is a modal formula except as prescribed by
			(i)-(iv)
	\end{enumerate}
\end{defi}
Further we will define the connectives $\wedge,\neg,\vee,\leftrightarrow,
\Diamond, \Box^n,\Diamond^n$ and $\top$
in the usual way. 
We will read $\Box\alpha$ as "$\alpha$ is provable. Normally in the alethic modal logic the
$\Box$ reads "$\alpha$ is necessarily true. Further the symbol $\bot$ stands
for falsehood. We will read $\Diamond \alpha$ as "$\alpha$" is consistent.
\subsection{Semantics}
In this section we will define the notation of truth for our modal logic. We
will begin by defining a Hintikka frame and a Kripke model
\begin{defi}[Hintikka frame]
	A \textit{Hintikka frame} is a tuple $\mathcal{H}=\la W,R\ra$ where is $W$ is
	non-empty set (we will call its members for nodes\footnote{Normally
		the members of this set will be called worlds, but this
	interpretation do not make senses in provability logic}) and where $R\subset
	W\times W$ is a relation that we will call the accessibility relation.
	We will use the notation $wRv$ to denote that the note $w$ sees the note
	$v$; i.e $(w,v)\in R$.
\end{defi}
We do not at the given time give the relation $R$ any properties. It could be
reflexive, transitive or anti-symmetric.
\begin{defi}[Kripke model]
	A \textit{Kripke model} is a tuple $\mathcal{K}=\la \mathcal{H},\phi\ra$ where
	$\mathcal{H}$ is a Hintikka frame and $\phi$ is a valuation function that to
	each proportional letter $p\in \Phi$ assigns a subset $\phi(p)$ of $W$.
	Formally:
	\[\phi:\Phi\rightarrow\mathcal{P}(W)\]
\end{defi}
We can now define the notion of truth in a given Kripke model $\mathcal{K}=\la
\mathcal{H},\phi\ra$. If a modal formula $\alpha$ is true at a node $w$ in a
Kripke model $\mathcal{K}=\la \mathcal{H},\phi\ra$ we will write:
\[\vDash_w^\mathcal{K}\alpha\]
This notation is taken from \cite{Lemmon1977}. Another notation instead of
$\vDash^\mathcal{K}_w\alpha$ is $\mathcal{K},w\vDash\alpha$.
\begin{defi}[Truth definition]
	We define the notion of \textit{truth} in a Kripke modal $\mathcal{K}=\la
	\mathcal{H},\phi\ra$ in the
	following way:
	\begin{enumerate}
		\item If $\alpha$ is propositional $p$ then:
			\[\vDash_w^\mathcal{K}\alpha\ \text{iff}\ w\in\phi(p)\]
		\item If $\alpha$ is $\bot$ then:
			\[\vDash_w^\mathcal{K}\alpha\ \text{iff never} \]
		\item If $\alpha$ is $\beta\rightarrow\gamma$ then:
			\[\vDash_w^\mathcal{K}\alpha\ \text{iff}\ \text{if}
				\vDash_w^\mathcal{K}\beta\ \text{then}\
			\vDash_w^\mathcal{K}\gamma\]
		\item if $\alpha$ is $\Box\beta$ then:
			\[\vDash_w^\mathcal{K}\alpha\ \text{iff for all}\ v\
				\text{such that}\ wRv\ \text{we have}\
			\vDash_v^\mathcal{K}\beta\]
	\end{enumerate}
\end{defi}
We will end this section with definitions of a \textit{valid} and
\textit{satisfiable} formula.
\begin{defi}
For a given Hintikka frame $\mathcal{H}=\la W,R\ra$ we say that $\alpha$ is
\textit{valid} in $\mathcal{H}$ and write $\vDash^\mathcal{H}\alpha$ if and only if
$\vDash^\mathcal{K}_w\alpha$ for all Kripke models on our frame $\mathcal{H}$ and all
notes $w\in W$. Further we say that $\alpha$ is \textit{satisfiable} in $\mathcal{H}$
if and only if $\vDash^\mathcal{K}_w\alpha$ for some Kripke model $\mathcal{K}$ on
the frame $\mathcal{H}$ and some note $w\in W$.

$\alpha$ is called \textit{valid} if and only if $\alpha$ is valid in all frames
$\mathcal{H}$, if this is the case we simply write $\vDash\alpha$. $\alpha$ is
called \textit{satisfiable} if and only if $\alpha$ is satisfiable on all frames
$\mathcal{H}$.

$A$ formula $\alpha$ is called valid on a class of frames $\mathcal{C}$ if for
all $\mathcal{H}\in\mathcal{C}$ we have that $\vDash^\mathcal{H}\alpha$. We
write this as $\vdash^\mathcal{C}\alpha$.

A formula $\alpha$ is \textit{valid} in a Kripke model $\mathcal{K}=\la
W,R,\phi\ra$ if and only if $\vDash^\mathcal{K}_w\alpha$ for all notes $w\in
W$. We write this as $\vDash^\mathcal{K}\alpha$

If a Kripke model $\mathcal{K}$ has a \textit{minimal node} $w_0$, i.e a node
such that for all $v\in W$ we have $w_0Rv$, then a modal formula $\alpha$ is
called  \textit{true} in $\mathcal{K}$ if we have that
$\vDash_{w_0}^\mathcal{K}\alpha$.
\end{defi}
\subsection{The modal logic GL}
In this section we will define the notion of a modal logic and define the modal
logic \GL, which will be our base modal logic in this project.
\begin{defi}
	A \textit{modal logic} $\Lambda$ is a set of modal formulas that contains all
	propositional tautologies, and is closed under \textit{modus ponens}
	(MP) and uniform substitution. If $\alpha\in\Lambda$ we say that
	$\alpha$ is a theorem of $\Lambda$ and write $\vdash_\Lambda\alpha$,
	else we write $\not\vdash_\Lambda\alpha$.

	We will also define the set
	$\Lambda_S=\{\alpha\vDash^\mathcal{S}\alpha,\ \text{for al structures}\
	\mathcal{S}\in S\}$. Where $S$ is any class of frames.
\end{defi}
We will further define when a given modal formula is deducible in a modal
logic:
\begin{defi}
	Let $\Lambda$ be a modal logic, let $\beta_1,\ldots,\beta_n$
	be modal formulas in $\Lambda$ and let $\alpha$ be a modal formula. We say that
	$\alpha$ is \textit{deducible} from $\beta_1,\ldots,\beta_n$ if
	$(\beta_1\wedge\cdots\wedge\beta_n)\rightarrow\alpha$ is a tautology.

	If $\Gamma\cup\{\alpha\}$ is a set of modal formulas, then $\alpha$ is
	\textit{deducible} in $\Lambda$ from $\Gamma$ if $\vdash_\Lambda\alpha$ or if
	there are formulas $\beta_1,\ldots,\beta_n\in\Gamma$ such that:
	\[\vdash_\Lambda(\beta_1\wedge\cdots\wedge\beta_n)\rightarrow\alpha\]
	In this case we write $\Gamma\vdash_\Lambda\alpha$ else we write
	$\Gamma\not\vdash_\Lambda\alpha$	
\end{defi}

We will now define what a \textit{normal modal logic} is. We will later look at
a modal logic that is not normal. But the modal logic \GL\ is a normal one.
\begin{defi}
	A modal logic $\Lambda$ is called \textit{normal} if it has the following
	axioms and deduction rules:
	\begin{enumerate}
		\item[Tau] All propositional tautologies. 
		\item[K] $\Box(p\rightarrow q)\rightarrow(\Box p\rightarrow\Box
			q)$
		\item[MP] $p,\ p\rightarrow q \vdash_\Lambda q$
			(\textit{modus ponens} )
		\item[Nec] $p \vdash_\Lambda\Box p$ (\textit{necessitation})
	\end{enumerate}
	If $\Gamma$ is a set of modal formulas we call the smallest normal
	logic contanting $\Gamma$ the normal modal logic axiomatized by
	$\Gamma$. The normal modal logic axiomatized by empty set is called
	\textbf{K} and
	this is the smallest normal modal logic.\footnote{Here \textbf{K}
	stands for Kripke}
\end{defi}
A few standard results follows from these definitions:
\begin{prop}
	If $\Lambda$ is a normal modal logic then:
	\begin{enumerate}
		\item If $\vdash_\Lambda\alpha\rightarrow\beta$ then
		$\vdash_\Lambda\Box\alpha\rightarrow\Box\beta$
	\item
		$\vdash_\Lambda\Box(\alpha\wedge\beta)\leftrightarrow\Box\alpha\wedge\Box\beta$
	\item
		$\vdash_\Lambda\Box(\alpha_1\wedge\cdots\wedge\alpha_n)\leftrightarrow
		\Box\alpha_1\wedge\cdots\wedge\Box\alpha_n$ for $n\geq 2$.
	\item If
		$\vdash_\Lambda\beta_1\wedge\cdots\wedge\beta_n\rightarrow\alpha$
		then
		$\vdash_\Lambda\Box\beta_1\wedge\cdots\wedge\beta_n\rightarrow\Box\alpha$,
		for $n\geq 0$.
	\end{enumerate}
\end{prop}
It should further be noted that the axiom \textbf{K} is equivalent to the
following formula:
\[\Box\alpha\wedge\Box(\alpha\rightarrow\beta)\rightarrow\beta\]
Smorynski uses this formula as an axiom instead of the axiom \textbf{K}.
We can extend the logic \textbf{K} in the following way, to get the logic
\textbf{GL} and the logic \textbf{4}.
\begin{defi}
	\textbf{K4} is the modal logic extending \textbf{K} by adding the
	following axiom:
	\begin{enumerate}
		\item[4] $\Box\alpha\rightarrow\Box\Box\alpha$
	\end{enumerate}
	\GL\ is the modal logic extending \textbf{K} by adding the two following
	axioms:
	\begin{enumerate}
		\item[4] $\Box p\rightarrow\Box\Box p$
		\item[L]
			$\Box(\Box p\rightarrow p)\rightarrow\Box p$
	\end{enumerate}
\end{defi}
Smorynski calls the modal logic \textbf{K4} for \textbf{BML} (Basic modal
logic).


It can be shown that the axiom 4 is redundant. k
It is now possible to define the notions of soundness, strong completeness and
weak completeness.
\begin{defi}
	Let $S$ be a class of frames or models.
	\begin{enumerate}
		\item A normal modal logic $\Lambda$ is sound with respect to $S$ if
		$\Lambda\subseteq\Lambda_S$, i.e if $\vdash_\Lambda\alpha$
		implies $\vDash^\mathcal{S}\alpha$ for all $\mathcal{S}\in S$.
	\item A modal logic $\Lambda$ is strongly complete with respect to $S$
		if for any set of formulas $\Gamma\cup\{\alpha\}$, if
		$\Gamma\vDash^\mathcal{S}\alpha$ then
		$\Gamma\vdash_\Lambda\alpha$ for all $\mathcal{S}\in S$ and it
		is weakly complete with respect to $S$ if for any formula
		$\alpha$ if $\vDash^\mathcal{S}\alpha$ then
		$\vdash_\Lambda\alpha$ for all $\mathcal{S}\in S$.
	\end{enumerate}
\end{defi}
To show completeness results you will create a \textit{canonical} model and
then show that the given modal logic is complete with respect to this model. It
can be shown that \textbf{K} is sound and strongly complete with respect to the
class of all frames. A lot of different completeness results can be found in
\cite{Lemmon1977} and \cite{Blackburn2002}. Here we will just state a few of
the results about \GL.

\begin{thm}
	\GL\ is not sound and strongly complete with respect to any class of
	frames.
\end{thm}

There is another result concerning the completeness and soundness of \GL. To
state this we first need some definitions concerning relations:
\begin{defi}
	A relation $R$ on frame $\mathcal{H}=\la W,R\ra$ is said to be
	\textit{transitive} if for all $w_1,w_2,w_3\in W$, whenever $w_1Rw_2$
	and $w_2Rw_3$ then $w_1Rw_3$.
	
	The relation $R$ is said to be \textit{well-founded} on
	$\mathcal{H}$ if every non-empty  subset $V\subseteq W$ has a minimal
	element with respect to $R$. In other words $R$ is well-founded if
	there is no infinite sequence $\ldots Rw_2Rw_1Rw_0$

	Further the relation $R$ is said to be \textit{conversely well-founded}
	on $\mathcal{H}$ if the converse $R^{-1}$ of $R$ is well-founded; i.e
	if there is no infinite sequence such that $w_0Rw_1Rw_2R\ldots$
\end{defi}
We can now state the following theor::
\begin{thm}
	\label{thm:GLcomplete}
	\GL\ is sound and weakly complete with respect to the class of
	transitive and conversely well-founded frames.
\end{thm}
In chapter \ref{chap:GL} we will show some further results about \GL\ and in
chapter
\ref{chap:Fixed} we will show a fixed point theorem about \GL.

\begin{prop}\label{prop:GL}
	$\vdash_\GL\Box(\alpha\leftrightarrow\beta)\rightarrow(\Box\alpha\leftrightarrow\Box\beta)$
\end{prop}
\textbf{Pris evt. nogle få deduktions former der vil blive brugt senere}



\section{Gödels Incompleteness Theorems}
In this section we will state and sketch the proof of Gödels incompleteness
theorems. In this sketch we will define a predicate $\text{Pr}(\cdot)$ that
says a given formula has a proof in arithemteics.


\subsection{Primitive Recursive Functions}
We will start of by defining a class of function that will both play a crucial
role in the proof of Gödels Incompleteness Theorems, and in the rest of this
project. 
\begin{defi}
	The class of primitive recursive functions is the smallest class closed
	under the following schemata:
	\begin{enumerate}[label=\Roman*.]
		\item $S(x)=x+1$ is primitive recursive.
		\item $Z(x)=0$ is primitive recursive.
		\item $P^n_i(\x)=x_i$ is primitive recursive.
		\item If $g,h_1,\ldots h_m$ are primitive recursive then so
			is
			$$f(\x)=g(h_1(\x),\ldots,h_m(\x)$$
		\item If $g$ and $h$ are primitive recursive and $n\geq 1$ then
			$f$ is also primitive recursive where:
			\begin{align*}
				f(0,\x)&=g(\x)\\
				f(x_1+1,x_2,\ldots,x_n)&=h(x_1,f(\x),x_2,\ldots,x_n)
			\end{align*}
	\end{enumerate}
\end{defi}
We can also define relations as being primitive recursive:
\begin{defi}
	A relation $R\subseteq \omega^n$ is primitive recursive if its
	chararestic function :
	\[\chi_R(\vec{x})=\begin{cases}
		0 &\text{if}\ R(\vec{x})\\
		1 &\text{if}\ R(\vec x)
	\end{cases}\]
\end{defi}
So with these definitions we can show that some well known functions are
primitive recursive.

\begin{prop}
	The following list of functions are all primitive recursive:
\begin{table}[!ht]
\begin{tabular}{p{1cm}p{6cm}p{5cm}}
	1.& $K^n_k(x_1,\ldots,x_n)=k$  & Constant  \\
	2.& $A(x,y)=x+y$ &Addition  \\
	3.&$M(x,y)=x\cdots y$  &Multiplication  \\
	4.&$E(x,y)=x^y$  &Exponentiation  \\
	5.&$pd(x)=\begin{cases}
		x-1, &x>0\\
		0,& x=0
	\end{cases}$&  Predecessor\\
		6.&$x\dot - y=\begin{cases}
			x-y&x,\geq y\\
			0,& x<y
		\end{cases}$  &Cut-off subtraction  \\
			7.&$sg(x)=\begin{cases}
				0,&x=0\\
				1,&x>0
			\end{cases}$& Signum \\
				8.& $\ol{sg}(x)=\begin{cases}
				1,&x=0\\
				0,&x>0
			\end{cases}$& Signum complement  \\
				9.& $|x-y|=\begin{cases}
				x-y,&x\geq y\\
				y-x,& x<y
			\end{cases}$& absolute value  \\
\end{tabular}
\end{table}
\end{prop}

\subsection{\PRA}
In this section we will specify the rules and language of \PRA. First of we
will define the language $\mathcal{L}_\PRA$ of \PRA.

\begin{defi}
	The language $\mathcal{L}_\PRA$ consists of the following symbols:
	\begin{align*}
		\text{Variables}&: v_0,v_1,\ldots \\
		\text{Constant} &: \0\\
		\text{Function symbols} &: \ol f\ \text{for each primitive
		recursive function}\ f\\
			\text{Relation symbols} &:\ =\\
			\text{propositional connectives:} &:\
			\neg,\wedge,\vee,\rightarrow\\
			\text{Quantifiers} &: \forall,\exists
	\end{align*}
\end{defi}
We will now define the notions of terms and formulae of \PRA. This part is
mostly done to settle notations.

\begin{defi}
	\begin{enumerate}
		\item The set of \textit{terms} of the langauge of \PRA\ is
			defined inductively by:
			\begin{enumerate}
				\item $\ol 0$ is a term and each $v_i$ is a
					term.
				\item If $f$ is an $n$-ary function symbol and
					$t_1,\ldots t_n$ are terms then $\ol
					f t_1\ldots t_n$ is a term.
			\end{enumerate}
		\item The set of \textit{formulae} of the language of  \PRA\ is
			defined inductively by:
			\begin{enumerate}
				\item If $t_1$ and $t_2$ are terms then 
					$=t_1t_2$ is a formula.
				\item If $F$ and $ G$ are formulae,
					so are
					$\neg F,\wedge F G,\vee F G$
					and $\rightarrow F G$.
				\item If $F$ is a formula and $v$ is a
					variable, then $\exists vF$ and
					$\forall vF$ are also formulae.
			\end{enumerate}
	\end{enumerate}
\end{defi}
We use Polish notation so we do not have parentheses in our language. In
practices we will use parentheses and infix notation. We will also need the
definition of a \textit{sentence}.

\begin{defi}
	A formula $F$ of \PRA\ is called a sentence if it has no free
	variables; i.e if all variables $v$ that occurs in $F$ are bound.
\end{defi}

Having defined the language of \PRA\ we can now state the different axioms of
\PRA:
\begin{defi}
	The axioms \textbf{PRA} are the following:
	\begin{enumerate}
		\item Propositional axioms
			\begin{enumerate}
				\item
					$F\rightarrow( G\rightarrow F)$
				\item
					$(F\rightarrow( G\rightarrow H)
					)\rightarrow((F\rightarrow G)\rightarrow
					(F\rightarrow H))$
				\item $F\wedge G\rightarrow F$
				\item $F\wedge G\rightarrow G$
				\item $F\rightarrow
					( G\rightarrow F\wedge G)$
				\item $F\rightarrow F\vee G$
				\item $ G\rightarrow F\vee G$
				\item
					$(F\rightarrow H)\rightarrow(( G\rightarrow
					 H)\rightarrow(F\vee G\rightarrow H))$
				\item
					$(F\rightarrow
					G)\rightarrow((F\rightarrow\neg G)
					\rightarrow\neg F)$
				\item $\neg\neg F\rightarrow F$
			\end{enumerate}
		\item Quantifier axioms
			\begin{enumerate}
				\item $\forall vF v\rightarrow F t$
				\item $F t\rightarrow \exists v F
					v$
			\end{enumerate}
			Where $t$ is substitutable for $v$ in $F v$ in
			both cases.
		\item Equality axioms
			\begin{enumerate}
				\item $v_0=v_0$
				\item $v_0=v_1\rightarrow v_1=v_0$
				\item $v_0=v_1\wedge v_1=v_2\rightarrow v_0=v_2$
				\item $v_i=w\rightarrow \bar{f}(v_1,\ldots,
					v_i,\ldots,v_n)=\bar{f}(v_1,\ldots,
					w,\ldots,v_n)$
			\end{enumerate}
			Where $1\leq i\leq n$ and $\ol{f}$ is an $n$.ary
			function symbol.
		\item Non-logical axioms
			\begin{enumerate}
				\item Initial functions
					\begin{enumerate}
						\item $\ol{Z}(v_0)=\0$
						\item $\neg(\0 =\ol S(v_0))$
						\item $\ol S(v_0)=\ol
							S(v_1)\rightarrow
							v_0=v_1$
						\item $\ol
							P^n_i(v_1,\ldots,v_n)=v_1$
							for $1\leq i\leq n$
					\end{enumerate}
				\item Derived functions
					\begin{enumerate}
						\item $\ol
							f(v_1,\ldots,v_n)=\ol
							g(\ol
							h_1(v_1,\ldots,v_n),\ldots,\ol
							h_m(v_1,\ldots v_n)$.
							Here $f$ is defined by
							composition of
							$g,h_1,\ldots, h_m$
						\item Let $f$ be defined by
							primitive recursion
							form the primitive
							recursive function $g$
							and $h$, then the
							following to things
							holds:\\
							$\ol
							f(\0,v_1,\ldots,v_n)=\ol
							g(v_1,\ldots, v_n)$
							and\\
							$\ol f(\ol
							Sv_0,v_1,\ldots,v_n)=\ol
							h(\ol
							f(v_01,v_1,\ldots,v_n),v_0,v_1,\ldots,v_n)$
					\end{enumerate}
				\item Induction\\
					$F(\0)\wedge\forall v(\delta
					v\rightarrow F (\ol Sv))\rightarrow
					\forall vF( v)$ where $F( v)$ is

					\[\exists v_n(\ol
					f(v,v_0,v_1,\ldots,v_n)=\ol 0)\]
			\end{enumerate}
	\end{enumerate}
\end{defi}
The induction axiom seems a bit strange at first. If we had allowed full
induction; i.e induction on every formula, the system we would have defined
would be know as \textbf{PA}; Peano Arithmetics.  Having written down the axioms we will move on to the
inference rules of \PRA.
\begin{defi}
	The inference rules of \PRA\ are the following:
	\begin{enumerate}
		\item From $F,F\rightarrow G,$ derive $G$.
			(modus ponens)
		\item From $F v\rightarrow G$ derive $\exists
			F\rightarrow G$, under the assumption that no
			$v$ occurs free in $G$.
		\item From $G\rightarrow F v$ derive $G\rightarrow
			\forall v F v$, under the assumption that no
			$v$ occurs free in $G$.
	\end{enumerate}
	A formal derivation in \PRA\ is a sequence of formulas of \PRA\
	$F_0,F_1,\ldots,F_k$ such that each $F_i$ is either
	an axiom of \PRA\ or follows from two other formulas $F_j,F_l$
	where $j,l<i$ by one of the three inferences rules.
\end{defi}
We will end this section with a theorem that states that \PRA\ can compute the
primitive recursive functions.
\begin{thm}
	Let $f$ be an $n$-ary primitive recursive function and let $\ol f$ be
	the function symbol representing it in \PRA. Then for all $n_1,\ldots,
	n_m,n\in\omega$ we have:
	\[f(n_1,\ldots,n_m)=n\Rightarrow\PRA\vdash\ol f(\ol{n_1},\ldots,
	\ol{n_m})=\ol n\]
\end{thm}
That \PRA\ can compute the primitive recursive functions makes it possible to
encode the syntax of \PRA. This will be explained in the next section.
\subsection{Arithmetication of the syntax}

In this section we will discus how each expression in the language $F$ of \PRA\ can
be given a unique code-number $\Godelnum{F}$ called the Gödel number. This makes it possible for \PRA\ to express
things about it self. For this encoding to work we will need the fundamental
theorem of arithmetic that states that every natural number $a\geq 2$ has a
unique representation:
\[a=p^{n_0}_{i_0}\cdots p^{n_k}_{i_k}\]
Where each $p$ is a distinct prime and all the $n_i$ are positive. If given a
sequence $(j_0,\ldots j_t)$ we can coded it with a unique code in the following
way:
$$c=2^{j_0+1}3^{j_1+1}\cdots p^{j_t+1}$$
So by this it is possible to give each formula of \PRA\ a unique code. We will
start of by listing the codes for some of our symbols of \PRA:

\begin{enumerate}
	\item $\Godelnum{\0}$ is $(0)$.
	\item $\Godelnum{=}$ is $(1)$.
	\item $\Godelnum{\neg}$ is $(2)$; $\Godelnum{\wedge}$ is $(3)$;
		$\Godelnum{\vee}$ is $(4)$ and $\Godelnum{\rightarrow}$ is
		$(5)$.
	\item $\Godelnum{\forall}$ is $(6)$ and $\Godelnum{\exists}$ is $(7)$.
	\item $\Godelnum{v_i}$ is $(8,i)$.
\end{enumerate}
These are the easy one to gives. The hard ones are the codes for the functions
symbols. We will define the codes for these inductively as seen in the
following table:

\begin{table}[!ht]
	\centering
\begin{tabular}{|l|l|}\hline
	Function  & Index \\\hline
	 $Z(x)=0$  & $(9,1,1)$ \\ \hline
	 $S(x)=x+1$  & $(9,2,1)$ \\\hline
	 $P^n_i(x_1,\ldots,x_n)=x_i$  & $(9,3,n,i)$ \\\hline
	 $f(\vec{x})=g(h_1(\vec{x}),\ldots,h_m(\vec{x}))$
		       & $(9,4,n,m,(g^*,h_1^*,\ldots,,h_m^*)$ \\\hline
	 $f(0,x_1,\vec{x})=g(\vec{x})$  & \\
	 $f(x+1,\vec{x})=h(f(x,\vec{x}),x,\vec{x})$ &
	 $(9,5,n+1,g^*,h^*)$\\ \hline
\end{tabular}
\caption{The Codes for the Function Symbols}
\end{table}

This encoding is primitive recursive so \PRA, can do this encoding. But the proof of that is very tedious, so
it is left out.
A lot of different relations and functions can be shown to be primitive recursive, one of the
important ones is the function $subst(x;y,z)$ which substitutes $z$ for $y$
in the formula $x$. With this it is possible to show primitive recursively
define our axioms of \PRA. This means that \PRA\ is capable of identifying its
own axioms. Further it can also identify that if a given sequence of formulas
is a proof of another given formula; i.e the relation that says "y codes a
derivation of the formula with code x". We shorten this as $\text{Prov}(x,y)$
and we further define the non primitive recursive relation "x codes a provable
formula" as: $\text{Pr}(x)\leftrightarrow\exists y(\text{Prov}(y,x))$
\subsection{The proof predicate and the theorems}

The relation $\text{Pr}(x)$ has the following properties called Löbs
derivability conditions:
\begin{enumerate}
	\item[D1] $\PRA\vdash F\Rightarrow\vdash\text{Pr}(\Godelnum{F})$
	\item[D2] $\PRA\vdash\text{Pr}(\Godelnum{F\rightarrow
		G})\rightarrow(\text{Pr}(\Godelnum{F})\rightarrow\text{Pr}(\Godelnum{G}))$
	\item[D3]
		$\PRA\vdash\Pr(\Godelnum{F})\rightarrow\Pr(\Godelnum{\Pr(\Godelnum{F})})$
\end{enumerate}
it is also clear that $\neg\Pr(\bot)$, where $\bot$ is some false statement
expresses consistence. We will denote this by $\text{Con}$.
We can further with the use of the $subst$ function prove the following
important lemma:
\begin{lem}[The fixed point lemma]
	Given any formula $G$ where the only free variable is $v$, we can
	find a sentence $F$ such that:
	\[\PRA\vdash F\leftrightarrow G(\Godelnum{F})\]
\end{lem}
This lemma makes it possible for us to show the first incompleteness theorem:
\begin{thm}[The First Incompleteness Theorem]
	The predicate $\Pr(\cdot)$ have the following properties for all
	formulas $G$:
	\begin{enumerate}
		\item $\PRA\vdash G\Rightarrow\ \vdash\Pr(\Godelnum{G})$
		\item $\PRA\vdash\Pr(\Godelnum{G})\Rightarrow\ \vdash G$
	\end{enumerate}
	If we let $\PRA\vdash\ F\leftrightarrow\neg\Pr(\Godelnum{F})$, we
	then have:
	\begin{enumerate}
		\item $\PRA\not\vdash F$
		\item $\PRA\not\vdash\neg F$
	\end{enumerate}
	I.e we have a sentence where neither it or its negation can be proven.
	So our system is incomplete.
\end{thm}
Since $\Pr(\cdot)$ fullfills the derivability conditions the second theorem can
also be proven. 
\begin{thm}[The Second Incompleteness Theorem]
	Under the assumption that \PRA\ is consistent we have that $\PRA\not\vdash\text{Con}$
\end{thm}
The main goal for the rest of this project is now to prove that the $\Box$ in
\GL "behaves" in the same way as the Pr($\cdot$) predicate from arithmetics.
How to should be understood will be explained latter on.
To prove this theorem we will first need a main result from Recursion Theory
called the recursion theorem.

We will need one more result about the proof predicate in the project. It is
called the formalized Löb's  theorem, and this theorem together with the
derivability conditions tells the full story about the predicate
$\text{Pr}(\cdot)$
\begin{thm}[Formalized Löb's Theorem]
	Let $F$ be any sentence of \PRA. Then:
	\[\PRA\vdash\text{Pr}(\Godelnum{\text{Pr}(\Godelnum{F})\rightarrow
	F})\rightarrow\text{Pr}(\Godelnum{F})\]
\end{thm}

We will in a later section need the following lemma:

\begin{lem}
	\label{lem:Prov}
	Let $\ol f$ be an $n$-ary primitive recursive function symbol. There is
	a function $g$ depending on $f$, such that:
	\[\PRA\vdash\ol fv_0\ldots v_{n-1}=v\rightarrow\text{Prov}(\ol
		gv_0\ldots v_{n-1},\Godelnum{f\dot v_0\ldots \dot v_{n-1}=\dot
	v_n})\]
\end{lem}

The Formalized Löb's Theorem and the derivability conditions can be shown to
hold for a wide range of arithmetical systems. A overview over this subject
will be given in chapter \ref{chap:PRA}. If we write $\vdash F$ instead of
$\PRA\vdash F$ it means we have not specified the arithmetical system we are
looking at.

If we set the properties of $\text{Pr}(\cdot)$ up next to the properties of the
$\Box$-operator of \GL, we get the following table:

	\begin{table}[h]
		\begin{tabular}{|l|l|}
			\hline
		$\vdash_{\textbf{GL}}\alpha\Rightarrow\
		\vdash_{\textbf{GL}}\Box\alpha $ &
				$\vdash F\Rightarrow\
				\vdash\text{Pr}(\Godelnum{F})$ \\
				\hline
				$\vdash_{\textbf{GL}}\Box(\alpha\rightarrow\beta)\rightarrow(\Box\alpha\rightarrow\Box\beta)
		$ &
				$\vdash\text{Pr}(\Godelnum{F\rightarrow G})\rightarrow
				(\text{Pr}(\Godelnum{F})\rightarrow\text{Pr}(\Godelnum{G}))$ \\
				\hline
				$\vdash_{\textbf{GL}}\Box\alpha\rightarrow\Box\Box\alpha
		$ &
				$\vdash\text{Pr}(\Godelnum{F})\rightarrow\text{Pr}
				(\Godelnum{\text{Pr}(\Godelnum{F})})$ \\
				\hline
				$\vdash_{\textbf{GL}}\Box(\Box\alpha\rightarrow\alpha)\rightarrow\Box\alpha
		$ &
				$\vdash\text{Pr}(\Godelnum{\text{Pr}(\Godelnum{F})\rightarrow
				F})
				\rightarrow\text{Pr}(\Godelnum{F})$\\ \hline
		\end{tabular}
	\caption{The properties of $\Box$ and the predicate $\text{Pr}(\cdot)$}
	\end{table}

	It is clear from this table that the predicate $\text{Pr}(\cdot)$
	and the $\Box$-operator of \GL\ have a lot of the same properties in common.
	Solovay's completeness theorems shows that the modal logic \GL\ in some
	way axiomatize the proof predicate of a wide range of different
	arithmetical systems.
\end{document}
