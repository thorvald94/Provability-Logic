
\documentclass[../main.tex]{subfiles}

\begin{document}
In this chapter we will enrich the language $\mathcal{L}_\Box$ with both more
modal operators or quantifiers. For these languages a number of different
results can be proven.  
In the case with more modal operators the results will be positive; i.e there
are versions of Solovay's  completeness theorems for these kinds of modal logic.
But in the case where we enrich our language with quantifiers our results will
be negative; i.e we cant prove versions of Solovay's completeness theorems for
these kinds of modal logic.

This chapter will skip the majority of the proofs of these
results, but the proofs can be found in \citet{Boolos1993}. The part about
multi-modal logic will also be  explained more in depth than the part about
the quantified modal logic, which will be very short.


\section{Multi-modal provability logic}

We can extend our language $\mathcal{L}_\Box$ with more modality operators. We
will start of by extending it with the modal operator:
$\OneBox$ and its dual $\OneDiamond$. In the next subsection, it will be
explained what we will mean by the $\OneBox$\ . Later on, we will extend the language
with up to $n$ different  modal operators.
\subsection{The system \GLB}

\begin{defi}
	We will say that a fragment of arithmetic  \textbf{T} is $\omega$-inconsistent iff for some
formula $\alpha(x)$, $\textbf{T}\vdash\exists x\alpha(x)$ and for every
$n\in\omega$ we have: $\textbf{T}\vdash\neg\alpha(x)$. \textbf{T} is called
$\omega$-consistent iff it is not $\omega$-inconsistent.
\end{defi}
We have that
if \textbf{T} is $\omega$-consistent then $\textbf{T}\not\vdash\exists x x\not
=x$ and thus \textbf{T} is consistent. So $\omega$-consistency implies
consistency; but the converse does not hold. 

\begin{defi}
	A sentence $\alpha$ is $\omega$-inconsistent in \textbf{T} if the
	axioms of \textbf{T} plus $\alpha$ is $\omega$-inconsistent. $\alpha$
	is $\omega$-consistent iff it is not $\omega$-inconsistent.

	We further say that a sentence $\alpha$ is $\omega$-provable in
	\textbf{T} iff $\neg\alpha$ is $\omega$-inconsistent with $\textbf{T}$.
\end{defi}
It is clear that if $\alpha$ is provable in \textbf{T} then it is
$\omega$-provable in \textbf{T}. We will introduce a new modal system
$\textbf{GLB}$, where \textbf{B} stands for bimodal, where we add two new modal
operators: $\OneBox$ and $\OneDiamond$ to our modal language
$\mathcal{L}_\Box$. We will call this new language $\mathcal{L}_{\oneBox}$.
\begin{defi}
	The language $\mathcal{L}_{\oneBox}$ is the language $\mathcal{L}_\Box$
	extended by the modal operator $\OneBox$, where the syntax $\OneBox$ is
	the same as that of $\Box$. The operator $\OneDiamond$ is defined as
	the dual of $\OneBox$.
\end{defi}
We now define the system of \GLB. We will later discuss the semantics of
\GLB, but this is not of importance at this point.

\begin{defi}
	The axioms of \GLB\ are all tautologies and all sentences of the
	following kind:
	\begin{enumerate}
		\item[AB1]
			$\Box(\alpha\rightarrow\beta)\rightarrow(\Box\alpha\rightarrow\Box\beta)$
		\item[AB2]
			$\OneBox(\alpha\rightarrow\beta)\rightarrow(\OneBox\alpha\rightarrow\OneBox\beta)$
		\item[AB3] $\Box(\Box\alpha\rightarrow\alpha)\Box\alpha$
		\item[AB4] $\OneBox(\OneBox\alpha\rightarrow\alpha)\OneBox\alpha$
		\item[AB5] $\Box\alpha\rightarrow\OneBox\alpha$
		\item[AB6] $\neg\Box\alpha\rightarrow\OneBox\neg\Box\alpha$
	\end{enumerate}
	The deduction rules are the following:
	\begin{enumerate}
		\item[MP] Modus ponens
		\item[Nec] $\Box$  -necessitation; from $\alpha$ infer
			$\Box\alpha$.
	\end{enumerate}
\end{defi}
It is clear that $\OneBox$ -necessitation is a rule of \GLB, since we have that
if $\GLB\vdash\alpha$ then $\GLB\vdash\Box$ and by AB5 we then have
$\GLB\vdash\OneBox\alpha$.

The next goal is to prove an arithmetical soundness theorem for \GLB. For this
end we need the following definition:

\begin{defi}
	We call the following rule the  $\omega$\textit{-rule}: Infer $\forall
	x\alpha(x)$ for all $\alpha(\ol n)$, where $n\in\omega$. A sentence is
	provable under the $\omega$-rule in \textbf{T}, if it belongs to all
	classes containing the axioms of \textbf{T} and closed under MP, Nec and
	the $\omega$-rule.
	We further say that a sentence $F$ is provable in \PRA\ by \textit{one
	application of the} $\omega$\textit{-rule} if for some formula $G(x)$
	we have that $\PRA\vdash G(\ol n)$ for all $n$ and $\PRA\vdash \forall
	x G(x)\rightarrow F$.
\end{defi}
It can be shown that  a sentence $F$ is $\omega$-provable if and only if it is provable
by one application of the $\omega$-rule. With this we can define the notion of
an $\omega$-proof.

\begin{defi}
	A sentence $\forall x G(x)\rightarrow F$ is called an
	$\omega$\textit{-proof} prof of $F$ if $\PRA\vdash G(\ol n)$ for all
	$n$ and $\PRA\vdash\forall x G(x)\rightarrow F$.
\end{defi}
It is clear that if $F$ has an $\omega$-proof then it $\omega$-provable.

We will state one last definition in the same vein as the others above:

\begin{defi}
	$\PRA^+$ is the theory extending \PRA\ with all sentences $\forall x
	G(x)$ such that for every $n\in \omega$ we have that $\PRA\vdash G(\ol
	n)$.
\end{defi}

The following theorem can then be proven about the relationships between the
above definitions:

\begin{thm}
	\label{thm:omega}
	The following are equivalent:
	\begin{enumerate}
		\item $F$ is $\omega$-provable.
		\item $\PRA^+\vdash F$.
		\item $F$ is provable by one application of the $\omega$-rule.
		\item There is an $\omega$-proof of $F$.
	\end{enumerate}
\end{thm}

We need a few more definitions before we can state the arithmetical soundness theorem for
\GLB:
\begin{defi}
	Let $\omega\text{Prov}(y,x)$ be the formula of \PRA\ the formalizes
	that $y$ is the code of  an $\omega$-proof of $x$, and let
	$\omega\text{Pr}(x)=\exists y\omega\text{Prov}(y,x)$
\end{defi}
We then have that $\omega\text{Pr}(x)$ by theorem \ref{thm:omega} is provable
coextensively with the formulas stating in \PRA\ the following properties:
"$\omega$-provable", "provable in $\PRA^+$ and "provable by one application of
the $\omega$-rule". This means that for each sentence $F$ in \PRA\ that we
have:
\[\PRA\vdash\text{Pr}(\Godelnum{F})\rightarrow\omega\text{Pr}(\Godelnum{F})\]
With these results we can now extend the interpretation of $\mathcal{L}_{\Box}$
in \PRA\ to one of $\mathcal{L}_{\OneBox}$:
\begin{defi}
	An interpretation $^*$ of $\mathcal{L}_{\OneBox}$ in \PRA\ is an
	extension of an
	interpretation $\mathcal{L}_\Box$ in \PRA\ where we add the following:
	\[\OneBox(\alpha)^*=\omega\text{Pr}(\Godelnum{\alpha^*})\]
\end{defi}
We can now state the arithmetical soundness theorem for \GLB:
\begin{thm}[The aritmetical soundness theorem for \GLB]
	For any modal formula $\alpha$ and interpretation of
	$\mathcal{L}_{\OneBox}$ in \PRA\ we have that:
	\[\vdash_\GLB\alpha\rightarrow\ \PRA\vdash\alpha^*\]
\end{thm}
We omit the proof here. We can not prove the arithmetical completeness theorem
for \GLB\ without the help of another multi modal logic.

This is because there is no good Kripke semantics for the logic \GLB. We will
start of by defining a Kripke modal for the logic \GLB; this definition is an
extension of a "normal" frame, but with two relations instead of one.

\begin{defi}
	A  frame for a modal logic with two modal operators is a triple
	$\mathcal{H}=\la W,R,R_1\ra$ where both $R$ and $R_1$ are relations on
	$W$. A model is a quadruple $\mathcal{K}=\la W, R,R_1,\phi\ra$ were
	$\phi$ is a valuation function on $W$. The truth of modal formula
	$\alpha$ in a given node $w\in W$ is defined in the obvious way, and
	the two main clauses of the definition are:
	\begin{enumerate}
		\item $\vdash_w^\mathcal{K}\Box\alpha$ iff for all $v\in W$ such
			that $wRv$ we have $\vdash_v^\mathcal{K}\alpha$
		\item $\vdash_w^\mathcal{K}\OneBox\alpha$ iff for all $v\in W$ such
			that $wR_1v$ we have $\vdash_v^\mathcal{K}\alpha$
	\end{enumerate}
\end{defi}
the problem here is that it can be shown that no matter what, the relation
$R_1$ 
will be the empty relation W and therefore $\OneBox\bot$  will be valid and thus be a 
contradiction to X from the arithmetical soundness of \GLB. Giorgie
Dzhaparidze found a way around this by looking at another multi modal logic,
which will be introduced in the next section. His result was improved  by
Konstantin Ignatiev.

\subsection{The system \textbf{IDzh} and completeness of \GLB}

Since \GLB\ does not have any good Kripke semantic we will look at a multi modal logic that
has such a semantic; we will call this modal logic for \textbf{IDzh} after Ignatiev
and Dzhaparidze.

\begin{defi}
	The language of \textbf{IDzh} is the same as \GLB. Further the axioms
	of \IDzh\ are all tautologies and the following modal formulas:
	\begin{enumerate}
		\item[AI1]
			$\Box(\alpha\rightarrow\beta)\rightarrow(\Box\alpha\rightarrow\Box\beta)$
		\item[AI2]
			$\OneBox(\alpha\rightarrow\beta)\rightarrow(\OneBox\alpha\rightarrow\OneBox\beta)$
		\item[AI3] $\Box(\Box\alpha\rightarrow\alpha)\rightarrow\Box\alpha$
		\item[AI4] $\OneBox(\OneBox\alpha\rightarrow\alpha)\rightarrow\OneBox\alpha$
		\item[AI5] $\Box\alpha\rightarrow\OneBox\Box\alpha$
		\item[AI6] $\neg\alpha\rightarrow\OneBox\neg\Box\alpha$.
	\end{enumerate}
	The deduction rules are the following:
	\begin{enumerate}
		\item[MP] Modus ponens 
		\item[Nec] $\Box$-necessitation
		\item[Nec$_1$] $\OneBox$-necessitation
	\end{enumerate}
	Here we add the inference Nec$_1$ since this rule can not be proven
	from the other rules and axioms of this modal logic.
\end{defi}
The modal logic \IDzh\ can be shown to be weaker than \GLB\ and thus
$\IDzh\subseteq\GLB$.

The multi modal logic \IDzh\ has the following Kripke semantic, for which a soundness and
completeness theorem can be proven.
\begin{defi}
	An \IDzh-model is a quadruple $\mathcal{K}=\la W,R,R_1,\phi\ra$, where
	$W$ is a finite non-empty set, $\phi$ is a valuation function $W$, and
	$R$ and $R_1$ are transitive irreflexive relations on $W$ such that for
	all $w,v_1,v_2\in W$ we have that:
	\[\text{If}\ wR_1v_1\ \text{then}\ wRv_2\ \text{if and only if}\
	v_1Rv_2\]
\end{defi}
This means that we can not have that $wRv$ and $wR_1v$ since then $vRv$;
contradicting the irreflexivity $R$.

For such models the following theorem can be proven:
\begin{thm}
	$\alpha$ is valid in all \IDzh-models if and only if
	$\vdash_\IDzh\alpha$.
\end{thm}
This theorem shows that \IDzh\ has a natural Kripke semantic to which it is
sound and complete.

We will need the following definitions to state the theorem from which the
arithmetical completeness theorem for \GLB\ follows:
\begin{defi}
	For any modal formula $\alpha$ we define $\Delta\alpha$ as the formula:
	\[\alpha\wedge\Box\alpha\wedge\OneBox\alpha\wedge\Box\OneBox\alpha\]
	We define the formula $\Psi\alpha$ as:
	\[\bigwedge_{\Box\beta\in
	S(\alpha)}\Delta(\Box\beta\rightarrow\OneBox\beta)\]
\end{defi}
It can be shown that $\vdash_\GLB\Psi\alpha$.
To show completeness of \GLB\ the following three statements should be proven to
be equivalent:
\begin{enumerate}
	\item $\vdash_{\textbf{IDzh}}\Psi\alpha\rightarrow\alpha$
	\item $\vdash_\GLB\alpha$
	\item $\PRA\vdash\alpha^*$ for all $^*$
\end{enumerate}
We have that $(1)\Rightarrow (2)$ since $\IDzh\subset\GLB$ and the fact that
$\vdash_\GLB\Psi\alpha$. That $(2)\Leftrightarrow(3)$ is just the arithmetical
soundness theorem for \GLB. The proof of $(3)\Rightarrow(1)$ will obviously
prove the arithmetical completeness theorem for \GLB.
The proof of this can be found in \citet{Boolos1993} and is beyond the scope of
this project.

We can also look at the multi modal logic \textbf{GLSB} which is the modal logic that have all
the theorems of \GLB\  and all formulas:
$\OneBox\alpha\rightarrow\alpha$ as axioms, and which only has modus ponens as
its rule of inference. A variant of Solovay's second completeness theorem can
be shown for this multi modal logic.

\subsection{The system \textbf{GLP}}
We will end this section by introducing the system \textbf{GLP}.
The language $\mathcal{L}_{[n]}$ of \textbf{GLP} is an extension of
$\mathcal{L}_\Box$, where instead of $\Box$ we write $[0]$ and we further add
a countable infinite amount of boxes:
$[1],[2],\ldots$ representing provability in $\PRA^+$, $\PRA^{++}$ and so on,
where $\PRA^{++}$ is  $\PRA^+$ with all formulas of the form $\forall x G(x)$
such that for every $n\in\omega$ such that $\PRA^+\vdash G(\ol n)$ added. We
define their duals $\la 0\ra,\la 1\ra, \la 2\ra,\ldots$ in the obvious way and
these represent consistency, $\omega$-consistency,
$\omega$-$\omega$-consistency and so on. We can now define the system
\textbf{GLP}:
\begin{defi}
	The axioms and inference rules of \textbf{GLP} are the following:
	\begin{enumerate}
		\item[A1]
			$[n](\alpha\rightarrow\beta)\rightarrow([n]\alpha\rightarrow[n]\beta)$
		\item[A2] $[n]([n]\alpha\rightarrow\alpha)\rightarrow[n]\alpha$
		\item[A3] $[n]\alpha\rightarrow[n+1]\alpha$
		\item[A4] $\neg[n]\alpha\rightarrow[n+1]\neg[n]\alpha$
		\item[MP] Modus Ponens
		\item[Nec$_0$] $[0]$-necessiation
	\end{enumerate}
\end{defi}

Following Dzhaparidze the arithmetical completeness theorem can be proven for
this system of multi modal logic. The proof follows that of the proof of the
arithmetical completeness theorem for \GLB.

Again we can look at the multi modal logic \textbf{GLSP} which has all
theorems and sentences $[n]\alpha\rightarrow\alpha$ as its axioms and which
only has modus ponens as it inference rule. Again a variant of Solovay's
second completeness theorem holds for this logic.

So Solovay's completeness theorems generalize nicely to multi modal logics;
i.e the concept of $\omega$-provability (and beyond) can also be axiomatized.


The proofs of these results are rather involved, but they still follow the same
idea as Solovay's original idea. This shows how deep the concept of embedding
modal logic into arithmetic is.
But this fundamental idea does not generalize to
all extensions of \GL, something which will be discussed in the next section.

\section{Quantifed provability logic}
There is another way to extend the logic \GL; we can add quantifiers and
predicates to the
language. We will call this logic  quantified modal logic (\textbf{QML}). The first
goal is to define what a formula is:
\begin{defi}
	$\alpha$ is a formula of \textbf{QML} if and only if it can be obtained from a
	formula of first order logic by
	replacing occurrences of the negation sign "$\neg$" with occurrences of
	$\Box$. 
\end{defi}

	There is not a version of Solovay's completeness theorems for this
	logic; i.e it is not arithmetically complete with respect to any
	fragment of arithmetic. Furthermore this logic  does not have the fixed point property and
	is not complete with respect to any class of Hintikka frames. These
	results can actually be shown to hold for an even more conservative
	extension of \GL: one where we only one place predicates are
	allowed and where we do not permit nestings of the $\Box$-operator.
	These results makes sense, since first-order modal logic generally do
	not have the same nice properties as propositional modal logic.
	Further
	information on this topic and proofs of these statements can be found
	in \citet{Boolos1993}, but this goes way beyond the scope of this
	project and has been left out.

	A list of even more results of provability logic can be found
	in \citet{sep-logic-provability} and \citet{ArteBe}, and these
	are even more beyond the scope of this project than the rest of this
	chapter. Further areas of mathematics where modal logic is used is
	discussed in \citet{Artemov2007}.
	\end{document}
