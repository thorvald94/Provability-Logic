
\documentclass[../main.tex]{subfiles}

\begin{document}
\section{Multi-modal provability logic}
\textbf{Evt. bare forklar kort hvad de vigtige definitioner og resultater er i
dette kapitel}

We can extend our language $\mathcal{L}_\Box$ with more modality operators. We
will start of by extending it with the modal operator:
$\OneBox$ and its dual $\OneDiamond$. In the next subsection it will be
explained what we will mean with the $\OneBox$. Latter on we will extend the language
with even more modal operators.
\subsection{The system \GLB}

\begin{defi}
We will say that a theory \textbf{T} is $\omega$-inconsitsent iff for some
formula $\alpha(x)$, $\textbf{T}\vdash\exists x\alpha(x)$ and for every
$n\in\omega$ we have: $\textbf{T}\vdash\neg\alpha(x)$. \textbf{T} is called
$\omega$-consistent iff it is not $\omega$-inconsistent.
\end{defi}
We have that
if \textbf{T} is $\omega$-consistent then $\textbf{T}\not\vdash\exists x x\not
=x$ and thus \textbf{T} is consistent. So $\omega$-consistency implies
consistency; but the converse does not hold. \textbf{Hvis at det ikke holde}. 

\begin{defi}
	A sentence $\alpha$ is $\omega$-inconsistent in \textbf{T} if the
	axioms of \textbf{T} plus $\alpha$ is $\omega$-inconsistent. $\alpha$
	is $\omega$-consistent iff it is not $\omega$-inconsistent.

	We further say that a sentence $\alpha$ is $\omega$-provable in
	\textbf{T} iff $\neg\alpha$ is $\omega$-inconsistent with $\textbf{T}$.
\end{defi}
It is clear that if $\alpha$ is provable in \textbf{T} then it is
$\omega$-provable in \textbf{T}. We will introduce a new modal system
$\textbf{GLB}$, where \textbf{B} stands for bimodal, where we add two new modal
operators: $\OneBox$ and $\OneDiamond$ to our modal language
$\mathcal{L}_\Box$. We will call this new language for $\mathcal{L}_{\oneBox}$
\begin{defi}
	The langauge $\mathcal{L}_{\oneBox}$ is the language $\mathcal{L}_\Box$
	extend with the modal operator $\OneBox$, where the syntax $\OneBox$ is
	the same as that of $\Box$. The operator $\OneDiamond$ is defined as
	the dual of $\OneBox$.
\end{defi}
We now define the system of \GLB. We will later on discuss the semantics of
\GLB, but this is not of important right now.

\begin{defi}
	The axioms of \GLB\ are all tautologies and all sentences of the
	following kind:
	\begin{enumerate}
		\item[A1]
			$\Box(\alpha\rightarrow\beta)\rightarrow(\Box\alpha\rightarrow\Box\beta)$
		\item[A2]
			$\OneBox(\alpha\rightarrow\beta)\rightarrow(\OneBox\alpha\rightarrow\OneBox\beta)$
		\item[A3] $\Box(\Box\alpha\rightarrow\alpha)\Box\alpha$
		\item[A4] $\OneBox(\OneBox\alpha\rightarrow\alpha)\OneBox\alpha$
		\item[A5] $\Box\alpha\rightarrow\OneBox\alpha$
		\item[A6] $\neg\Box\alpha\rightarrow\OneBox\neg\Box\alpha$
		\item[R1] Modus ponens
		\item[R2] $\Box$  -necessitation; from $\alpha$ infer
			$\Box\alpha$.
	\end{enumerate}
\end{defi}
\textbf{Kommenter på disse regler}

It is clear that $\OneBox$ -necessitation is a rule of \GLB, since we have that
if $\GLB\vdash\alpha$ then $\GLB\vdash\Box$ and by A5 we then have
$\GLB\vdash\OneBox\alpha$.

The next goal is to prove an arithmetical soundness theorem for \GLB. For this
end we need the following definition:

\begin{defi}
	We call the following rule for the  $\omega$-rule: Infer $\forall
	x\alpha(x)$ for all $\alpha(\ol n)$, where $n\in\omega$. A sentence is
	provable under the $\omega$-rule in \textbf{T} if it belongs to all
	classes containing the axioms of \textbf{T} and closed under R1, R2 and
	the $\omega$-rule.
\end{defi}

\subsection{The problem with \GLB\ and its semantics}

\subsection{The system \textbf{IDzh} and completeness of \GLB}

\subsection{The system \textbf{GLP}}
We will end this section by introducing the system \textbf{GLP}
The language $\mathcal{L}_{[n]}$ of \textbf{GLP} is an extension of
$\mathcal{L}_\Box$, where instead of $\Box$ we write $[0]$ and we furhter add
a countable infinite amount of boxes:
$[1],[2],\ldots$ representing provability in $\PA^+$, $\PA^{++}$ and so on. We
define their duals $\la 0\ra,\la 1\ra, \la 2\ra,\ldots$ in the obvious way and
these represent consistency, $\omega$-consistency,
$\omega$-$\omega$-consistency and so on. We can now define the system
\textbf{GLP}:
\begin{defi}
	The axioms and inferences rules of \textbf{GLP} are the following:
	\begin{enumerate}
		\item[A1]
			$[n](\alpha\rightarrow\beta)\rightarrow([n]\alpha\rightarrow[n]\beta)$
		\item[A2] $[n]([n]\alpha\rightarrow\alpha)\rightarrow[n]\alpha$
		\item[A3] $[n]\alpha\rightarrow[n+1]\alpha$
		\item[A4] $\neg[n]\alpha\rightarrow[n+1]\neg[n]\alpha$
		\item[R1] Modus Ponens
		\item[R2] $[0]$-necessiation
	\end{enumerate}
\end{defi}

\section{Quantifed provability logic}

\end{document}
