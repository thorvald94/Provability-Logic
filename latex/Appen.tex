\documentclass[../main.tex]{subfiles}

\begin{document}
\chapter{Neccesation free definition of \GL\ and \GLS}

There is another way to define the modal logic \GLS. The definition of this
section is the one Smorynski uses in \cite{Smor1985}

In this section we will take \textbf{K4} as our base logic and see \GL\  as an
extension of this.
We will need the following notation to
define this. Given a set $\Gamma$:
\begin{enumerate}
	\item $\Gamma\vdash_{\textbf{MP}}\alpha$ if $\alpha$ is derivable form $\Gamma$ by
		only using $R1$
	\item $\Gamma\vdash_{\textbf{Nec}}\alpha$ if $\alpha$ is derivable form $\Gamma$ by
		using $R1$ and $R2$.
\end{enumerate}
We will further define the following set for any set $\Gamma$:
\[
	\Gamma^{\textbf{Nec}}=\Gamma\cup\{\Box\beta:\beta\in\Gamma\}
\]
This set allows us to avoid Nec by the axioms of \textbf{K4}, which the
following lemma shows:
\begin{lem}
	Let $\Gamma$ include all the instances  of axioms of \textbf{K4}. Then:
	\[
		\Gamma\vdash_{\textbf{Nec}}\alpha\Leftrightarrow\Gamma^{\textbf{Nec}}\vdash_{\textbf{MP}}\alpha
	\]
\end{lem}
\begin{proof}
	The way $\Leftarrow$ is trivial. So we will only show $\Rightarrow$.
	To prove this, it is enough to show that the set of theorems of
	$\Gamma^{\textbf{Nec}}$ is closed under $R2$; i.e to show that:

\[\Gamma^{\textbf{Nec}}\vdash\alpha\Rightarrow\Gamma^{\textbf{Nec}}\vdash_{\textbf{MP}}\Box\alpha\]

Let $\Gamma^{\textbf{Nec}}\vdash_{\textbf{MP}}\alpha$. Then we have a sequence
	$\alpha_1,\ldots,\alpha_n=\alpha$, where for each $1\leq i\leq n$ we
	have that $\alpha_i$ is one of the following three:
	\begin{enumerate}
		\item We have that it is an axiom $\alpha_i\in\Gamma$ 
		\item It is the necessitation  $\Box\beta_1$ of an axiom
			$\beta_1\in\Gamma$
		\item It is a consequence of \textbf{MP} of
			$\alpha_j,\alpha_k=\alpha_j\rightarrow\alpha_i$
			where $j,k<i$.
	\end{enumerate}
	We will show  by induction on the length $i$ of a 'initial' segment
	$\alpha_1,\ldots,\alpha_i$ that
	$\Gamma^{\textbf{Nec}}\vdash_{\textbf{MP}}\Box\alpha_i$. 
	\begin{enumerate}
		\item If we have that $\alpha_i\in\Gamma$. Then by definition
			we have that $\Box\alpha_i\in\Gamma^{\textbf{Nec}}$ and thus
			$\Gamma^{\textbf{Nec}}\vdash_{\textbf{MP}}\Box\alpha_i$.
		\item If $\alpha_i=\Box\beta_i$ for some $\beta_i\in\Gamma$,
			then we can make the following deduction:
			\begin{align*}
				\Gamma^{\textbf{Nec}}&\vdash_{\textbf{MP}}\Box\beta_i &\text{Since}\
				\Box\beta_i\in\Gamma^{\textbf{Nec}}\\
						     &\vdash_{\textbf{NP}}\Box\beta_i\rightarrow\Box\Box\beta_i
					&\text{By}\ \textbf{4}\\
					&\vdash_{\textbf{MP}}\Box\Box\beta_i\ (=\Box\alpha_i) 
					&\text{By}\ \textbf{MP}
			\end{align*}
			Which shows the result.
		\item If $\alpha_i$ follows form
			$\alpha_j,\alpha_k=\alpha_j\rightarrow\alpha_i$ by
			\textbf{MP} we have by the induction hypothesis that
			$\Gamma^{\textbf{Nec}}\vdash_{\textbf{MP}}\Box\alpha_j$ and
			$\Gamma^{\textbf{Nec}}\vdash_{\textbf{MP}}\Box(\alpha_j\rightarrow\alpha_i)$. By
			\textbf{K} we also have that
			$\Gamma^2\vdash_{\textbf{MP}}\Box\alpha_j\wedge\Box(\alpha_j\rightarrow\alpha_i)\rightarrow\Box\alpha_i$
			and by propositional logic we get that:
			$\Gamma^2\vdash_{\textbf{MP}}\Box\alpha_i$.
	\end{enumerate}
	This ends the induction and the lemma will follow.
\end{proof}
By this lemma we can get a \textbf{Nec}-free definition of \textbf{K4} and \GL\  that we will call
$\textbf{K4}^{\textbf{Nec}}$ and $\GL^{\textbf{Nec}}$. We can further by the deduction theorem from propositional logic get
the following:

\begin{lem}
	$\Gamma\vdash\alpha$ iff there is a finite set
$\{\gamma_0,\ldots,\gamma_{n-1}\}\subseteq \Gamma$ such that:
$K\vdash\wedge\gamma_i\rightarrow\alpha$
\end{lem}

From all this we can define \GLS\ in an alternative way:
\begin{defi}
	The system of modal logic \text{\textbf{GLS}} is defined as the system
	of propositional modal logic that have all the theorems of
	$\GL^{\textbf{nec}}$\ and
	 addition of the axiom of reflexion: 
	 \begin{enumerate}
		 \item[\textit{Refl}:]$\Box\alpha\rightarrow\alpha$ 
 	\end{enumerate}
	and which sole rule of inference is \textit{modus ponens}. 
\end{defi}
\end{document}
