\documentclass[../main.tex]{subfiles}

\begin{document}
In this section we will prove the so called \textit{fixed point theorem} for
\GL.  This theorems says that there for some sentences of \GL\ exists fixed
points. There has been a lot of different ways to prove this theorem; Boolos
list three different ways in his book \cite{Boolos1993} Here we
will follow the proof from \cite{Olson1990}.


Before stating the theorem, we will have to state the following 
definitions:

\begin{defi}
	We abbreviate $\Box\alpha\wedge\alpha$ as $\sBox\alpha$ for every
	$\alpha$ in our langauge.
\end{defi}
$\sBox$ is called "strong box".
\begin{remark}
	\label{rem:acc}
By our semantic of modal logic, this we have that $\vDash_w\sBox\alpha$ is true iff $\vDash_v\alpha$
for all $v\in\{w\}\cup\text{acc}(w)$, where $\text{acc}(w)$ are $\text{acc}(w)$
is the collection of "states" that can be "seen" from $w$, i.e:
$\text{acc}=\{v\in W:wRv\}$
\end{remark}
Another useful definition that we can make, since we for every Kripke model
$\mathcal{K}=\la W,R,\phi\ra$ of \GL\  have that $W$ is a finite set is the following:
\begin{defi}
	Let $\mathcal{K}=\la W,R,\phi\ra$ be a Kripke model of \GL. The
	\textit{rank} of the notes $w\in W$ is defined in the following
	way: $\text{rank}(w)=0$ iff there is no world $v$ such that
	$wRv$. Otherwise we have that
	$\text{rank}(w)=1+\text{max}\text{rank}(v):wRv\}$.
\end{defi}
Since $W$ is finite and that $R$ is irreflexive we have that the for each $w\in
W$ the $\mathcal{K}$-rank of $w$ is unique.

We will need the following three lemmas in our proof of the Fixed Point Theorem:
\begin{lem}
	\label{lem:acc}
	Given any Kripke model $\mathcal{K})\la W,R,\phi\ra$ of \GL, $w\in W$ and
	sentence $\alpha$, we have that if $\vDash_w\Box\alpha$ then
	$\vDash_v\Box\alpha$ for any $v\in\text{acc}(w)$. Further we have that
	$\vDash_v\sBox\alpha$ for all $x\in\text{acc}(w)$.
\end{lem}
\begin{proof}
	Assume that we have $\vDash_w\Box\alpha$, $w$ sees  $v$ and
	$v$ sees $v'$. Since $R$ is transitive we have that $w$ is also
	connected to $v'$ and thus we have that $\vDash_{v'}\alpha$. Since
	$v'$ was chosen at random we have that $\vDash_v\Box\alpha$. We also
	have that $\vDash_v\alpha$ and thus we have $\vDash_v\sBox\alpha$.
\end{proof}
\begin{lem}
	\label{lem:con}
	Given any Kripke model $\mathcal{K}=\la W,R,\phi\ra$, $w\in W$ and
	sentence $\alpha$, if $\not\vDash_w\Box\alpha$ then there is "notes"
	$v$ connected to $w$ such that $\vDash_v\Box\alpha$ and
	$\not\vDash_v\alpha$.
\end{lem}
\begin{proof}
	Assume that $\not\vDash_w\Box\alpha$ then there is a notes $v$
	connected to $w$ such that $\not\vDash_v\alpha$. Let $v$ be the notes
	with the least rank with this property and suppose that $vRv'$. Since
	$v'$ is of less rank than $v$ we have that $\vDash_{v'}\alpha$. Now
	since $v'$ was chosen arbitrarily we have that $\vDash_v\Box\alpha$ and
	the lemma is proven.
\end{proof}
The next lemma will be crucial in the proof of the fixed-point theorem.
\begin{lem}[Semantic Substitution Lemma]
	\label{lem:sem}
	For any sentences $\alpha,\beta$ and $\gamma$ we have that the
	following formula is valid in all models of \GL:
	$$\sBox(\beta\leftrightarrow\gamma)\rightarrow(\alpha(\beta)\leftrightarrow\alpha(\gamma))$$
\end{lem}
The meaning of $\alpha(\beta)$ is that we replace every occurrences of $p$ in
$\alpha$ with $\beta$; the meaning of $\alpha(\gamma)$ is similar.

\begin{proof}
	We start of by fixing $\beta$ and $\gamma$. The proof will be by
	induction on the complexity of $\alpha$. We will only proof the part
	where $\alpha$ is $\Box\sigma$.

	So suppose that $\alpha$ is $\Box\sigma$, where
	$\sBox(\beta\leftrightarrow\gamma)\rightarrow(\sigma(\beta)\leftrightarrow\sigma(\gamma))$
	is valid. Let $\mathcal{K}$ be any model of \GL\ and let $w\in W$.
	Suppose that $\vDash_w\sBox(\beta\leftrightarrow\gamma)$. Let $v$ be
	any world seen by $w$. Then by lemma \ref{lem:acc} we have
	$\vDash_v\sBox(\beta\leftrightarrow\gamma)$. Since 
	$\sBox(\beta\leftrightarrow\gamma)\rightarrow(\sigma(\beta)\leftrightarrow\sigma(\gamma))$
	is valid we get that
	$\vDash_v\sigma(\beta)\leftrightarrow\sigma(\gamma)$ and since
	$v$ was chosen arbitrary we get
	$\vDash_w\Box(\sigma(\beta)\leftrightarrow\sigma(\gamma))$. By
	proposition \ref{prop:GL} and weak completeness we can conclude
	$\vDash_w\Box\sigma(\beta)\leftrightarrow\sigma(\gamma)$.
\end{proof}

We are now almost ready to state state and prove the fixed-point theorem. We
will just need the next to definitions:
\begin{defi}
	A sentence $\alpha$ is called modalized in $p$ if every occurence of
	$p$ in  $\alpha$ is under the scope of $\Box$.
\end{defi}
We will also need the following definition:
\begin{defi}
	A sentence $\alpha$ is said to be $n$-\textit{decomposable} iff for some
	sequence $q_1,\ldots,q_n$ consisting of distinct senctence letters that
	do not occur in $\alpha$ we have some sentence $\beta(q_1,\ldots,q_n)$
	that do not contain $p$ and another sequence of distinct sentences 
	$\gamma_1(p),\ldots,\gamma_n(p)$, which each contains $p$ that we have
	$$\alpha=\beta(\gamma_1(p),\ldots,\gamma_n(p))$$.
\end{defi}
It should be noted that if $\alpha$ is modalized in $p$ that we then have that
$\alpha$ is $n$-decomposable for some $n$.


We can now state the fixed point theorem.
\begin{thm}
	If $\alpha$ is modalized in $p$, then there exists a formula $\sigma$
	in which the only sentence letters that occurs are these other than $p$
	that occur in $\alpha$, and such that:
	\[\sBox(p\leftrightarrow\alpha)\rightarrow(p\leftrightarrow\sigma)\]
	The formula $\sigma$ is called a \textit{fixed-point} of $\alpha$.
\end{thm}
We will later on proof that $\vdash_\GL\sigma\leftrightarrow\alpha(\sigma)$ for
such a $\sigma$ and thus the name fixed point makes sense.
\begin{proof}
	We will prove this by showing that if $\alpha$ is $n$-decomposable
	then it has a fixed point. We will show this by induction on $n$.

	\textbf{Base case:} Suppose that $\alpha$ is $0$-decomposable. Then we
	have that $p$ dose not occur in $\alpha$ and it can thus itself be the
	sentence $\beta$.

	\textbf{Induction step:}
	Assume that every sentence that is $n$-decomposable has a fixed point.
	We now have to show that every sentence that is $(n+1)$-decomposable
	also has a fixed point. To show this we will assume the following:
	$$\alpha(P)=\beta(\Box\gamma_1(p),\ldots,\Box_{n+1}(p))$$
	Further for each $i$ let:
	$$\alpha_i(p)=\beta(\Box\gamma_1(p),\ldots,\Box\gamma_{i-1},\top,\Box\gamma_{i+1},\ldots,\Box\gamma_{n+1}(p))$$
	Thus we have that for each $i$ that $\alpha(p)$ is $n$-decomposable,
	so it has a fixed point, that we call $\sigma_i$. Lastly we define:
	$$\sigma=\beta(\Box\gamma_1(\sigma_1),\ldots,\Box\gamma_{n+1}(\sigma_{n+1}))$$
	Our goal is to show that $\sigma$ is a fixed point of $\alpha$.
	\begin{lem}
		\label{lem:fix}
		For each $i$ we have that:
		$$\vdash_\GL\sBox(p\leftrightarrow\alpha)\rightarrow\sBox\Box\gamma_i(p)\leftrightarrow\gamma_i(\sigma_i))$$
	\end{lem}
	\begin{proof}
		Since we have that \GL\ is complete, we just ave to show that
		for any model $\mathcal{K}=\la W,R,\phi\ra$ and any $w\in W$
		that:
		\begin{equation}
		\label{eqn:1}
			\mathcal{K}\vDash\sBox(p\leftrightarrow\alpha)\rightarrow\sBox(\Box\gamma_i(p)\leftrightarrow
		\Box\gamma_i(\sigma_i))
	\end{equation}
		So we will start of by fixing $i$, $\mathcal{K}$ and $w\in W$.
		We will show \ref{eqn:1} by assuming
		$\vDash_w\sBox(p\leftrightarrow\alpha)$ and then deduce:
		$\vDash_w\sBox(\Box\gamma_i(p)\leftrightarrow\Box\gamma_i(\sigma_i))$;
		this is equivalent to $\vDash_v\Box
		\gamma_i(p)\leftrightarrow\Box\gamma_i(\sigma_i)$ for all
		$v\in\{w\}\sup\text{acc}(w)$ by remark \ref{rem:acc}. So let
		$v\in\{w\}\cup\text{acc}(w)$ and assume that
		$\vDash\Box\gamma_i(p)$, i.e
		$\vDash_v\Box\gamma_i(p)\leftrightarrow\top$. By lemma
		\ref{lem:acc} we have that for any $v'\in\text{acc}(v)$ that
		$\vDash_{v'}\Box\gamma_i(p)$ and thus
		$\vDash_{v'}(\Box\gamma_i(p)\leftrightarrow\top$. This means
		that we have:
		$$\vDash_v\sBox(\Box\gamma_i(p)\leftrightarrow\top)$$
		And thus by lemma \ref{lem:sem} we get that
		$\vDash_v\alpha_i\leftrightarrow\alpha$ and since our $v$ was
		chosen arbitrarily we have that
		$\vDash_w\Box(\alpha_i\leftrightarrow\alpha)$ and thus by
		lemma \ref{lem:acc} we get that:
		$\vDash_v\sBox(\alpha_i\leftrightarrow\alpha)$. Since we have
		assumed that $\vDash_w\sBox(p\leftrightarrow\alpha)$ we again
		have by lemma \ref{lem:acc} that
		$\vDash_v\sBox(p\leftrightarrow\alpha)$, and hence we have
		$\vDash_v\sBox(p\leftrightarrow\alpha_i)$. Since our logic is
		complete and we have assumed by the induction hypothesis that
		$\alpha_i$ has a fixed point $\sigma_i$ we have that
		$\vDash_v(p\leftrightarrow\gamma_i)$, and thus, since $v$ was
		chosen arbitrarily we have that
		$\vDash_w\Box(p\leftrightarrow\gamma_i)$. So by using
		\ref{lem:acc} again we get that
		$\vDash_v\sBox(p\leftrightarrow\gamma_i)$. We will now use
		lemma \ref{lem:sem} again and get that:
		\begin{equation}
			\label{eqn:sub1}
			\vDash_v\gamma_i(p)\leftrightarrow\gamma_i(\sigma_i)
		\end{equation}
		and
		\begin{equation}
			\label{eqn:sub2}
			\vDash_v\Box\gamma_i(p)\leftrightarrow\Box\gamma_i(\sigma_i)
		\end{equation}
		Notice that these two holds for any $v\in\{w\}\cup\text{acc}(w)$ such that
		$\vDash_v\Box\gamma_i(p)$.
		Further by \ref{eqn:sub2} we can deduce
		$\vDash_v\Box\gamma_i(p)\rightarrow\Box\gamma_i(\sigma_i)$

		For the next step of the prove of this lemma we will assume that
		$\not\vDash_v\Box\gamma_i(p)$. This means by lemma
		\ref{lem:con} that there is some world $v'$ where
		$v'\in\text{acc}(v)$, such that $\not\vDash_{v'}\gamma_i(p)$
		and $\vDash_{v'}\Box\gamma_i(p)$. \ref{eqn:sub1} holds for $v'$
		since $v'\in\{w\}\cup\text{acc}(w)$ and thus we have
		$\vDash_{v'}\gamma_i(p)\leftrightarrow\gamma_i(\sigma_i)$. This
		gives that $\not\vDash_{v'}\gamma_i(\sigma_i)$ and thus since
		$vRv'$ we have that $\not\vDash_v\Box\gamma_i(\sigma_i)$. By
		contraposition we then get:
		$\vDash_v\Box\gamma_i(\sigma_i)\rightarrow\Box\gamma_i(p)$, and
		thus we have shown that:
		$$\vDash_v\Box\gamma_i(p)\leftrightarrow\Box\gamma_i(\sigma_i)$$
		We have now shown the lemma.
	\end{proof}
	We now go back and finish our proof of the fixed point theorem. Suppose
	that $\mathcal{K}$ is a model and that $w\in W$ such that
	$\vDash_w\sBox(p\leftrightarrow\alpha)$. By lemma \ref{lem:fix} and
	completeness we get
	$\vDash_w\sBox(\Box\gamma_i(p)\leftrightarrow\Box\gamma_i(\sigma_i))$.
	By using lemma \ref{lem:sem} $(n+1)$ times we can deduce that:
	$$\vDash_w\beta(\Box\gamma_i(p),\ldots,\Box\gamma_{n+1}(p))\leftrightarrow\beta(
	\Box\gamma_1(\sigma_1),\ldots,\Box\gamma_{n+1}(\sigma_{n+1}))$$
	i.e $\vDash_w\alpha\leftrightarrow\sigma$. 

	Since we have $\vDash_w p\leftrightarrow\alpha$ we get $\vDash_w
	p\leftrightarrow\sigma$, we can obtain
	$\vDash_w\sBox(p\leftrightarrow\alpha)\rightarrow(p\leftrightarrow\sigma)$.
	Since our $\mathcal{K}$ and $w$ was chosen at random we have that
	$\sBox(p\leftrightarrow\alpha)\rightarrow(p\leftrightarrow\sigma)$ is
	valid. By completeness we then have:
	$\vdash_\GL\sBox(p\leftrightarrow\alpha)\rightarrow(p\leftrightarrow
	\sigma)$
\end{proof}
The following result follows from the fixed-point theorem.
\begin{thm}\label{thm:exi}
	Let $\alpha(p)$ be modalized in $p$, and let $\sigma$ be a fixed-point
	of $\alpha$. Then:
	\[\vdash_\GL\sBox(p\leftrightarrow\sigma)\rightarrow(p\leftrightarrow\alpha)\]
\end{thm}
\begin{proof}
	Suppose that $\mathcal{K}=\la W,R,\phi\ra$ is a finite transitive and irreflexive model
	in which
	$\sBox(p\leftrightarrow\sigma)\rightarrow(p\leftrightarrow\alpha)$ is
	invalid. This means that for some $w\in W$ of least rank that we have:
	$\vDash_w\sBox(p\leftrightarrow\sigma)$ and thus $\vDash_w
	p\leftrightarrow\sigma$ and $\not\vDash_w p\leftrightarrow\alpha$. If $wRv$
	then $\vDash_v\sBox(p\leftrightarrow\sigma)$ and since $x$ of lower
	rank than $w$ we also have $\vDash_v p\leftrightarrow\alpha$. Let
	$\varphi'$ be like $\varphi$ expect that $p\in\phi'(w)$ if and only if
	not $p\in\phi(w)$. Set $\mathcal{K}'=\la W,R,\phi'\ra$, and this model
	is clearly transitive and irreflexive.

	In the rest of this prove we will use theorem: \textbf{State og bevis:
	Continuty heorem}. 


	The formula $\alpha$ is a truth-functional compound (\textbf{Hvad
	betyder dette?} of closed formulas $\Box\beta$ and proportional letters $q$
	such that each $q$ is not $p$. We have that
	$\vDash_w^\mathcal{K}\Box\beta$ if and only if
	$\vDash_v^\mathcal{K}\beta$ for all $v$ such that $wRv$, if and only if
	$\vDash_v^{\mathcal{K}'}$ for all $v$ such that $wRv$ (by continuity),
	if and only if $\vDash_w^{\mathcal{K}'}\Box\beta$. We further have by
	the definition of $\mathcal{K}'$ that $\vDash_w^\mathcal{K}\alpha$ if
	and only iff $\vDash_w^{\mathcal{K}'}$ and $\vDash_w^\mathcal{K}p$ if
	and only if not $\vDash_w^{\mathcal{K}'}p$. Thus we have
	$\vDash_w^{\mathcal{K}'}p\leftrightarrow\alpha$ and b the conti theorem
	we again get $\vDash_v^{\mathcal{K}'}p\leftrightarrow\alpha$ for all
	$v$ such that $wRv$. But this means that we get
	$\vDash_w^{\mathcal{K}'}\sBox(p\leftrightarrow\alpha)$.
	
	Since $\sigma$ does not contain $p$ we get by the conti them that
	$\vDash_w^\mathcal{K}\sigma$ if and only if
	$\vDash_w^{\mathcal{K}'}\sigma$. Since we have that
	$\vDash_w^\mathcal{K} p$ if and only if not $\vDash_w^{\mathcal{K}'}p$
	we get $\not\vDash_w^{\mathcal{K}'}p\leftrightarrow \sigma$ and thus
	$\sBox(p\leftrightarrow\sigma)\rightarrow(p\leftrightarrow\sigma)$ is
	invalid.

	By soundness and completeness of \GL\ we thus get that if
	$\vdash_\GL\sBox(p\leftrightarrow\alpha)\rightarrow(p\leftrightarrow\sigma)$
	then
	$\vdash_\GL\sBox(p\leftrightarrow\sigma)\rightarrow(p\leftrightarrow\alpha)$.
\end{proof}
From this we can prove the following corollary that show that the name
fixed-point is appropriate:
\begin{cor}
	Let $\alpha(p)$ be modalized in $p$ and let $\sigma$ be a fixed-point
	of $\alpha$. Then:
	\[\vdash_G\sigma\leftrightarrow\alpha(\sigma)\]
\end{cor}
\begin{proof}
	Since we have assumed uniform substitution and by theorem \ref{thm:exi}
	the result of substituting $\sigma$ for $p$ in
	$\sBox(p\leftrightarrow\sigma)\rightarrow(p\leftrightarrow\alpha)$ is a
	theorem of \GL. This means that
	$\vdash_\GL\sBox(\sigma\leftrightarrow\sigma)\rightarrow(\sigma\leftrightarrow\alpha(\sigma))$.
	$\sBox(\sigma\leftrightarrow\sigma)$ is obviously a theorem of \GL\ so
	we get:
	\[\vdash_\GL\sigma\leftrightarrow\alpha(\sigma)\]
\end{proof}
With this theorem a lot of different fixed points can be calculated. The
formula on the left is the formula $\alpha(p)$ and the formula on the right is
the formula $\sigma$.
\begin{align*}
	1. &\ \neg\Box p & \neg\Box\bot\\
	2. &\ \Box p & \top\\
	3. &\ \Box\neg p & \Box\bot\\
	4. &\ \neg\Box\neg p & \neg\Box\bot\\
	5. &\ \neg\Box\Box\neg p & \neg\Box\Box\bot\\
	6. &\ \Box p \rightarrow\Box\neg p & \Box\Box\bot\rightarrow\Box\bot\\
	7. &\ \Box (\neg
	p\rightarrow\Box\bot)\rightarrow\Box(p\rightarrow\Box\bot) &
	\Box\Box\Box\bot\rightarrow\Box\Box\bot\\
	8. &\ \Box p\rightarrow q & \Box q\rightarrow q\\
	9. &\ \Box(p\rightarrow q) & \Box q\\
	10. &\ \Box p\wedge q & \Box q\wedge q\\
	11. &\ \Box(p\wedge q) & \Box q\wedge q\\
	12. &\ q\vee \Box p & \top\\	
	13. &\ \neg\Box(q\rightarrow p) & \Diamond q\\
	14. &\ \Box(p\rightarrow q)\rightarrow\Box\neg p &
	\Box(\Box\bot\rightarrow q)\rightarrow\Box\bot\\
	15. &\ q\wedge(\Box(p\rightarrow q)\rightarrow\Box\neg p) &
	q\wedge\Box\neg q\\
	16. &\ \Diamond p\rightarrow(q\wedge\neg\Box(p\rightarrow q)) &
	\Diamond\top \rightarrow (q\wedge\neg\Box(\Box\bot\rightarrow q))\\
	17. &\ \Box(\Box(p\wedge q)\wedge\Box(p\wedge r)) & \Box(\Box q \wedge
	\Box r)
\end{align*}
\begingroup\vspace*{-\baselineskip}
\captionof{figure}{A Table of Fixed Points}
\label{Fig}
\vspace*{\baselineskip}\endgroup

We will after the proof of Solovay's Completeness Theorems come back to this
table, and make some conclutions about the fixed points.
\end{document}
