\documentclass[../main.tex]{subfiles}

In this section it will be shown how much induction there is in \PRA and other
arithmetical theories. We will look at different axioatizable subtheories of
first order arithmetic called fragments. These fragments can be categories in
the following three categories:

\textbf{Strong fragments:} The fragments that can prove the arithmetized cut elimination theorem.

\textbf{Weak fragments:}
Those fragments that can not prove the arithmetized cut-elimination theorem.

\textbf{Very weak fragments:} The fragments which do not contain any induction
axioms

We will start of by defining a very weak fragment called Robinson's theory.
This fragments was first considered in [Ref]. We
will denote this theory with $Q$
This theory have the following axioms:
\begin{align*}
	&\forall x(\neg Sx\not =0)\\
	&\forall x\forall y (Sx=Sy\rightarrow x=y)\\
	&\forall x(x\not =0\rightarrow\exists y(Sy=x)\\
	&\forall x(x+0=x)\\
	&\forall x\forall y(x+Sy=S(x+y)\\
	&\forall x(x\cdot 0=0)\\
	&\forall (x\cdot Sy=x\cdot y+x)
\end{align*}
This theory dose not have the inequality symbol. We will extend $Q$ with the
following axiom:
$$x\leq y\leftrightarrow \exists z(x+z=y)$$
This extension of $Q$ is denoted by $Q_\leq$
\begin{defi}
	Giving a class $\Gamma$ of either $\Sigma_n$ or $\Pi_n$ formulas we define
	$\Gamma$-\textit{induction} ($\Gamma$-\textit{ind}) to be the following
	schema:
	$$\varphi(0)\wedge\forall v(\varphi (x)\rightarrow\varphi(Sx))\rightarrow \forall
	v\varphi (v)$$
	for $\varphi\in\Gamma$. Further we define $\Gamma$-\textit{Least Number
	Principle} ($\Gamma$-\textit{MIN}) to be the following schema:
	$$\exists x\varphi(x)\rightarrow\exists(\varphi(x)\wedge\neg\exists
	y(y<x\wedge\varphi(y)))$$
	For $\varphi\in\Gamma$. Lastly we will define the replacement axioms
	for $\Gamma$, ($\Gamma$-REPL) as the following formulas:
	$$(\forall x\leq t)\exists y\varphi(x,y)\rightarrow\exists z(\forall
	x\leq t)(\exists y\leq z)\varphi(x,y)$$?
\end{defi}

From the above axioms we can create a hierarchy of different strong fragments
of arithmetics. We will define the following theories:

\begin{defi} We define the following fragments:

	\begin{enumerate} 
		\item The theory $I\Sigma_n$ is the theory that is axiomatized by the axioms
			of $Q_\leq$ and the $\Sigma_n$-IND axioms. 
		\item The theory $I\Delta_0$ is
			$Q_\leq$ plus the $\Delta_0$-IND axioms. 
		\item The theory $L\Sigma_n$ is defined by the theory
			$I\Delta_0$ plus the $\Sigma_n$-MIN axioms
		\item The theory $B\Sigma_n$ is defined by the theory
			$I\Delta_0$ plus the $\Sigma_n$-REPL axioms.
		\item 
	Lastly the theory Peano arithmetics is defined as the theory $Q$ plus
	induction for all first order formulas.
	\end{enumerate}
\end{defi}
From the definition it is clear that the theory $I\Delta_0$ plays a crucial
role. It can be shown that in $Q$ a lot of the basic facts about arithmetic can
be shown. These facts will not be shown here, but a list of them can be found
in [Buss].

\begin{defi}
	A predicate symbol
\end{defi}
\begin{defi}
	A function symbol.
\end{defi}
\section{\PRA}
\begin{prop}
	\PRA\ is the thoery?
\end{prop}
We will further define $\PRA^-$ as being the sub-theory of \PRA\ where we
restrict us self to only having induction for p.r formulas. Then we get the
following result:
\begin{prop}
	Over $\PRA^-$ the following schemata are equivalent:
	\begin{enumerate}
		\item $\Sigma_n$-\textit{Ind}
		\item $\Pi_n$-\textit{Ind}
		\item $\Sigma_n$-\textit{LNP}
		\item $\Pi_n$-\textit{LNP}
	\end{enumerate}
\end{prop}
\begin{proof}
\end{proof}
