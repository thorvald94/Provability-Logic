\documentclass[../main.tex]{subfiles}

In this section we will generalize the class of $\sigma_1$ sets to the classes
of $\sigma_n$, $\pi_n$ and $\delta_n$ sets. These definitions will make it
possible for us to look at different fragments of arithmetic, since we can
distinguish these fragments by which class of sets they allow induction on. We
will start off by defining a "base" fragment called \textbf{EA} and then
augment it with stronger inductions axioms. 

We will further look at how much induction we have in \PRA; i.e which class of
formulae from the arithmetical hierarchy we can do induction on in \PRA. Lastly
we will prove a generalization of Löb's third derivability condition for \PRA.

This section should be seen as overview of the subject, and a lot of statements
will be left unproven; since these will not play a major role in whats is to
come.

\section{The arithmetical hierarchy}


In this section we will introduce the so called arithmetical hierarchy. Parts
of it has already be defined; i.e the sets $\Sigma_1$. In the arithmetical
hierarchy, we can classify sets with respect to their quantifier
complexity in their syntactical definition.

In the next chapter about the fragments of arithmetics, we will use the
arithmetical hierarchy as a tool to detemine the amount of induction we will
have in a given fragment.

The goal of this section is to introduce the arithmetical hierarchy and prove
some results about it. The most of these results might not be used going
forward.

\begin{defi}
	We define the sets $\Sigma_n$ and $\Pi_n$ in the following way:
	\begin{enumerate}
		\item A set $A$ is in $\Sigma_0$ ($\Pi_0$) if and only if $A$
			is recursive.
		\item For $n\geq 1$ the set $A$ is in $\Sigma_n$ if there is a
			recursive relation $R(x,\y)$ such that:
			$$x\in A\ \text{iff}\ \exists y_1\forall y_2\exists
			y_3\cdots Qy_nR(x,\y)$$
			Here $Q$ is $\exists$ if $n$ is odd and $Q$ is
			$\forall$ if $n$ is even. We define $A$ being in
			$\Pi_n$ likewise. $A$ is in $\Pi_n$ if:
			$$x\in A\ \text{iff}\ \forall y_1\exists y_2\forall
			y_3\cdots Qy_nR(x,\y)$$
		\item $A$ is in $\Delta_n$ if $A\in\Sigma_n\cap\Pi_n$
	\end{enumerate}
	We further say that a formula $\varphi$ is $\Sigma_n$ ($\Pi_n$) if it
	is $\Sigma_n$ ($\Pi_n$) as a relation of the variables that are free in
	it. 
\end{defi}
In the rest of this project we will mostly look at formulas that are either
$\Sigma_n$ or $\Pi_n$ and not sets, that have this property.

We can show a few properties of these sets.
\begin{prop}
	\begin{enumerate}
		\item $A\in\Sigma_n\Leftrightarrow\ol A\in\Pi_n$
		\item $A\in\Sigma_n (\Pi_n)\Rightarrow(\forall
			m>n)(A\in\Sigma_m\cap\Pi_m)$
		\item $A,B\in\Sigma_n(\Pi_n)\Rightarrow A\cup B,A\cap
			B\in\Sigma_n(\Pi_n)$
		\item $(R\in \Sigma_n\wedge n>0\wedge A=\{x:\exists y
			R(x,y)\})\Rightarrow A\in\Sigma_n$
		\item $(B\leq_m A\wedge A\in\Sigma_n)\Rightarrow B\in \Sigma_n$
		\item If $R\in\Sigma_n(\Pi_n)$ and $A$ and $B$ are defined by:
			$$\la x,y\ra \in A \Leftrightarrow \forall z<y
			R(x,y,z)$$
			and
			$$\la x,y\ra \in B \Leftrightarrow \exists z<y
			R(x,y,z)$$
			Then we have $A,B\in\Sigma_n(\Pi_n)$
	\end{enumerate}
\end{prop}
\begin{proof}
	\begin{enumerate}
		\item If we have that:
			\[
				A=\{x:\exists y_1\forall y_2\cdots
				R(x,y_1,\ldots)\}
			\]
			Then we have:
			\[
				\ol A=\{x:\forall y_1\exists y_2\cdots
				\neg R(x,y_1,\ldots)\}
			\]
			Which is clearly $\Pi_n$.
		\item If for example $A=\{x:\exists y_1\forall y_2
			R(x,y_1,y_2)$\}, then we can make the following
			reformulation of $A$:
			\[
				A=\{x:\exists y_1\forall y_2\exists
				y_3(Rx,y_1,y_2)\wedge y_3=y_3)\}
			\]
			This kind of reformulation can be done for any set in
			$\Sigma_n$ ($\Pi_n$)
		\item Let the following to sets be defined:
			\[
				A=\{x:\exists y_1\forall y_2\cdots
				R(x,y_1,y_2,\ldots)\}\\
			\]
			\[
				B=\{x:\exists  z_1\forall z_2\cdots
				S(x,z_1,z_2,\ldots)\}\\
			\]
			Then we have:
			\begin{align*}
				x\in A\cup B&\Leftrightarrow \exists y_1\forall
				y_2\cdots R(x,y_1,y_2,\ldots)\vee \exists
				z_1\forall y_2\cdots S(x,z_1,z_2\ldots)\\
					    &\Leftrightarrow \exists y_1\exists z_1\forall
				y_2\forall z_2\cdots (R(x,y_1,y_2,\ldots)\vee
				S(x,z_1,z_2,\ldots))\\
					    &\Leftrightarrow \exists u_1\forall
					    u_2\cdots
					    (R,(u_1)_0,(u_2)_0,\ldots)\vee
					    S(x,(u_1)_1,(u_2)_1,\ldots))
			\end{align*}
			Which is clearly $\Sigma_n$. The same argument can be
			made for
			$A\cap B$ and for $\Pi_n$ sets.
		\item This follows by quantifier contraction, in the same way
			as (3)
		\item Let 
			\[ 
				A=\{x:\exists y_1\forall y_2\cdots
				R(x,y_1,y_2,\ldots)\}
			\]
			And let $B\leq_m A$ via the function $f$. Then we have:
			\[
				B=\{ x:\exists y_1\forall y_2\cdots
				R(f(x),y_1,y_2,\ldots)\}
			\]
		\item We will prove this by induction on $n$.

			\textbf{Base case:} Let $n=0$. Then $A$ and $B$ are
			clearly recursive. 

			\textbf{Induction step:} Now assume that $n>0$ and
			suppose that $R\in\Sigma_n$. Our induction hypothesis
			says that (6) is true for all $m<n$. Then by (4) we
			have that $B\in\Sigma_n$. Further we have
			$S\in\Pi_{n+1}$ such that the following holds:
			\begin{align*}
				\la x,y\ra \in A&\Leftrightarrow(\forall z<y)
				R(x,y,z)\\
						&\Leftrightarrow(\forall z<y)
						\exists u S(x,y,z,u)\\
						&\Leftrightarrow \exists
						\sigma(\forall z<y)
						S(x,y,z,\sigma(z))
			\end{align*}
			WE have that $\sigma$ range is in $\omega^{<\omega}$.
			By the indction hypothesis we have that $(\forall
			z<y)S\Pi_{n+1}$ but by the above deduction we must then
			have that $A\in\Sigma_n$.
	\end{enumerate}
\end{proof}

\section{Fragments of arithmetic}

\textbf{Strong fragments:} The fragments that can prove the arithmetized cut elimination theorem.

\textbf{Weak fragments:}
Those fragments that can not prove the arithmetized cut-elimination theorem.

\textbf{Very weak fragments:} The fragments which do not contain any induction
axioms

We will start of by defining a very weak fragment called Robinson's theory.
This fragments was first considered in [Ref]. We
will denote this theory with $EA$ for elementary aritmetics.
This theory have the following axioms:
\begin{align*}
	&\forall x(\neg Sx\not =0)\\
	&\forall x\forall y (Sx=Sy\rightarrow x=y)\\
	&\forall x(x\not =0\rightarrow\exists y(Sy=x)\\
	&\forall x(x+0=x)\\
	&\forall x\forall y(x+Sy=S(x+y)\\
	&\forall x(x\cdot 0=0)\\
	&\forall (x\cdot Sy=x\cdot y+x)
\end{align*}
This theory dose not have the inequality symbol. We will extend $Q$ with the
following axiom:
$$x\leq y\leftrightarrow \exists z(x+z=y)$$
This extension of $Q$ is denoted by $Q_\leq$
\begin{defi}
	Giving a class $\Gamma$ of either $\Sigma_n$ or $\Pi_n$ formulas we define
	$\Gamma$-\textit{induction} ($\Gamma$-\textit{ind}) to be the following
	schema:
	$$\varphi(0)\wedge\forall v(\varphi (x)\rightarrow\varphi(Sx))\rightarrow \forall
	v\varphi (v)$$
	for $\varphi\in\Gamma$. Further we define $\Gamma$-\textit{Least Number
	Principle} ($\Gamma$-\textit{MIN}) to be the following schema:
	$$\exists x\varphi(x)\rightarrow\exists(\varphi(x)\wedge\neg\exists
	y(y<x\wedge\varphi(y)))$$
	For $\varphi\in\Gamma$. Lastly we will define the replacement axioms
	for $\Gamma$, ($\Gamma$-REPL) as the following formulas:
	$$(\forall x\leq t)\exists y\varphi(x,y)\rightarrow\exists z(\forall
	x\leq t)(\exists y\leq z)\varphi(x,y)$$?
\end{defi}

From the above axioms we can create a hierarchy of different strong fragments
of arithmetics. We will define the following theories:

\begin{defi} We define the following fragments:

	\begin{enumerate} 
		\item The theory $I\Sigma_n$ [$I\Pi_n$]is the theory that is axiomatized by the axioms
			of $Q_\leq$ and the $\Sigma_n$ [$\Pi_n$]-IND axioms. 
		\item The theory $I\Delta_0$ is
			$Q_\leq$ plus the $\Delta_0$-IND axioms. 
		\item The theory $L\Sigma_n$ is defined by the theory
			$I\Delta_0$ plus the $\Sigma_n$-MIN axioms
		\item The theory $B\Sigma_n$ is defined by the theory
			$I\Delta_0$ plus the $\Sigma_n$-REPL axioms.
		\item The theory $I\Sigma_n^R$ is defined as the closure of
			$Q_\leq$ under the $\Sigma_n$ induction rule:
			\[\frac{F(0),\ \forall
					x(F(x)\rightarrow\varphi(x+1))}{\forall
			x F(x)}\]
		\item The theory $I\Sigma_n^-$ [$\Pi_n^-$] is the theory axiomatizedbe the
			axioms of $Q_\leq$ and the schema of induction of
			$\Sigma_n$ [$\Pi_n^-$] functions $f(x)$, which only has $x$ as a
			free variable.
	Lastly the theory Peano arithmetics is defined as the theory $Q$ plus
	induction for all first order formulas.
	\end{enumerate}
\end{defi}
From the definition it is clear that the theory $I\Delta_0$ plays a crucial
role. It can be shown that in $Q$ a lot of the basic facts about arithmetic can
be shown. These facts will not be shown here, but a list of them can be found
in [Buss].




\section{\PRA s placement in the arithmetical hierarchy}
Having introduced the arithmetical hierarchy in the previous section, it is not
obvious where to place \PRA\ in this hierarchy. But the following theorem can
be shown.
\begin{prop}
	\PRA\ is the equivalent to the theory $I\Sigma_1^R$
\end{prop}

The proof of this is non-trivial and will go way beyond the scope of this
project. The statement will not be used going forward, and its just included
to give some intuition about the placement of \PRA\ in the arithmetical hierarchy.
The proof can be found in \parencite{ArteBe}. This makes it possible to place \PRA\
in the arithmetical hierarchy. We can make a visualization of how all the
different fragments relate to each other:
\newpage
\begin{figure}[h]
	\begin{center}
\begin{tikzpicture}[baseline={(0,0)}]
	\node[label={\textbf{Q}}] () at (0,-1) {};
	\node[] (EA) at (0,0) {};
	\node[label={$I\Sigma_1^R=$ \PRA}] at (3,-1) {};
	\node[] (PRA) at (2,0) {};
	\node[label={$I\Pi_1^-$}] (IPi) at (0,2) {};
	\node[label={$I\Sigma_1^-$}] (Isig) at (1.5,2) {};
	\node[] (Isig) at (2,2) {};
	\node[label={$I\Sigma_1$}] at (4,1) {};
	\node[] (ISig) at (4,2) {};
	\node[label={$I\Pi_2^-$}] at (1.5,4) {};
	\node[] (IPi2) at (2,4) {};
	\node[label={$I\Sigma_1+I\Pi_2^-$}] at (3.5,4) {};
	\node[] (ISigPi) at (4,4) {};
	\node[label={$I\Sigma_2^R$}]  at (6,1) {};
	\node[] (ISig2R) at (6,2) {};
	\node[label={$I\Sigma_2^-$}] at (5.5,4) {};
	\node[] (ISig2-) at (6,4) {};
	\node[label={$I\Pi_3^-$}] at (5.5,6) {};
	\node[] (IPi3) at (6,6) {};
	\node[label={$I\Sigma_2+I\Pi_3^-$}] at (7.5,6) {};
	\node[] (ISigPi3) at (8,6) {};
	\node[label={$I\Sigma_2$}] at (8,3) {};
	\node[] (ISig2) at (8,4) {};
	\node[label={$I\Sigma_3^R$}] at (10,3) {};
	\node[] (ISig3R) at (10,4) {};
	\node[label={$I\Sigma_3^-$}] at (9.5,6) {};
	\node[] (ISig3) at (10,6) {};
	\node[] (Tom1) at (10,7) {};
	\node[] (Tom2) at (11,6) {};

	 \draw (EA) -- (PRA);
	 \draw (EA) -- (IPi);
	 \draw (IPi) -- (ISig2R);
	 \draw (PRA) -- (IPi2);
	 \draw (IPi2) -- (ISig3R);
	 \draw (ISig2R) -- (IPi3);
	 \draw (IPi3) -- (Tom2);
	 \draw (ISig) -- (ISigPi);
	 \draw (ISig2) -- (ISigPi3);
	 \draw (ISig3R) -- (Tom1);

    \node at (EA)[circle,fill,inner sep=1.5pt]{};
    \node at (PRA)[circle,fill,inner sep=1.5pt]{};
    \node at (IPi)[circle,fill,inner sep=1.5pt]{};
    \node at (ISig)[circle,fill,inner sep=1.5pt]{};
    \node at (Isig)[circle,fill,inner sep=1.5pt]{};
    \node at (IPi2)[circle,fill,inner sep=1.5pt]{};
    \node at (ISigPi)[circle,fill,inner sep=1.5pt]{};
    \node at (ISig2R)[circle,fill,inner sep=1.5pt]{};
    \node at (ISig2)[circle,fill,inner sep=1.5pt]{};
    \node at (IPi3)[circle,fill,inner sep=1.5pt]{};
    \node at (ISigPi3)[circle,fill,inner sep=1.5pt]{};
    \node at (IPi3)[circle,fill,inner sep=1.5pt]{};
    \node at (ISig2-)[circle,fill,inner sep=1.5pt]{};
    \node at (ISig3R)[circle,fill,inner sep=1.5pt]{};
    \node at (ISig3)[circle,fill,inner sep=1.5pt]{};
\end{tikzpicture}
\end{center}
\caption{How the fragments relate to each other.}
\end{figure}

We will now turn to another version of the Selection Theorem. 
\begin{thm}\label{thm:sel}
	Let $\tau v_0\ldots v_{n-1}v_n$ be $\Sigma_1$. There is then a
	$\Sigma_1$-formula $\text{Sel}(\tau)$ with exactly the same free
	variables such that:
	\begin{enumerate}
		\item $\text{Sel}(\tau)v_0\ldots v_n\rightarrow\tau v_0\ldots
			v_n$
		\item $\text{Sel}(\tau)v_0\ldots
			v_{n-1}v_n\wedge\text{Sel}(\tau)v_0\ldots
			v_{n-1}v^*\rightarrow v_n=v^*$
		\item $\exists v_n\tau v_0\ldots v_{n-1}v_n\rightarrow \exists
			v_n\text{Sel}(\tau v_0\ldots v_{n-1}v_n$
	\end{enumerate}
	Further this can be proven in \PRA.
\end{thm}
\begin{proof}
	Since $\tau$ is $\Sigma_1$ it is recursively enumerable. So we have
	that:
	\[\tau v_0\ldots v_n:\ \exists v(fvv_0\ldots v_n=\ol 0)\]
\end{proof}

\subsection{Induction In \PRA}
We will further define $\PRA^-$ as being the sub-theory of \PRA\ where we
restrict us self to only having induction for p.r formulae and not $\Sigma_1$
formulae. Then we get the
following result:
\begin{prop}
	Over $\PRA^-$ the following schemata are equivalent:
	\begin{enumerate}
		\item $\Sigma_n$-\textit{Ind}
		\item $\Pi_n$-\textit{Ind}
		\item $\Sigma_n$-\textit{LNP}
		\item $\Pi_n$-\textit{LNP}
	\end{enumerate}
\end{prop}
It shall be noted that we will only use the case of $n=1$ going forward.

\begin{proof}
	The implications $(1)\Rightarrow(4)$ and $(2)\Rightarrow(3)$ was proven
	in proof of theorem \ref{thm:sel}. The converse of these are done in
	the same way.

	We will now show that $(1)\Rightarrow (2)$. The $(2)\Rightarrow (1)$ is
	similar. We will prove the case where $n=1$, and the cases where $n>1$
	is identical, \textit{modulo} the closure of $\Sigma_n$ under bounded
	quantification.

	So assume $\Sigma_1$-IND and suppose for $Fv\in\Sigma_1$ that the
	following instance of $\Pi_1$-IND fails:
	\[(\neg F(\0)\wedge\forall v(\neg F(v)\rightarrow\neg
	F(Sv)))\rightarrow\forall v\neg F(v)\]
	This means that we have $\neg F(\ol 0)$, $\forall v(\neg
	F(v)\rightarrow\neg F(Sv)$ and $\exists v F(v)$. We chose $v_0$ such
	that $F(v_0)$. We will use $\Sigma_1$ induction on the variable $v$ in
	$F(v_0\dot - v)$ to prove that $F(\ol 0)$ and get a contradiction. 

	\textbf{Base step:} It
	is clear that $F(v_0\dot -\ol 0)$, since this is just $F(v_0)$.

	\textbf{Induction step:} Assume that $F(v_0\dot - v)$ is true. We have
	that $S(v_0\dot - S(v))=v_0\dot -v$ unless we already have that
	$v=v_0$, in which case we have that $v_0\dot - S(v)=v_0-v$. So we can
	conclude that $\forall v F(v_0\dot -v)$ and thus we have $F(\ol 0)$ and
	get our contradiction.

	The converse of i,e $(2)\Rightarrow(1)$ is done in a similar  way.
\end{proof}

This means that \PRA\ can do induction on $\Pi_1$ formulas, since it has
induction for $\Sigma_1$ induction. It also tells us that \PRA\ has the least
number principle for $\Sigma_1$ and $\Pi_1$ formulas.
For \PRA\ the following theorem can be proven, but the proof will not be given
here. But we will use theorem.

\begin{thm}
	$\PRA\vdash\text{Bool}(\Sigma_1)-Ind$ where $\text{Bool}(\Sigma_1)$ is
	the class of combinations of $\Sigma_1$ formulae.
\end{thm}


\section{Further extensions of the arithmetical hierarchy}

If we have that $I\Gamma$  is a fragment of Peano Arithmetics, then we can look
at the fragment
$I\Gamma+\text{EXP}$ which proves induction over $\Gamma$-formulas and proves
that for all $x$ the its power $2^x$ exists. We can also look at
$I\Gamma+\Omega_1$ which is a weaker theory than $I\Gamma+\text{EXP}$.  Here
$\Omega_1$ is a axiom that asserts that for all $x$ its power
$x^{\text{log}(x)}$ exists. It is also clear that since $\PRA=I\Sigma_1^R$ that
the two fragments $I\sigma_1+\text{EXO}$ and $I\sigma_1+\Omega_1$  are weaker than \PRA.  We will return to these two
fragments later on in chapter \ref{chap:Complete}.

It should here be stated that if look at theories $T$ such that
$Q_\leq\subseteq T$, then it will be possible to encode the
syntax of that theory in a similar way to what we have done with \PRA; i.e in a
way such that the proof predicate of that theory $\text{PR}_T(x)$ is a
$\Sigma_1$ formula. We will call such a theory for a \textit{RE}-theory. We can then derive all the incompleteness result that we
have obtained for \PRA\ in that theory. Sometimes a little \textit{tweaking} of
the encode should be made, but we can get the following result that will not be
proven here:
\begin{thm}
	Let $T$ be a \textit{RE}-theory such that $\PRA\leq\subseteq\textbf{EA}$. Then the
	following holds:
	\begin{enumerate}
		\item For any sentences $F$ and $G$ we get Löbs derivability
		conditions
	\item Gödel's incompleteness theorems for any sentence such that
		$\PRA\vdash F\leftrightarrow\ \neg\text{Pr}_T(\Godelnum{F})$
	\item Löb's theorem for any sentence $F$
	\end{enumerate}
\end{thm}
This shows that the provability predicate of such fragments $T$ behaves in the same
way as the one from \PRA.


\section{Sigma Completeness}
We can make a generalization of $D3$. We use
this generalization in some of our proofs later on. We will first need to prove
the following lemma:
\begin{lem}
	Let $F v_0\ldots v_{n-1}$ be a $\Sigma_1$. Then there is a
	recursively enumerable formula such that 
	\[\PRA\vdash F v_0\ldots v_{-n1}\leftrightarrow\exists v(fvv_0\ldots
	v_{n-1}=\ol 0)\]
\end{lem}
\begin{proof}
\end{proof}
\begin{thm}[Demonstrable $\Sigma_1$ completeness]
	\label{thm:DemoSiga}
	Let $Fv_0\ldots v_{n-1}$ be a $\Sigma_1$ formula with free variables.
	Then:
	$$\PRA\vdash Fv_0\ldots
	v_{n-1}\rightarrow\text{Pr}(\Godelnum{f\dot v_0\ldots\dot v_{n-1}})$$
\end{thm}
This theorem states that if a $\Sigma_1$ formula $F$ is true, then it can be
proven to be true in \PRA\; i.e \PRA\ is complete with respect to $\Sigma_1$
formulae.
\begin{proof}
	By [ja, fra hvad?] we have that there is a $\Sigma_1$ formula $\exists
	v(gvv_0\ldots v_{n-1}=\ol 0$ such that we have:
	\begin{equation}
		\label{eq:Com1}
		\PRA\vdash fv_0\ldots v_{n-1}\leftrightarrow\exists v(gvv_0\ldots
		v_{n-1}=\ol 0)
	\end{equation}
	and by $D1$ we have:
	\begin{equation}
		\label{eq:Com2}
		\PRA\vdash\text{Pr}(\Godelnum{fv_0\ldots
		v_{n-1}\leftrightarrow\exists v(gvv_0\ldots v_{n-1}=\ol 0)})
	\end{equation}
	We can now make the following deductions:
	\begin{align*}
		\PRA\vdash fv_0\ldots v_{n-1}&\rightarrow\exists v (gvv_0\ldots
		v_{n-1}=\ol 0) 
					     &&\text{By \ref{eq:Com1}}\\
					     &\rightarrow\exists v\text{Pr}(
					     \Godelnum{hvv_0\ldots v_{n-1}=\ol
					     0}
					     && \ref{lem:Prov}\\
					     &\rightarrow\text{Pr}(\Godelnum{
						     \exists v(hvv_0\ldots
					     v_{n-1}=\ol 0)})
					     &&\text{By D1 and D2}\\
					     &\rightarrow\text{Pr}(\Godelnum{
					     fv_0\ldots v_{n-1}})
					     &&\text{By \ref{eq:Com2}}
	\end{align*}
\end{proof}
This theorem gives a difference between $\Sigma_1$ and $\Pi_1$ sentences in
\PRA. If $F$ is $\Pi_1$ we can have that both $F+\text{Con}(\PRA+F)$ and
$F+\text{Con}(\PRA+\neg F)$ are consistent, but if $F$ is $\Sigma_1$ we get
that $F+\text{Con}(\PRA+\neg F)$ is inconsistent.
