\documentclass[../main.tex]{subfiles}

\begin{document}

\textbf{Skriv kort hvad projektet omhandler og opris de vigtisge resultater}

\section{Notation}

We denote the non-negative integers with $\omega$; i.e
$\omega=\{0,1,2,3,\ldots\}$, and $\omega\times\omega=\omega^2$ and so on.
Subsets of $\omega$ will be denoted by $A,B,C\ldots$ and arbitrary sets will be
denoted by
$\Gamma,\Phi,\Theta\ldots$. Lower case Latin letter $a,b,c$ and $x,y,z$ will
denote integers, and $\vec{x}$ will be shorthand for $x_1,\ldots,x_n$. Further recursive functions will be denoted by
$\varphi,\psi,\vartheta,\ldots$, and other functions that are not recursive will be denoted by
$f,g,h,\ldots$. $\varphi(x)\downarrow$ will denote that $\varphi(x)$ is defined and
$\varphi(x)\downarrow=y$ denotes that $\varphi(x)$ is defined and has value $y$.
$\varphi(x)\uparrow$
denotes that $\varphi(x)$ is undefined. $\text{dom}\ \varphi$ and $\text{im}
\varphi$ denotes the
domain and image of $\varphi(x)$.
A few special functions have their own symbols: $S$ for the
successor function, $C$ for the constant function and $P$ for the projection.
Relations will be denoted by $R$ and $<,>,\leq,\geq$ will be used in the usual
sense on integers.

We will use the standard symbols of propositional  logic
$\wedge,\vee,\rightarrow,\neg$. Further we will sometimes use $\exists$ and
$\forall$ in the metalanguage of modal logic, and use them in the language of
first-order arithmetics. The differences of these two uses should be clear
from the context. We will argument the propositional logic with the modal unary
connective $\Box$ and its dual $\Diamond:=\neg\Box\neg$. We will denote modal
formulas by the greek letters $\alpha,\beta,\gamma,\sigma$. The symbol
$\mathcal{H}$ will denote a
Hintikka frame  which is a tuple $\la W,R\ra$ where $W$ is a set of "nodes" and
$R$ is a relation on $W$and
$\mathcal{K}$ will denote a Kripke model, which is a Hintikka frame with a
valuation $\phi$ on.

\section{Historical Introduction}
Humans have been interested in numbers and calculation with these since before
the beginning of history; i.e arithmetics. 

\subsection{Modal Logic}
The Greek philosopher Aristotle did try to develop some kind of modal logic
about necessary and possibility in both his logical works and in his
work about the \textit{First Philosophy}.{textbf{Ref til de rigtige steder} }

With our modern eyes his thoughts about nesscaration and possibility seems a
bit confusing. But he gets something right:j

\[\text{Eksempel på noget han forstår}\]

The next big step for modal logic came in the enlightenment when the
rationalist philosopher Gottfried W. Leibniz came up with the idea of possible
worlds.  Leibniz thought that we live in the best possible world, since it is
this world God has chosen to create. With this idea of a metaphysical possible
worlds, it was possible to make a clear definition of when a proposition is
necessary true; and a proposition is necessary true, if it is true in all
possible worlds.

\textbf{Hvad sker der mellem Leibniz og Kripke; især udviklingen indenfor
syntax}

In the 1950s there had been a lot of research into the syntax of modal logic.
But there was still not a clear definition of the semantics of modal logic,
even though philosophers and logicians still had the intuitive definition of
necessary true from Leibniz. The answer came in the late 1950s, when the
philosophers and logicians Jaako Hintikka
and Saul Kripke developed the concept of a Kripke model.\footnote{They were not
	alone in this development. But for simplicity sake \textbf{ja}}
Hintikka came up with the idea of a frame; i.e a pair $\la W,R\ra$ where $W$ is
a set of worlds and $R$ is an arbitrary relation on these. With this
construction it was possible to explain for a given world, what other worlds
was accessible to it; i.e possible worlds for it. Kripke came up with the same
idea, but added a valuation function to the frame, which to each formula of the
language ascribed a set of worlds in which it was true.  This construction made
it possible for Kripke to prove some completeness results about modal logics.

When the concept of the Kripke model became wider known, it started a
\textit{Golden} age of research in modal logic. A lot of completeness results
about different modal logic was proven in the 1960s and early 1970s; including
the weak completeness theorem for the logic that would later be known as
provability logic; this was done by Krister Segerberg without him having
knowledge of what the interpretation of this logic could be.

A deeper look into the development of modal logic can be found in
\cite{Goldblatt2003}.
\subsection{Arithmetics and Recursion Theory}

\textbf{Skriv noget om axiomer systemer og historie}

In the second half of the 1800s there was a development in the axiomatization
of aritmethics. Dedeking and Peano developed axioms system for arimethics,
these was later evovled into was today is known as the Peano axioms for
arithmetics. 
\textbf{Hilberts program}
\textbf{Primitive Recursive arithmetics (Skolem)}
\textbf{Russels Principia}
In 1931 Kurt Gödel's paper: \textit{Title} showed that no mathematical system that
is strong enough to do simple arithemtic can't shown its own consistency. The
original proof was build on the system of Russel and Whitehead and had a
assumption of $\omega$-consistency. This
was a blow to the Hilbert program, since the result showed that the goal of it
could not be fulfilled. Later on Rosser improved the result so that it used
normal consistency.

Later on in the 1930s Gödel extended the primitive recursive function to the  recursive
functions. A few different definitions of intuitively computable functions was
set forward by Kleene, Turing and Church and it was shown that all these
different notions of intuitively computable functions gave rise to the same
class of functions.

This lead Church to state the so called \textit{Church-Turing Thesis}, that
states that this class of functions are the computable functions.


Gödels Incompleteness Theorems is one of the major results in recursive theory.

\subsection{Provability Logic}
The first time that someone made a connection between these two different areas
of mathematics was Kurt Gödel in his very short  paper: \{ªsdf\}. Here we saw
the intuitionist's laog, bla bla.

After Löb's derivability conditions was found and after the develpment of the
semantics of modal in the 1960s, it became clear that there was a connection
between the logic that Segerberg had proven to be weakly complete and the proof
predicate from arithmetics.

This lead to research done in different places around the from, from The
Netherlands, Italy and the United States. It was clear early on that if
$\alpha$ was a theorem of provability logic, then for all interpretations of
$\alpha$ in a fragment of arithmetics was also a theorem of this fragment; but
the other way was harder to show. In the same time there was proven a number of
important theorems about provability; one of these, called the Fixed Point
Theorem,  will be proven in chapter
\ref{chap:Fixed}.
A short overview of this early
development can be found in \textbf{Ref:Sambin Boolos}

It was proven in 1976 by Solovay that the modal
logic now known as provability logic, is the logic of the provability predicate
in Peano Arithemics; i.e if for all interpretations
of a formula $\alpha$, if this interpretation was a theorem of Peano
arithmetics,
then the formula $\alpha$ was a theorem of provability logic. This also means that
provability logic can be seen as an axiomataxion of the proof predicate.
Solovays result will be the main result in this project; it will be stated
precisely and proven in Chapter \ref{chap:Complete}.

After Solovay's article was published, the research focus of provability logic
was to generalize Solovay's result to other fragments of arithmetics and if
it did also hold for multi-modal provability logic and quantified provability
logic. The results of these results will be commented on in the last part of
chapter \ref{chap:Complete} and in chapter \ref{chap:Further}
\end{document}
