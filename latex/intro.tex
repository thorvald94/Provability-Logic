\documentclass[../main.tex]{subfiles}

\begin{document}

\section{Notation}

We denote the non-negative integers with $\omega$; i.e
$\omega=\{0,1,2,3,\ldots\}$, and $\omega\times\omega=\omega^2$ and so on.
Subsets of $\omega$ will be denoted by $A,B,C\ldots$ and arbitrary 
$\Gamma,\Phi,\Theta\ldots$. Lower case Latin letter $a,b,c$ and $x,y,z$ will
denote integers. Further recursive functions will be denoted by
$F,G,H,\ldots$, and other functions that are not recursive will be denoted by
$f,g,h,\ldots$. $F(x)\downarrow$ will denote that $F(x)$ is defined and
$F(x)=y$ denotes that $F(x)$ is defined and has value $y$. $F(x)\uparrow$
denotes that $F(x)$ is undefined. $\text{dom}F$ and $\text{im} F$ deontes the
domain and image of $F(x)$.
A few special functions have their own symbols: $S$ for the
successor function, $C$ for the constant function and $P$ for the projection.
Relations will be denoted by $R$ and $<,>,\leq,\geq$ will be used in the usual
sense on integers.

We will use the standard symbols of propositional  logic
$\wedge,\vee,\rightarrow,\neg$. Further we will sometimes use $\exists$ and
$\forall$ in the metalanguage of modal logic, and use them in the language of
first-order arithmetics. The differences of these two uses should be clear
from the context. We will argument the propositional logic with the modal unary
connective $\Box$ and its dual $\Diamond:=\neg\Box\neg$. We will denote modal
formulas by the greek letters $\varphi,\psi,\theta,\ldots$. The symbol
$\mathcal{H}$ will denote a
Hintikka frame  which is a tuple $\la W,R\ra$ where $W$ is a set of "nodes" and
$R$ is a relation on $W$and
$\mathcal{K}$ will denote a Kripke model, which is a Hintikka frame with a
valuation $\phi$ on.

\section{Historical Introduction}
Both the studies of arithmetics and modal logic can be traced back to the
accent Greeks.\footnote{The study of arithmetics can be traced even further
back}

The first time that someone made a connection between these two different areas
of mathematics was Kurt Gödel in his paper: \{ªsdf\} To actually understand how
this connection was made, a bit of the back story is needed. 


A deeper look into the development of modal logic can be found in [reffff].W
\end{document}
