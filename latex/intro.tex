\documentclass[../main.tex]{subfiles}

\begin{document}

This project is about the subject provability logic from mathematical logic,
and the main result will be Solovay's completeness theorems. These theorems
states that the $\Box$-operator in  modal logic \GL\  axiomatizes the proof
predicate $\text{Pr}(\cdot)$ from a wide range of fragments of arithmetics, and
that the modal logic \GLS\ axiomatizes the proof predicate of true arithmetics.
For proving these theorems we will need some results from recursion theory and
general knowledge about fragments of arithmetics and the modal logics \GL\ and
\GLS. We will further prove the fixed point theorem for \GL.
The project has the following structure: 
\begin{enumerate}
	\item[\textbf{Chapter 1:}] This chapter will list some prerequisite results and
		definitions about the modal logic \GL\ and the arithmetical system primitive
		recursive arithmetic. 
	\item[\textbf{Chapter 2:}] In this chapter there will be a short introduction to recursion
		theory and the main goal is to show the recursion theorem.
	\item[\textbf{Chapter 3:}] Here we will introduce the arithmetical hierarchy and
		fragments of arithmetics. We will further comment on the amount
		of induction we have available in primitive recursive
		arithmetics
	\item[\textbf{Chapter 4:}] We will start of by strengthening the completeness
		theorem for \GL, with the finite tree completeness theorem.
		Further we will look at an expansions of \GL\ known as \GLS,
		which we will use later on for Solovay's second completeness
		theorem.
	\item[\textbf{Chapter 5:}] The main goal of this chapter is to state and prove
		the fixed point theorem. We will also use it to calculate a few fixed
		points.
	\item[\textbf{Chapter 6:}] In this chapter we will prove both of Solovay's
		completeness theorems. After the proofs of these we will look
		at applications of them.
	\item[\textbf{Chapter 7:}] Here we will round up the project with some
		perspective to further results in provability logic.
\end{enumerate}

In the rest of this introduction we will introduce a bit of notational
conventions and give a short historical overview of the different logical
fields, that will play a role in the project.

The bibliography of this project consist of a wide range of logical,
mathematical and philosophical texts that in one way or another has a
connection to of is referenced in the project. The main sources for this project
is \parencite{Smor1985}, \parencite{Boolos1993} and \parencite{Soare1987}.

\section{Notation}

We denote the non-negative integers with $\omega$; i.e
$\omega=\{0,1,2,3,\ldots\}$, and $\omega\times\omega=\omega^2$ and so on.
Subsets of $\omega$ will be denoted by $A,B,C\ldots$ and arbitrary sets will be
denoted by
$\Gamma,\Phi,\Theta\ldots$. Lower case Latin letter $a,b,c$ and $x,y,z$ will
denote integers, and $\vec{x}$ will be shorthand for a finite sequences, e.g.$x_1,\ldots,x_n$. Further recursive functions will be denoted by
$\varphi,\psi,\vartheta,\ldots$, the graphs of recursive functions will be
denoted by $\tau,\xi$ and $\zeta$ . Functions that are primitive recursive will be denoted by
$f,g,h,\ldots$. 

We will further introduce a system of arithmetic called \textit{primitive
recursive arithmetic} (\PRA, for short) and the formulas of this
system will be denoted with $F,G$ and $G$.

For a partial function $\varphi(x)\downarrow$ will denote that $\varphi(x)$ is defined and
$\varphi(x)\downarrow=y$ denotes that $\varphi(x)$ is defined and has value $y$.
$\varphi(x)\uparrow$
denotes that $\varphi(x)$ is undefined. $\text{dom}(\varphi)$ and $\text{im}
(\varphi)$ denotes the
domain and image of $\varphi(x)$.
A few special functions have their own symbols: $S$ for the
successor function, $Z$ for the zero function and $P$ for the projection.
The formulae of the language \PRA\ will be denoted by $F,G$ and $H$.

Relations will be denoted by $R$ and $<,>,\leq,\geq$ will be used in the usual
sense on integers.

We will use the standard symbols of propositional  logic
$\wedge,\vee,\rightarrow,\neg$. Further we will sometimes use $\exists$ and
$\forall$ in the metalanguage of modal logic, and use them in the language of
first-order arithmetics. The differences of these two uses should be clear
from the context. We will argument the propositional logic with the modal unary
connective $\Box$ and its dual $\Diamond:=\neg\Box\neg$. We will denote modal
formulas by the greek letters $\alpha,\beta,\gamma,\sigma$. The symbol
$\mathcal{H}$ will denote a
Hintikka frame  which is a tuple $\la W,R\ra$ where $W$ is a set of "nodes" and
$R$ is a relation on $W$and
$\mathcal{K}$ will denote a Kripke model, which is a Hintikka frame with a
valuation $\phi$ on.

\section{Historical Introduction}
Humans have been interested in numbers and calculation with these since before
the beginning of history; i.e arithmetics. 

\subsection{Modal Logic}
The Greek philosopher Aristotle did try to develop some kind of modal logic
about necessary and possibility in both his logical work \textit{De
Interpretatione} and in his
 \textit{Metaphysic}.

With our modern eyes his thoughts about nessaration and possibility seems a
bit confusing. But he gets some implications about these notions right. A good
overview of this part of his logical work can be found in \parencite{Lemmon1977} and
in \parencite{Luka1957}.
In the late antiquity and the middle ages there was done some work building on 
Aristotle earlier work. 

The next big step for modal logic came in the enlightenment when the
rationalist philosopher Gottfried W. Leibniz came up with the idea of possible
worlds.  Leibniz thought that we live in the best possible world, since it is
this world God has chosen to create. With this idea of a metaphysical possible
worlds, it was possible to make a clear definition of when a proposition is
necessary true; and a proposition is necessary true, if it is true in all
possible worlds. But Leibniz did not any further works on modal logic as a
whole.

Modern modal logic is said to be started by Clarence Irving Lewis. He set out
to develop a formal system without the paradoxes of materiel
implication. This led to his development of a wide range of different modal
logical system in the first part of the 20th century. His method was
syntactical, since there were yet to be develop a "smart" semantic for modal
logic. But in the start of the 20th century, C. I. Lewis had started the modern
modal logic, and a lot of other philosophers and logicians began to have an
interest in the subject.


In the 1950s there had been a lot of research into the syntax of modal logic.
But there was still not a clear definition of the semantics of modal logic,
even though philosophers and logicians still had the intuitive definition of
necessary true from Leibniz. The answer came in the late 1950s, when the
philosophers and logicians Jaako Hintikka
and Saul Kripke developed the concept of a Kripke model.\footnote{They were not
	alone in this development. But for simplicity sake \textbf{ja}}
Hintikka came up with the idea of a frame; i.e a pair $\la W,R\ra$ where $W$ is
a set of worlds and $R$ is an arbitrary relation on these. With this
construction it was possible to explain for a given world, what other worlds
was accessible to it; i.e possible worlds for it. Kripke came up with the same
idea, but added a valuation function to the frame, which to each formula of the
language ascribed a set of worlds in which it was true.  This construction made
it possible for Kripke to prove some completeness results about modal logics.

When the concept of the Kripke model became wider known, it started a
\textit{Golden} age of research in modal logic. A lot of completeness results
about different modal logic was proven in the 1960s and early 1970s; including
the weak completeness theorem for the logic that would later be known as
provability logic; this was done by Krister Segerberg without him having
knowledge of what the interpretation of this logic could be.

A deeper look into the development of modal logic can be found in
\parencite{Goldblatt2003}.
\subsection{Arithmetics and Recursion Theory}


Even though humans have been using arithmetics for millennials, it had not been
axiomatized.

It was first in  the second half of the 1800s there was a development in the axiomatization
of arithmetics. Richard Dedekind and Giuseppe Peano developed axioms system for arithmetics,
these was later evolved into was today is known as the Peano axioms for
arithmetics. 

Another important achievement at the turn of the century, was David Hilbert's
proof that the consistency of Euclidean geometry could be proven by proving
that arithmetics was consistent. \textbf{Hans geometri værk} This was one of the first steps in what later
would be known as the Hilbert program. 
The Hilbert program was not really a program until the end of 1920, where it
became clear that its goal was to prove that the \textit{ideal} transfinite mathematics was
consistent, and that the proof of this should be conducted in a \textit{real}
finite mathematics system.

An example of such a finite system could be the \textit{Primitive recursive
arithmetics} that was developed by Toralf Skolem; the definition of this system
will be given in the next chapter. This system can be seen as a
(induction-wise) weaker version of Peano aritmetics. Tait  argues
that such a system fulfills the finitist conditions of the Hilbert program.
This system will be the main system of this project and the definition of it
will come in the next chapter.

Another (mostly historically) important mathematical system of arithmetics was
the one developed by Bertrand Russell and Alfred North Whitehead in their work:
\textit{Principia Mathematica}. The goal of this work was somewhat different
than the one of the Hilbert program. Russell had a thesis that shortly (and not
the most precis) says that all of mathematics could be reduce to logic. In
Kantian terms he thought that mathematics was analytical \textit{a priori}. 

In 1931 Kurt Gödel's paper: \textit{Über formal unentscheidbare Sätze der
Principia Mathematica und verwandter Systeme I} it was shown  that the system of
Russell and Whitehead was not complete, and under the assumption that the
system was consistent that it could not prove its own consistency.
This result can be generalized to systems strong enough to do simple
arithmetic.
The
original proof was build on the system of Russel and Whitehead and had a
assumption of $\omega$-consistency. This
was a blow to the Hilbert program, since the result showed that the goal of it
could not be fulfilled. Later on John Barkley Rosser in his article
\parencite{Ross1936} improved the
result so that it  consistency was enough.

Later on in the 1930s Gödel extended the primitive recursive function to the  recursive
functions. A few different definitions of intuitively computable functions was
set forward by Kleene, Turing and Church and it was shown that all these
different notions of intuitively computable functions gave rise to the same
class of functions.

This lead Church to state the so called \textit{Church-Turing Thesis}, that
states that this class of functions are the computable functions.


\subsection{Provability Logic}
The first time that someone made a connection between modal logic and the proof
predicate of some form of arithmetic, was in 1933 when Kurt Gödel published his very short  paper: \textit{Eine
Interpretation des intuitionistischen Aussagenkalkuls}. Here Gödel
stated a hypothesis that it was possible to axiomatize the what the
intuitionist meant by a proof in a version of modal logic.

After Löb's derivability conditions was found and after the develpment of the
semantics of modal in the 1960s, it became clear that there was a connection
between the logic that Segerberg had proven to be weakly complete and the proof
predicate from arithmetics.

This lead to research done in different places around the from, from The
Netherlands, Italy and the United States. It was clear early on that if
$\alpha$ was a theorem of provability logic, then for all interpretations of
$\alpha$ in a fragment of arithmetics was also a theorem of this fragment; but
the other way was harder to show. In the same time there was proven a number of
important theorems about provability; one of these, called the Fixed Point
Theorem,  will be proven in chapter
\ref{chap:Fixed}.
A short overview of this early
development can be found in \parencite{Bool1991}.

It was proven in 1976 by Solovay that the modal
logic now known as provability logic, is the logic of the provability predicate
in Peano Arithemics; i.e if for all interpretations
of a formula $\alpha$, if this interpretation was a theorem of Peano
arithmetics,
then the formula $\alpha$ was a theorem of provability logic. This also means that
provability logic can be seen as an axiomataxion of the proof predicate.
Solovays result will be the main result in this project; it will be stated
precisely and proven in Chapter \ref{chap:Complete}.

After Solovay's article was published, the research focus of provability logic
was to generalize Solovay's result to other fragments of arithmetics and if
it did also hold for multi-modal provability logic and quantified provability
logic. The results of these results will be commented on in the last part of
chapter \ref{chap:Complete} and in chapter \ref{chap:Further}

The majority of the main results about provability logic were proven in the late
70s, 80s and early 90s. This project will not go beyond these early results.
\end{document}
