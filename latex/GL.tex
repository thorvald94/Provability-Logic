\documentclass[../main.tex]{subfiles}

\begin{document}

In this section we will prove a lot of different properties about \textbf{GL},
that will be used in the following chapters. 
Further in the end we will introduce the system $\text{\textbf{GLS}}$.

But first we will look at a more general modal logic called $\textbf{K4}$.

\subsection{The modal logic \Kfour}

We will start of by defining the modal logic \textbf{K}.

\begin{defi}
	The modal logic \textbf{K} is the modal logic that has the following
	axioms and rules of inferences:
	\begin{align*}
		&\text{Axioms} &A1:\ \text{All propositional tautologies.}\\
		& &A2:
	\end{align*}
\end{defi}
We can now define the modal logic \Kfour.

\begin{defi}
	\Kfour\ is \textbf{K} with the following axiom added:
	\[\Box\varphi\rightarrow\Box\Box\varphi\]
\end{defi}
We will start of by defining the system \textbf{GL}. 

\begin{defi} \textbf{GL} is the modal system with the following axioms and
	rules of inferences:
	\begin{enumerate}
		\item [A1] All propositional tautologies.
		\item [A2]
			$\Box\varphi\wedge\Box(\varphi\rightarrow\psi)\rightarrow\Box\psi$
		\item [A3]
			$\Box(\Box\varphi\rightarrow\varphi)\rightarrow\Box\varphi$
		\item [R1] $\varphi,\varphi\rightarrow\psi\vdash\psi$
		\item [R2] $\varphi\vdash\Box\varphi$
	\end{enumerate}
\end{defi}
Further a Kripke model $\mathcal{K}$ of \GL is a tuple such that
$\mathcal{K}=\la W,R,\phi\ra$ where $W=\{0,\ldots,n\}$ and where $R$ is a
conversely well-founded transitive relation on $W$. Further the valuation
function $\phi$ is defined in the usual way.
:
\section{Tress and \GL}

In [Tsss] it was shown that \textbf{GL} was complete with resect to conversely
well-founded transitive frames. We will first start of by defining the notion
of a tree.

\begin{defi}
	A tree is a tuple $(W,<,w_0)$ where $(W,<)$ where:
	\begin{enumerate}
		\item $<$ is transitive and asymmetric.
		\item $w_0$ is the minimal element of $<$. I.e $w_0
			<w$ for all $w\in W$.
		\item The set of predecessors of any element is finite and
			linearly ordered byh $<$.
	\end{enumerate}
\end{defi}
It is clear that finite trees are conversely well-founded frames. This fact is crucial
in the proof of the following theorem:
\begin{thm}
	Let $\varphi$ be a modal sentence. Then the followig are equivalent:
	\begin{enumerate}
		\item $\vdash_\GL\varphi$
		\item $\varphi$ is true in all models on finite tress
		\item $\varphi$ is valid in all models on finite tress
	\end{enumerate}
\end{thm}
\begin{proof}
	The implications $(1)\Rightarrow (2)$ and $(1)\Rightarrow(3)$ follows
	by the completeness theorem. We also have that $(2)\Leftrightarrow (3)$
	is true [Kom med et kort argument]. Thus we will just have to show $(2)\Rightarrow(1)$

We will show this by contraposition. Assume that $\vdash_{GL}\varphi$ and
let $\mathcal{K}$ be the following model $\mathcal{K}=\la W,R,w_0,\phi\ra $ be a counter model i.e
	$\not\vDash_{w_0}^{K'}\varphi$. The goal is now to define a finite tree
	model: $\mathcal{K}_T=\la W_T,<_T,\phi_T\ra$. We will do this be
	letting $W_T$ consists of finite $R$-increasing sequences form $K$.

	\textbf{Stage 0:} Let the sequence $w_0$ be a part of $K_T$.

	\textbf{Stage $n+1$:} For each sequence $(w_0,\ldots w_n)\in W_T$ we
	will look at $\Gamma=\{\Box\psi\in S(\varphi):\ \vDash_{w_n}\Box\psi\}$.
	If $\Gamma=\emptyset$ then we do not extend the sequence $(w_0,\ldots
	w_n)$. Otherwise we will for each $\Box\psi\in\Gamma$ choose a node
	$v\in W$ such that $w_0Rv$ and such that we have:
	$$\vDash_v\Box\psi,\ \not\vDash_v\psi$$
	We can do this because of axiom 3. We will then add the sequence
	$(w_0,\ldots w_n,v)$ to the set $W_T$

	The model $\mathcal{K}_T$ then further consists of $<_T$ is the strict
	ordering by extension of finite sequences. $\phi_T$ is the defined in
	the following way:
	$$(w_0,\ldots,w_n)\in \phi_T(p)\Leftrightarrow w_n\in\phi(p)$$
	This leads way to the following notation that we will use in the rest
	of the proof:
	$$\vDash_{(w_0,\ldots,w_n)}^{\mathcal{K}_T}p\Leftrightarrow\vDash_{w_n}^\mathcal{K} p$$
	We will now prove two claims, by which the theorem will follow 

	\textbf{Claim 1:} $(W_T,<_T,w_0)$ is a finite tree with origin $(w_0)$.
	It is clear that that is a finite tree with origin $(w_0)$.
	\textbf{Udfyld og forstå resten}

	\textbf{Claim 2:} For all $\psi\in S(\varphi)$ and for all $(w_0,\ldots
	w_n)\in W_T$ we have:
	$$\vDash^{\mathcal{K}_T}_{(w_0,\ldots,w_n)}\psi\Leftrightarrow\ \vDash_{w_n}^\mathcal{K}\psi$$
	This proof is done by induction on the complexity of $\psi$. We will
	only look at the case $\psi=\Box\theta$. So we have by the induction
	hypothesis:
	\begin{align*}
		\vDash_{w_0}\Box\theta&\Rightarrow\forall v(w_0Rv\Rightarrow\ 
		\vDash_v\theta)\\
				      &\Rightarrow\forall v((w_0,\ldots
				      w_n,v)\in W_T\Rightarrow\
				      \vDash_v\theta)\\
				      &\Rightarrow\forall v((w_0,\ldots
				      w_n,v)\in W_T\Rightarrow\
				      \vDash_{(w_0,\ldots w_n,v)}^{\mathcal{K}_T}\theta)
	\end{align*}
	And the last line is the same as $\vDash_{(w_0,\ldots,w_n)}\Box\theta$
	For the other way, we will use contraposition:
	\begin{align*}
		\not\vDash_{w_0}\Box\theta&\Rightarrow \exists v(w_0Rv\ \&\
		\not\vDash_v\theta)\\
					  &\Rightarrow\exists v((w_0,\ldots
					  w_n,v)\in W_T\ \&\
					  \not\vDash_v\theta)\\
					  &\Rightarrow\exists v((w_0,\ldots
					  w_n,v)\in W_T\ \&\
					  \not\vDash_{(w_0,\ldots,w_n,v)}^{\mathcal{K}_T}\theta)
	\end{align*}
	And the last line is the same as
	$\not\vDash_{(w_0,\ldots,w_n)}\Box\theta$. 

	The theorem now follows since we have:
	$$\not\vDash_{w_0}\varphi\Rightarrow\
	\not\vDash_{[w_0)}^{\mathcal{K}_T}\varphi$$
\end{proof}
\section{The system \text{\textbf{GLS}}}
In this section we will define the modal logic \textbf{GLS}. Before we can do
this in a smart way, we will define the notion of a $R2$ free axiomation; i.e a
system that is not closed under $R2$.

\subsection{Avoiding \textit{R2}}

\textbf{Metatekst}

We will need the following notation to
define this. Given a set $\Gamma$:
\begin{enumerate}
	\item $\Gamma\vdash\varphi$ if $\varphi$ is derivable form $\Gamma$ by
		only using $R1$
	\item $\Gamma\vdash_2\varphi$ if $\varphi$ is derivable form $\Gamma$ by
		using $R1$ and $R2$.
\end{enumerate}
We will further define the following set for any set $\Gamma$:
\[
	\Gamma^2=\Gamma\cup\{\Box\psi:\psi\in\Gamma\}
\]
This set allows us to avoid $R2$. Why?
We can now state and prove the following lemma:
\begin{lem}
	Let $\Gamma$ include all the instanceses of $A1-A3$. THen:
	\[
		\Gamma\vdash_2\varphi\Leftrightarrow\Gamma^2\vdash\varphi
	\]
\end{lem}
\begin{proof}
	The way $\Leftarrow$ is trivial. So we will only show $\Rightarrow$.
	To prove this, it is enough to show that the set of theorems of
	$\Gamma^2$ is closed under $R2$; i.e to show that:

	$$\Gamma^2\vdash\varphi\Rightarrow\Gamma\vdash\Box\varphi$$

	Let $\Gamma^2\vdash\varphi$. Then we have a sequence
	$\varphi_1,\ldots,\varphi_n=\varphi$, where for each $1\leq i\leq n$ we
	have that $\varphi_i$ is one of the following three:
	\begin{enumerate}
		\item We have that it is an axiom $\varphi_i\in\Gamma$ 
		\item It is the necessitation  $\Box\psi_1$ of an axiom
			$\psi_1\in\Gamma$
		\item It is a consequence of $R1$ of
			$\varphi_j,\varphi_k=\varphi_j\rightarrow\varphi_i$
			where $j,k<i$.
	\end{enumerate}
	We will show  by induction on the lenght $i$ of a 'intial' segment
	$\varphi_1,\ldots,\varphi_i$ that $\Gamma^2\vdash \Box\varphi_i$. 
	\begin{enumerate}
		\item If we have that $\varphi_i\in\Gamma$. Then by definition
			we have that $\Box\varphi_i\in\Gamma^2$ and thus
			$\Gamma^2\vdash\Box\varphi_i$.
		\item If $\varphi_i=\Box\psi_i$ for some $\psi_i\in\Gamma$,
			then we can make the following deduction:
			\begin{align*}
				\Gamma^2&\vdash\Box\psi_i &\text{Since}\
				\Box\psi_i\in\Gamma^2\\
					&\vdash\Box\psi_i\rightarrow\Box\Box\psi_i
					&\text{By}\ A3\\
					&\vdash\Box\Box\psi_i\ (=\Box\varphi_i) 
					&\text{By}\ R1
			\end{align*}
			Which shows the result.
		\item If $\varphi_i$ follows form
			$\varphi_j,\varphi_k=\varphi_j\rightarrow\varphi_i$ by
			$R1$ we have by the induction hypothesis that
			$\Gamma^2\vdash\Box\varphi_j$ and
			$\Gamma^2\vdash\Box(\varphi_j\rightarrow\varphi_i)$. By
			$A2$ we also have that
			$\Gamma^2\vdash\Box\varphi_j\wedge\Box(\varphi_j\rightarrow\varphi_i)\rightarrow\Box\varphi_i$
			and by propositional logic we get that:
			$\Gamma^2\vdash\Box\varphi_i$.
	\end{enumerate}
	This ends the induction and the lemma will follow.
\end{proof}
By this lemma we can get a $R2$ free definition of \GL that we will call
$\GL^2$. We can further by the deduction theorem from propositional logic get
the following:

\begin{lem}
	$\Gamma\vdash\varphi$ iff there is a finite set
$\{\theta_0,\ldots,\theta_{n-1}\}\subseteq \Gamma$ such that:
$K\vdash\wedge\theta_i\rightarrow\varphi$
\end{lem}

\subsection{Definition of \textbf{GLS}}
\begin{defi}
	The system of modal logic \text{\textbf{GLS}} is defined as the system
	of propositional modal logic that have all the theorems of $\GL$ and
	 all the sentences $\Box\varphi\rightarrow\varphi$ as its axioms and which sole
	rule of inference is \textit{modus ponens}. 
\end{defi}
It should be noted that this modal logic is not a \textit{normal} one.
The main theorem about this modal logic is the following:
\begin{thm}
	\label{thm:MainGLS}
	Let $\varphi$ be a modal sentences. Define the following set: $S_\Box(\varphi)=\{\Box\psi:\
	\Box\psi\ \text{ is a subformula of}\ \varphi\}$ the set of
	subformulas of $\varphi$ that is boxed. Then the following to
	statements are equivalent:
	\begin{enumerate}
		\item $\vdash_\GLS\varphi$
		\item $\vdash_\GL\bigwedge_{\Box\psi\in
			S_\Box(\varphi)}(\Box\psi\rightarrow\psi)\rightarrow\varphi$
	\end{enumerate}
\end{thm}

\textbf{Decidudud???}

\begin{defi}
	A model $\mathcal{K}=\la W,R,\phi\ra$ with start node $w_0$ is $\varphi$-sound if for every
	$\Box\psi\in S_\Box(\varphi)$  we have:
	$\vDash_{w_0}\Box\psi\rightarrow\psi$
\end{defi}
We can reduce theorem \ref{thm:MainGLS} to the following:
\begin{thm}
	\label{thm:lll}
	Let $\varphi$ be a modal sentences. Then:
	\begin{enumerate}
		\item $\vdash_\GLS\varphi$ iff $\varphi$ is true in all
			$\varphi$-sound models of $\GL$
		\item $\vdash_\GL\varphi$ iff $\varphi$ is valid in all
			$\varphi$-sound models of $\GL$
	\end{enumerate}
\end{thm}
\begin{proof}[Proof that \ref{thm:lll}\ \Rightarrow\ \ref{thm:MainGLS}:]
	Assume \ref{thm:lll}.1. We hen have that $\vdash_\GLS\varphi$ iff $\varphi$ is true in all
	$\varphi$-sound models of $\GL$; which is the same all models of
	$\GL+\bigwedge_{\Box\psi\in S_\Box(\varphi)}(\Box\psi\rightarrow\psi)=\GL^s$ and hence:
	$$\vdash_\GLS\varphi\Leftrightarrow\vdash_{\GL^s}\varphi$$
\end{proof}

For the proof of \ref{thm:lll} we will make use of the following usefull
construction:

\begin{defi}
	Let $\mathcal{K}=\la W,R,\phi\ra$ be a Kripke Model. The derived model
	$\mathcal{K'}$ is defined as follows: $W'=W\cup \{w'\}$, where
	$w'\not\in W$. We further have that $R'$ is defined as $wR'v$ iff $WRv$
	for $w,v\in W$; $w'$ is the new minimum element. Lastly we have that
	$w\in\phi'(p)$ iff $w\in\phi(p)$ for $w\in K$, where $p$ is any atomic
	formula. Further we have that $w'\in\phi'(p)$ iff $w_0\in\phi(p)$.
\end{defi}
We can further define the notion of a sequences of successive derived models:
$\mathcal{K}^{(1)},\mathcal{K}^{(2)},\ldots$ that have $w_1,w_2,\ldots$ as
their minima. We define $\mathcal{K}^{(1)}=\mathcal{K}$ and
$\mathcal{K}^{(n+1)}=(\mathcal{K}^{(n)})'$ and let $w_n$ denote the minimum of
$\mathcal{K}^{(n)}$. This can graphically been seen as the following:

\textbf{Indfør tegninger}

It should be noted that for $w\in\mathcal{K}$ and for any sentence $\theta$
that we have:
$$\vDash_w\theta\ \text{in}\ \mathcal{K}\Leftrightarrow\ \vDash_w^{(n)}\theta\
\text{in}\ \mathcal{K}^{(n)}$$
For the rest of this project we will omit the annoying superscripts.

The following lemma will explain a bit deeper how these derived models works:

\begin{lem}
	\label{lem:GLS}
	Let $\mathcal{K}$ be a $\varphi$-sound model, $S(\varphi)$ the set of
	subformulas of $\varphi$ and $\mathcal{K}^{(n)}$ be the $n$-derived
	model. Then for all $\psi\in S(\varphi)$ we have:
	$$\vDash_{w_0}\psi\Leftrightarrow\vDash_{w_n}\psi$$
\end{lem}
\begin{proof}
	We will prove this by induction on $n$. The lemma will follow from the
	case $n=1$ [Why?], and this will be proven by induction om complexity
	of $\psi$. The only non trivial case is the case where
	$\psi=\Box\theta$, so this is the only case that will be shown:

	Let $\psi=\Box\theta\in S(\varphi)$. 

	\textbf{($\Rightarrow$)}\ The following is clear for
	$\Box\theta\in S(\varphi)$:
	$$\vDash_{w_0}\Box\theta\Rightarrow\vDash_{w_0}\theta$$
	This follows since our model is $\varphi$-sound. Therefore we have:
	$\forall v>w_1(\vDash_{w_1}\theta)$ and thus $\vDash_{w_1}\Box\theta$

	\textbf{($\Leftarrow$)} Here we can make the following deduction:
	\begin{align*}
		\vDash_{w_1}\Box\theta&\Rightarrow\forall
		v>w_1(\vDash_v\theta)\\
				      &\Rightarrow\forall
				      v>w_0(\vDash_v\theta)\\
				      &\Rightarrow\vDash_{w_0}\Box\theta
	\end{align*}
\end{proof}
We can now return to our proof of the main theorem of this section by proving
\ref{thm:lll}
\begin{proof}[Proof of \ref{thm:lll}]
	We will start of by proving (1). One way is easy:

	\textbf{($\Leftarrow$)} If $\varphi$ is true in all $\varphi$-sound
	models $\mathcal{K}$ of $\GL$ then:
	$$\vdash_{\GL^s}\varphi$$
	And therefore we have $\vdash_\GLS\varphi$.

	\textbf{($\Rightarrow$)} This way takes a bit more work; it will be
	proven by contraposition. So suppose that $\varphi$ is false in
	$\varphi$-sound model $\mathcal{K}=\la W,R,\phi\ra$. Then we have:
	$$\vDash_w\bigwedge_{\Box\psi\in
	S_\Box(\varphi)}(\Box\psi\rightarrow\psi)$$
	But still also have: $\not\vDash_{w_0}\varphi$. We will show that
	$\not\vdash_\GLS\varphi$ by showing that for any finite set $\Gamma$
	that for the system
	$\GL^\Gamma=\GL+\bigwedge_{\theta\in\Gamma}(\Box\theta\rightarrow\theta)$
	that we have:
	$$\not\vdash_{\GL^\Gamma}\varphi$$
	By lemma \ref{lem:GLS} we have that every  model in the sequence
	$\mathcal{K}^{(0)},\mathcal{K}^{(1)},\ldots$ is an $\varphi$-sound
	model, wherein $\varphi$ is false. We will just have to show that for
	any finite set $\Gamma$ that is a number $m$ such that:
	$$\vDash_{w_m}\bigwedge_{\theta\in\Gamma}(\Box\theta\rightarrow\theta)$$
	This can be shown by showing that there is a $m$ such that for each
	$n>m$ that the following holds:
	$$\vDash_{w_n}\Box\theta\rightarrow\theta$$
	But we either have that $\vDash_{w_m}\Box\theta$ for all $n$, thus
	$\vDash_{w_n}\theta$ for all $n$; or we have
	$\not\vDash_{w_m}\Box\theta$ for some $m$ and thus
	$\not\vDash_{w_n}\Box\theta$ and
	$\vDash_{w_n}\Box\theta\rightarrow\theta$ for al $n>m$. This shows (1).
	We will now prove (2).\textbf{Bevis dette i hånden}
	
	The implication $\Rightarrow$ is trivial. 
\end{proof}
\end{document}
